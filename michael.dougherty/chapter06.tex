%\setcounter{chapter}{6}
%\setcounter{page}{700}
\chapter{Basic Integration\label{FirstIntegrationChapter}}

In this chapter we will consider the problem of recovering 
a function from knowledge of its derivative, or, equivalently,
for a given function we will try to find another function
whose derivative is the given function's.  The general process
is called  {\it antidifferentiation}, or
{\it integration}.  The meaning of the first term is obvious:
we are working backwards from the  derivative to the function.
The meaning of the other name for the process
is mainly left for Chapter~\ref{AppsDefIntegrals}.

The main purpose of this chapter is to develop the first, basic
techniques for computing antiderivatives $F(x)$ for a given
function $f(x)$, i.e., given $f(x)$ we look for $F(x)$ so that
\begin{equation}
F'(x)=f(x).\label{IntroducingF(x)}\end{equation}
As we will see early in this chapter and throughout the next, 
antidifferentiation (finding some such $F(x)$), a.k.a.\ integration,
is less straightforward than differentiation (finding $f'(x)$).  
However, there are easily as many applications of antidifferentiation 
as there are of differentiation so it is a worthwhile process.  
In the first section we will limit ourselves to two applications:
\begin{enumerate}
\item Given the slope $f'(x)$, find the function $f(x)$
      by antidifferentiation;\footnote{%
%%% FOOTNOTE
In this application, the part of $f(x)$ in (\ref{IntroducingF(x)})
is played by $f'(x)$, while the part of $F(x)$ is played by $f(x)$.}
%%% END FOOTNOTE

      Moreover, given $f''(x)$, find $f'(x)$, and then $f(x)$.
\item Given velocity $v$, find position $s$. 

      Moreover, given
      acceleration $a$, find $v$ and then $s$.
\end{enumerate}
Later in the chapter
we will look at the geometric
significance of antiderivatives $F(x)$ of a function $f(x)$.  
Just as the geometric meaning of 
slope gave us a useful perspective for arriving at derivative theorems
(mean value theorem, first derivative test, etc.), so too will the
antiderivatives benefit from geometric analysis.  To make
that analysis will require us to consider another major theoretical device,
namely {\it Riemann sums}, which---together with the 
Fundamental Theorem of Calculus---will open the topic of integration
to innumerable applications.  To illustrate the reasonableness of
the Fundamental Theorem of Calculus, we will again look closely at the
velocity-position connection as well, in
Chapter~\ref{AppsDefIntegrals}.


In Chapter~\ref{AdvancedIntTechniques}
we will develop more advanced techniques of
antidifferentiation, so that we can use all of our integration
techniques in Chapter~\ref{AppsDefIntegrals}, which
is devoted to applications.

For now we will concentrate on the actual computation of
antiderivatives of the more basic types.
In the first section we will limit ourselves to those 
which arise from our known derivative formulas.  In subsequent
sections we will then explore the substitution technique, which
is the antidifferentiation analog to the chain rule
and is thus arguably the most important
of the integration techniques.  It will be developed at length.  



%In Chapter~\ref{DerivativeChapter} we described how we can take
%a function such as position $s(t)$, and---using a particular limit
%which was the derivative---compute the instantaneous rate of change in
%the position with repsect to time $t$, that instantaneous
%rate of change for that case being {\it velocity} $v(t)$.  
%We did this by looking
%at the net change in the function over smaller and smaller intervals,
%such as $[t,t+\Delta t]$ if $\Delta t>0$, and $[t+\Delta t,t]$
%if $\Delta t<0$, and dividing by how long it took for
%those changes to occur (i.e., the directed lengths of the intervals)
%to get average rates of change of $s$ with respect to $t$ on
%those intervals. Then we let the lengths of the intervals approach
%zero in the limit. 
%Thus what we computed was
%the limit---wherever it existed---of difference quotients:
%\begin{equation}
%\frac{s(t+\Delta t)-s(t)}{\Delta t} \longto s'(t)=v(t)
%\qquad\text{as}\qquad\Delta t\to0.\end{equation}
%Because we started with a function and used quotients of {\it differences}
%to find its instantaneous rate of change, the process was called
%{\it differentiation} (and for each $t$ at which the limit 
%existed, we called $s(t)$ {\it differentiable}).

%Next we will be interested in recovering the net change
%in the function over intervals---for instance
%the net change of $s(t)$ over the
%interval $t\in\left[t_0,t_f\right]$, namely $s(t_f)-s(t_0)$---from
%knowledge of the derivative $s'(t)$ over that interval.  
%In other words, for this context we are interested in recovering
%the total change in position, over some time interval, 
%from knowledge of the velocity at each time in the interval.
%This process of somehow
%accumulating all these instantaneous changes $s'(t)$ into the total change
%$s(t_f)-s(t_0)$ is, naturally enough,
%called  {\it integration} (as opposed to {\it differentiation}).

%Perhaps predictably, so-called {\it antiderivatives} will 
%be central to our discussion
%here and in subsequent chapters.  In this chapter, we will first
%develop the idea of antiderivatives, and then pursue
%the most basic methods of computing them.
%We will then explore Riemann Sums, and how they arise in
%very natural ways.  
%For the climax of the chapter, we will see how the Fundamental Theorem of 
%Calculus connects these things.\footnote{At this point it 
%can be useful to re-read the introduction to Chapter~\ref{DerivativeChapter}
%to recall the previous development to this point.}
\newpage
\section{First Indefinite 
Integrals (Antiderivatives)\label{IndefiniteIntegrals}}
In this section we introduce {\it antiderivatives}, which are 
exactly what the name implies.  These are also called
{\it indefinite integrals} for reasons which will eventually
become clear.
\subsection{Indefinite Integrals and Constants of Integration}

\begin{definition}Consider a function $f(x)$ which is defined on an open
interval $(a,b)$.  Another function $F(x)$, also defined
on $(a,b)$ is called an {\bf antiderivative} of $f(x)$ on
the same interval if and only if $F'(x)=f(x)$ on $(a,b)$.

If instead $f(x)$ is defined on a closed interval $[a,b]$, we still call
$F:[a,b]\longrightarrow\Re$ an {\bf antiderivative} of $f(x)$
on $[a,b]$ if and only if 
\begin{align*}
F'(x)&=f(x),\qquad x\in(a,b),\\
\lim_{\Delta x\to0^+}\frac{F(a+\Delta x)-F(a)}{\Delta x}&=f(a),\\
\lim_{\Delta x\to0^-}\frac{F(b+\Delta x)-F(b)}{\Delta x}&=f(b).
\end{align*}
\end{definition}
In other words, on an open interval we require $F'(x)=f(x)$,
while on the closed interval we also require the right derivative
of $F$ to be $f$ at $x=a$, and the left derivative of $F$ to be 
$f$ at $x=b$.

Notice that the definition implies that $F(x)$ is continuous
on the interval in question
(since $F'=f$ exists implies continuity of $F$).  
For a simple example, consider $f(x)=2x+3$ on any open (or nontrivial closed)
interval. An antiderivative of $f(x)$ can be
$F(x)=x^2+3x$, since then $F'(x)=2x+3=f(x)$.  
However, another perfectly good antiderivative
can be $F(x)=x^2+3x+5$, or $F(x)=x^2+3x-100,000$, since the
derivative of the trailing constant term will always be zero.
In logical terms we can write
\begin{align}
F(x)=x^2+3x\implies& F'(x)=2x+3,\notag\\
F(x)=x^2+3x+5\implies& F'(x)=2x+3,\notag\\
F(x)=x^2+3x-100,000\implies& F'(x)=2x+3,\notag\\
F(x)=x^2+3x+C,\text{ some }C\in\Re\iff& F'(x)=2x+3.
\label{FirstAntiderivative}
\end{align}
That the last one (\ref{FirstAntiderivative})
is an equivalence we will prove shortly.
To signify that equivalence, we write
\begin{equation}
\int (2x+3)\,dx=x^2+3x+C,\qquad C\in\Re.\label{EasyIndefIntegEx}\end{equation}
We call the right hand side of (\ref{EasyIndefIntegEx})
{\it the most general antiderivative}, or just
{\it the antiderivative,} of $2x+3$ (with respect to $x$). It is 
also called the {\it indefinite integral} 
of $2x+3$ (again with respect to $x$), and we will
eventually migrate to using that term as our default.\footnote{%
%%%%%% FOOTNOTE
The indefinite integral has a strong connection to the 
very important {\it definite integral, }which is a measure
of accumulated change as computed from the instantaneous
rate of change.  This computation is our eventual goal and---to
restate the introduction to this chapter---the connection 
between antiderivatives (indefinite integrals)
and accumulated change (definite integrals) is precisely the 
subject of the Fundamental Theorem of Calculus.}
%%%%%% END FOOTNOTE
The constant
$C$ is called the {\it constant of integration},
since it must be included to achieve all solutions to the question
of what is an antiderivative of $f(x)=2x+3$.
On the left hand side of (\ref{EasyIndefIntegEx})
we have
\begin{itemize}
\item $\ds{\int}$: the {\it integral sign}, derived from a old style German 
           S for reasons we will discuss later;
\item $(2x+3)$: the {\it integrand}, whose antiderivatives we seek; and
\item $dx$: the differential of $x$, which signifies which variable
            the antiderivative is with respect to.
\end{itemize}
We now note that, indeed, all antiderivates of $2x+3$
are necessarily of the form $x^2+3x+C$.  To prove this, on some interval
define $F(x)=x^2+3x$, which we can easily see is an antiderivative
of $2x+3$ on that interval.
Next suppose $G(x)$ is another such antiderivative,
i.e., that $G'(x)=2x+3$, on that same interval.  Then on that interval
$$\frac{d}{dx}\left[G(x)-F(x)\right]
=G'(x)-F'(x)=(2x+3)-(2x+3)=0
\implies G(x)-F(x)=C,$$
for some $C\in\Re$.  Thus any antiderivative $G(x)$
must be of the form $G(x)=F(x)+C$, q.e.d.\footnote{%%%%%
%%%%% FOOTNOTE
Recall that a function with the zero function for its derivative
on an interval
must be constant on that interval: 
$$(\forall x\in I)[(F(x)-G(x))'=0]\implies (\exists C\in\Re)
      (\forall x\in I)[F(x)-G(x)= C].$$}
%%%%  END FOOTNOTE
To be clear on the notation, we now insert the following definition.
\begin{definition}
If $F(x)$ is an antiderivative of $f(x)$, with respect to $x$,
on the interval $I$, then on that interval we write
\begin{equation}
\int f(x)\,dx=F(x)+C,\label{DefIndefIntegral} 
\end{equation}
where $C$ is an arbitrary constant of integration.
\end{definition}
\bex Consider $f(x)=2\sin x\cos x$.  One antiderivative
is $F(x)=\sin^2x$, since
$$F'(x)=\frac{d\sin^2x}{dx}=\frac{d(\sin x)^2}{dx}
=2\sin x\cdot\frac{d\sin x}{dx}=2\sin x\cos x.$$
However, another antiderivative is $G(x)=-\cos^2x$, since
$$G'(x)=\frac{d(-\cos^2x)}{dx}
=-\,\frac{d(\cos x)^2}{dx}
=-\left[2\cos x\cdot\frac{d\cos x}{dx}\right]
=-\left[2\cos x(-\sin x)\right]=2\sin x\cos x.$$
Note that
$$F(x)-G(x)=\sin^2x-\left(-\cos^2x\right)=\sin^2x+\cos^2x=1,$$
so we see that $F$ and $G$ do actually differ by a constant.
To report the most general antiderivative of $f(x)=2\sin x\cos x$,
either of the following are valid (but understood to 
have different ``$C$'s''):
\begin{align*}
\int2\sin x\cos x\,dx&=\sin^2x+C,\\
\int2\sin x\cos x\,dx&=-\cos^2x+C.
\end{align*}
\eex

Especially when dealing with trigonometric functions---and
all their interconnectedness by various identities---it is 
common to find very different-looking forms
of the general antiderivative, all of which differ by
constants from each other. It is occasionally important to be 
alert for apparent discrepancies which are explained 
by this nature of the general antiderivative.

\bex Suppose $f(x)=x+1$.  Then both forms below are
general antiderivatives:
\begin{align*}
\int(x+1)\,dx&=\frac12x^2+x+C,\\
\int(x+1)\,dx&=\frac12(x+1)^2+C.
\end{align*}
We can see this by taking derivatives of each.  We can also see
that if we label $F(x)=\frac12x^2+x$ and $G(x)=\frac12(x+1)^2$, 
so the antiderivatives above are just $F(x)$ and $G(x)$ plus
constants, respectively, and then compute
$$F(x)-G(x)=\left[\frac12x^2+x\right]-\left[\frac12\left(x^2+2x+1
   \right)\right]=-\,\frac12,$$
so these do differ by a constant, as expected.
\eex


\subsection{Power Rule for Integrals}
Where the rules for computing derivatives were
straightforward (which is not to say necessarily ``easy''),
those for computing antiderivatives are not so algorithmic.
Indeed the methods are varied.  Nonetheless, they are necessary
to learn for a reasonably complete understanding of standard
calculus, and we begin with the {\it power rule for integrals:}
\begin{align}
\int x^n\,dx&=\frac1{n+1}x^{n+1}+C,\qquad n\ne-1,\label{IntegralPowerRule}\\
\int \frac1x\,dx&=\ln|x|+C.\label{PwerRuleN=-1}
\end{align}
Here the intervals in question are those upon which $x^n$ is
defined.  To check (\ref{PwerRuleN=-1}), we simply
notice $\frac{d}{dx}\ln|x|=\frac1x$.
For (\ref{IntegralPowerRule})
we compute:
$$\frac{d}{dx}\left[\frac1{n+1}x^{n+1}\right]
=\frac1{n+1}\cdot(n+1)x^{[(n+1)-1]}=x^n,\qquad\text{q.e.d.}$$
In performing the above computation, 
we must notice that $n+1$ and $1/(n+1)$ are constants,
and so are preserved throughout the computation (and do
not, for instance, require product/quotient/chain rules since they do not
vary).
We also note that the formula to the right of (\ref{IntegralPowerRule})
is meaningless for the case $n=-1$, so we need
(\ref{PwerRuleN=-1}) for that case.  

When checking any antiderivative by differentiation, it is customary to 
not include the arbitrary additive constant, since its derivative is zero.
However, it is certainly correct to include it, as in
$\frac{d}{dx}\left[\ln|x|+C\right]=\frac1x+0=\frac1x$.


We can apply the power rule immediately
as in the following:
\begin{align*}
\int x^2\,dx&=\frac13x^3+C,\\
\int x^3\,dx&=\frac14x^4+C,\\
\int\frac1{x^2}\,dx&=\int x^{-2}\,dx=\frac1{-1}x^{-1}+C
=-x^{-1}+C=\frac{-1}x+C,\\
\int\frac{x}{x^2}\,dx&=\int\frac1x\,dx=\ln|x|+C.
\end{align*}
We can check all of these by taking derivatives of our answers,
with respect to $x$ (i.e., by applying $d/dx$).\footnote{
%%%%%%  FOOTNOTE
In fact, it is a very useful exercise to check these antiderivatives
by quick mental calculations.  Since the derivative formulas have
been used extensively to this point, the processes of computing
antiderivatives can be well-informed by their connections
to known derivative techniques.  In particular, mistakes in 
computing antiderivatives can often be immediately corrected.
Perhaps even more importantly, the approximate form of an antiderivative
can often be anticipated.  For instance, we know that the derivative
of a fourth-degree polynomial is necessarily a third-degree polynomial.
It should be clear (though some argument is necessary to prove) that
the antiderivative of a third-degree polynomial is necessarily a 
fourth-degree polynomial.  Anticipating the final form is also
very useful in substitution problems, which are introduced in 
Section~\ref{FirstSubstitutionSection} and ubiquitous thereafter.
%%%%%  END FOOTNOTE
}
As with derivatives, many functions which are powers of the variable are
not explicitly written as such.  Furthermore as with derivatives,
the variable name in antiderivative formulas does not matter as long as 
it is matched in the differential:
\begin{align*}
\int t^9\,dt&=\frac1{10}t^{10}+C,\\
\int \sqrt{u}\,du&=\int u^{1/2}\,du=\frac1{\frac12+1}u
             {\vphantom{X^X}}^{\left[\frac12+1\right]}
    +C=\frac23u^{3/2}+C,\\
\int\frac{1}{z\sqrt[4]z}\,dz&=\int z^{-5/4}\,dz
   =\frac1{\frac{-5}4+1}z{\vphantom{X^X}}^{\left[\frac{-5}4+1\right]}+C
   =\frac{z^{-1/4}}{-\frac14}+C=\frac{-4}{\sqrt[4]{z}}+C.
\end{align*}
To check these, we can apply respectively $d/dt$, $d/du$ and
$d/dz$.  
Before we go on we note that (\ref{IntegralPowerRule}),
intepretted formally, applies to the case $n=0$ as well:
$\int x^0\,dx=\frac{1}{0+1}x^{0+1}+C$, i.e.,
\begin{equation}\int 1\,dx=x+C.\label{IntegralOfOne}\end{equation}
Taking the derivative of the right-hand side of 
(\ref{IntegralOfOne}) quickly shows it is in fact true.
This integral is often abbreviated
\begin{equation}\int\,dx=x+C.\label{IntegralOf_dx}\end{equation}
As with the chain rule computations, the differential $dx$
is formally treated as a factor as in the following:\label{UsingdxAsAFactor}
$$\int\frac{dx}{x^3}=\int x^{-3}\,dx=\frac{1}{-2}x^{-2}+C
=\frac{-1}{2x^2}+C.$$
Later we will see this treatment of $dx$ justified and exploited
in several contexts.

Now we state two very general results which may seem obvious,
but are worth exploring with some care because of a technical
consideration regarding the constant of integration.\footnote{%
%%%%  FOOTNOTE
Many calculus students become careless about this constant of integration
and just ``tack it on the end'' when computing antiderivatives.
However, its correct placement is crucial
in several contexts, so it is useful to be vigilant from the beginning
of integration study.  Carelessness
in its placement can cause trouble already in this section,
but will be particularly troublesome in subsequent third-semester
calculus and in differential equations studies.} %
%%%%%%%%%% END FOOTNOTE
These are also useful to us immediately because they allow to 
use the power rule multiple times to compute the derivatives of
polynomials.
\begin{theorem}
Suppose that $F'(x)=f(x)$ and $G'(x)=g(x)$ on some interval in 
consideration, and that $k\in\Re$ is a fixed constant.  Then
 on that same interval we have
\begin{align}
\int\left[f(x)+g(x)\right]\,dx&=F(x)+G(x)+C,\label{IntegralOfASum}\\
\int\left[k\cdot f(x)\right]\,dx&=kF(x)+C.\label{k*Integral}\end{align}
where $C\in\Re$ is a constant of integration.
\end{theorem}
These can be proved by taking derivatives.
For instance, $\frac{d}{dx}[kF(x)]=k\cdot\frac{d}{dx}F(x)=kF'(x)=kf(x)$.
 Note that this theorem can  be rewritten
\begin{align}
\int\left[f(x)+g(x)\right]\,dx&=\int f(x)\,dx+\int g(x)\,dx,
   \label{AltIntegralOfSum}\\
\int kf(x)\,dx&=k\int f(x)\,dx,\qquad k\ne 0,\label{Alt_k*Integral}\\
\int 0\,dx&=C.\label{IntegralOfZero}\end{align}
A moment's reflection shows that  we do need (\ref{Alt_k*Integral})
and (\ref{IntegralOfZero}) to catch both cases summarzied in
(\ref{k*Integral}), else we lose the constant of integration if
we let $k=0$ in (\ref{Alt_k*Integral}).  
(Note that, indeed, $\frac{d}{dx}C=0$ which
verifies (\ref{IntegralOfZero}).)  More importantly, these new
forms (\ref{AltIntegralOfSum})--(\ref{IntegralOfZero})
are not inconsistent with those of the theorem when
we consider the arbitrary constants.  For instance,
if we assume $F'(x)=f(x)$ and $G'(x)=g(x)$ as in the theorem,
then we can write (\ref{AltIntegralOfSum}) as follows:
\begin{align*}\int f(x)\,dx+\int g(x)\,dx
&=\left(F(x)+C_1\right)+\left(G(x)+C_2\right)\\
&=F(x)+G(x)+\underbrace{\left(C_1+C_2\right)}_{\text{``$C$''}}
=F(x)+G(x)+C,\end{align*} 
where $C_1$ and $C_2$ are arbitrary constants, and their sum
will also be an arbitrary constant which we can name $C$.

With the above theorem and the power rule, we can now
compute the indefinite integrals of polynomials and other
linear combinations of powers.
\bex Consider the following integrals:

\begin{enumerate}[(a)]
\item $\ds{\int\left(4x^2-9x+7\right)\,dx
  =4\cdot\frac{x^3}{3}-9\frac{x^2}{2}+7\cdot x+C
  =\frac43x^3-\frac92x^2+7x+C}$,
\item $\ds{\int\left(\frac{3x-9}{x^3}\right)\,dx
=\int\left(3x^{-2}-9x^{-3}\right)\,dx
=-3\cdot\frac{x^{-1}}{-1}-9\cdot\frac{x^{-2}}{-2}+C
=\frac3x+\frac9{2x^2}+C}$,
\item$\ds{\int\left(5x^2\right)^3\,dx=\int125x^6\,dx=125\cdot\frac{x^7}7+C}$,
\item$\ds{\int\left(x^2+7\right)^2\,dx
=\int\left(x^4+14x^2+49\right)\,dx
=\frac{x^5}5+14\cdot\frac{x^3}3+49x+C}$,
\item$\ds{\int\sqrt{2x}\,dx=\int\sqrt2\sqrt{x}\,dx
=\int2^{1/2}x^{1/2}\,dx
=2^{1/2}\cdot\frac{x^{3/2}}{3/2}+C
=\sqrt{2}\cdot\frac23x^{3/2}+C}$.
\end{enumerate}
\eex
When we use an integration rule such as the power rule for
integrals,  as with derivatives it is important
that the variable which the antiderivative is with respect to 
matches the variable in the function.  For instance,
\begin{align}
\int x^3\,dx&=\frac14x^4+C,\\
\int w^3\,dw&=\frac14w^3+C,\\
\int (5x-11)^3\,dx&\ne\frac14(5x-11)^4+C.\label{BadIntegral1}
\end{align}
What goes wrong in (\ref{BadIntegral1}) is the integral analog
to what goes wrong  below (which is that we need the chain rule
to make the variables of differentiation match):
\begin{align*}
\frac{d}{dx}\left[(5x-11)^4\right]&\ne4(5x-1)^3;\\
\frac{d}{dx}\left[(5x-11)^4\right]
&=4(5x-11)^3\cdot\frac{d(5x-11)}{dx}=4(5x-11)^3\cdot5\ne4(5x-11)^3,\end{align*}
The problem is that the differential, $dx$, is that of
$x$ and not $(5x-11)$.  In the next section we will address
a kind of integral version of the chain rule (commonly known
as {\it integration by substitution} for reasons which
will be clear later), which would make short work of this
integral.  Without it, we may need to expand $(5x-11)^3$
as a polynomial, or guess the solution and check that it works,
and possibly make adjustments.
Either way, it should suffice to 
point out that we need to take care in using integral formulas
such as the integral power rule,
page~\pageref{IntegralPowerRule}.\footnote{%
%%% FOOTNOTE
Without giving away the substitution technique, we will note here
that we can rewrite the integral in (\ref{BadIntegral1})
so the differential matches the term $(5x-11)$.  The argument
would be the analog of our early chain rule expansions, such as
(\ref{ChainRuleDecompFor(x^3+1)^2}), 
page~\pageref{ChainRuleDecompFor(x^3+1)^2}.
The idea is that $d(5x-11)=5dx$ (recall the meaning of $d(5x-11)/dx$),
and so $dx=d(5x-11)/5$, which allows us to rewrite the integral
as follows:
$$\int(5x-11)^3\,dx=\int(5x-11)^3\frac{d(5x-11)}{5}
  =\frac15\cdot\frac14\cdot(5x-11)^4+C=\frac1{20}(5x-11)^4+C.$$
Except for the factor of $\frac15$ in the second interval, 
that rewriting had an integral power rule form.

Our integration by substitution method will be more systematic
than the above computation.  Consequently it will read
better, and be less error-prone.  That method
will then be  called upon extensively from then on.}
 




\subsection{Finding $C$}
Many times we are interested in a particular antiderivative.
Since all antiderivatives (on a particular interval) differ
by a constant, this means we need to find one antiderivative,
and then try to ``fix,'' i.e., determine, the  constant.  
\bex Find $f(x)$ so that $f'(x)=2x$ and $f(3)=7$.

\underline{Solution}: For a problem such as this, it is common
to write
$$f(x)=\int f'(x)\,dx,$$
where it is understood that we will eventually find the exact antiderivative
so that the function is well-defined.
For our particular problem, one might continue to write
$$f(x)=\int 2x\,dx=2\cdot\frac{x^2}2+C=x^2+C.$$
Now we find the particular $C$, and we do this by inputting the
``datum'' (sometimes called ``data point'') $f(3)=7$.
Graphically this means that the point $(3,7)$ is on the curve.
Since $f(x)=x^2+C$, we find $C$ using this datum:
\begin{alignat*}{2}
&&f(3)&=7\\
&\iff\qquad\qquad& 3^2+C&=7\\
&\iff &9+C&=7\\
&\iff &C&=-2.
\end{alignat*}
Thus $f(x)=x^2-2.$
\label{FirstFindCExample}
\eex
A graphical way of interpretting the example above is to realize
that all the curves $y=x^2+C$ are parabolas, and in fact are just
vertical shifts of the curve $y=x^2$.  Our task in 
Example~\ref{FirstFindCExample} was then to find which shift
satisfies both $f'(x)=2x$ and $f(3)=7$.  
In Figure~\ref{FigureForFirstFindCExample}, $y=x^2+C$
is graphed for various values of $C$.  Once we require
the graph to pass through a particular point---in
this case the point $(3,7)$, we ``pin down''
a particular curve, i.e., we
determine exactly one
curve from the family of curves, as graphed in 
Figure~\ref{FigureForFirstFindCExample}.
\begin{figure}
\begin{center}
\begin{pspicture}(-5,-1.8)(5,5.4)
\psset{xunit=.5cm,yunit=.3cm}
\psaxes[Dx=2,Dy=2]{<->}(0,0)(-10,-6)(10,18)
\psplot{-4.2426407}{4.2426407}{x dup mul}
\psplot{-4}{4}{x dup mul 2 add}
\psplot{-3.7416574}{3.7416574}{x dup mul 4 add}
\psplot{-3.4641016}{3.4641016}{x dup mul 6 add}
\psplot{-3.1622777}{3.1622777}{x dup mul 8 add}
\psplot{-2.8284271}{2.8284271}{x dup mul 10 add}
\psplot{-2.449497}{2.4494897}{x dup mul 12 add}
\psplot{-1.4142136}{1.4142136}{x dup mul 16 add}
\psplot{-2}{2}{x dup mul 14 add}
\psplot{-4.472136}{4.472136}{x dup mul 2 sub}
\psplot{-4.6904158}{4.6904158}{x dup mul 4 sub}
\pscircle[fillstyle=solid,fillcolor=black](3,7){.075}
\rput(5.4,7.){$(3,7)$}
\psline{->}(4.6,7.)(3.14,7)
%\psplot{-4}{4}{1 3 div x 3 exp mul}
\end{pspicture}

\end{center}
\caption{Partial view of the 
family of curves $y=x^2+C$. For Example~\ref{FirstFindCExample},
we needed to find the value of $C$ satisfying $f'(x)=2x$, i.e.,
$f(x)=x^2+C$, so that $(3,7)$ was on the curve.  Note that
the slopes of all curves given above are the same for a given
$x$-value, but only one passes through $(3,7)$.}  
\label{FigureForFirstFindCExample}\end{figure}

Finding a particular antiderivative is also very useful
in kinematics.  For instance, if we know the velocity
function $s'(t)=v(t)$, we can find the position function
$s$ if we are given one position datum to fix the constant. 
With the understanding that the constant is to be
determined, it is often written:
\begin{equation}
s(t)=\int s'(t)\,dt=\int v(t)\,dt.
\end{equation}
A common datum to prescribe is that $s(0)=s_0$
(where $s_0$ is some fixed number), but any data
which ``pins down'' the function will suffice.
\bex Suppose $v=t^2+11t-25$, and $s(1)=4$.  Find $s(t)$.

\underline{Solution}:
$$s(t)=\int v(t)\,dt=\int\left(t^2+11t-25\right)\,dt
=\frac{t^3}3+\frac{11t^2}2-25t+C.$$
Using $s(1)=4$ we get
$$\frac{1^3}3+\frac{11\cdot1^2}2-25(1)+C=4
\iff \frac13+\frac{11}2-25+C=4,$$
and so $C=4+25-\frac{11}2-\frac13=29-\frac{35}6=\frac{174}6-\frac{35}6
=\frac{139}6.$  Finally, this gives us
$$s(t)=\frac{t^3}3+\frac{11t^2}2-25t+\frac{139}6.$$
\eex

Now we will derive a well-known formula of physics.
\bex Suppose that acceleration is given by a constant, 
say $s''(t)=a$ (where $a$ is fixed).  Suppose further
that $s(0)=s_0$ and $v(0)=v_0$.  Now we work ``backwards''
from the acceleration towards the position function as
follows:
$$v(t)=s'(t)=\int s''(t)\,dt=\int a\,dt=at+C_1.$$
(Note that the last computation required that acceleration, $a$, be constant.)
Using $v(0)=v_0$, we then have 
$$a\cdot0+C_1=v_0\iff C_1=v_0.$$
This gives us the following equation, which itself is well known
to physics students:
\begin{equation}v(t)=at+v_0.\label{VelocityWithConstAccel}\end{equation}
Now we integrate (\ref{VelocityWithConstAccel}):
$$s(t)=\int s'(t)\,dt=
\underbrace{\int v(t)\,dt=\int\left(at+v_0\right)\,dt}_{
(\ref{VelocityWithConstAccel})}
=a\cdot\frac{t^2}2+v_0t+C_2.$$
Finally, using $s(0)=s_0$, we get
$$a\cdot\frac{0^2}2+v_0(0)+C_2=s_0\implies C_2=s_0.$$
Thus
\begin{equation}s=\frac12at^2+v_0t+s_0.\label{PositionWithConstAccel}
\end{equation}
\eex
It is important to note that (\ref{PositionWithConstAccel})
followed under the special condition that acceleration is 
constant (such as occurs when an object is
in freefall in a constant gravitational field, with no other 
resistance).  Nonconstant acceleration will not give
(\ref{VelocityWithConstAccel}) or (\ref{PositionWithConstAccel}).
However, the method for computing $v$ and $s$, given $a$, is
the same when $a$ is not constant:
\begin{enumerate}
\item find $\ds{v(t)=\int a(t)\,dt}$, using one datum regarding velocity
      at a particular time, to fix the constant
       of integration;
\item find $\ds{s(t)=\int v(t)\,dt}$, using another datum regarding
      position at a particular time, to fix the
      second constant of integration.
\end{enumerate}
Actually, two position data can fix the constants as well, since
we can just carry the first constant into the second calculation,
and then we will have two equations with two unknowns (the
constants of integration), and then solve for both constants.

\bex Suppose $a(t)=3t^2$, $s(0)=3$ and $s(1)=5$.  Find $v(t)$ and 
$s(t)$.


\underline{Solution}:
First we will find $v(t)$, to the extent that we can:
$$v(t)=\int a(t)\,dt=\int 3t^2\,dt =3\cdot\frac{t^3}{3}+C_1=t^3+C_1.$$
Next we find the form of $s(t)$:
$$s(t)=\int v(t)\,dt=\int\left(t^3+C_1\right)\,dt=\frac{t^4}4+C_1t+C_2.$$
So we know that $s(t)=\frac14t^4+C_1t+C_2$, for some $C_1,C_2$.
Using the facts that $s(0)=3$ and $s(1)=5$, we get the
following system of two equations in two unknowns:
$$\left\{\begin{array}{rcrcrcr}
\frac{0^4}4&+&C_1(0)&+&C_2&=3\\
\frac{1^4}4&+&C_1(1)&+&C_2&=5
\end{array}\right.
\iff \left\{\begin{array}{rcrcr}&&C_2&=3\\ C_1&+&C_2&=\frac{19}4
\end{array}\right.
$$
From the second form of the system, we see $C_2=3$,
 and so $C_1=\frac{19}4-3=\frac{7}4$.  
Putting all this together we first get
$$s(t)=\frac{t^4}4+\frac{7t}4+3,$$
from which we can calculate $v(t)=s'(t)$ (or just read $v(t)$
off of our first integral calculation, inserting $C_1=\frac74$) to get
$$v(t)=t^3+\frac74.$$ 
\eex 
\subsection{First Trigonometric Rules\label{FirstTrigRulesSubsection}}
With every derivative formula for functions comes an analogous
antiderivative formula, which is more or less the derivative
formula in reverse.  Sometimes the reverse is more obvious than
other times.  For instance, the power rule formula for derivatives is sometimes
seen algorithmically as ``{\it multiply} by the exponent (`bring
the power down') and {\it decrease} the exponent by one,''
as in
$$\frac{d}{dx}\left[x^n\right]=n\cdot x^{n-1}.$$  
If we are careful to reverse the process, we need to do
the inverse steps in reverse order:  {\it increase} the exponent
by one, and then {\it divide} by the exponent.  That is the
essence of (\ref{IntegralPowerRule}):
$$\int x^n\,dx=\frac{x^{n+1}}{n+1}+C.$$  
By other sophisticated arguments, in the next section we
will see a kind of reverse chain rule.  A bit later in the text
we will also come across what can be loosely called a
reverse product rule called integration by parts,
although there really is no good analog of the product rule
with integrals {\it per se}.\footnote{%
%%%%%%%%%  FOOTNOTE
Integration by parts is really an integration technique
which takes advantage of the product rule for derivatives---or
more precisely a permutation of the product rule for 
derivatives---but is not itself a product rule for integrals;
it does not by itself give a formula for
$\int f(x)g(x)\,dx$.
Instead it gives a formula which can be summarized by
$$\int f(x)g'(x)\,dx=f(x)g(x)-\int g(x)f'(x)\,dx,$$
which follows from integrating---i.e., 
applying $\int(\cdots)\,dx$ to both sides of---the following
rearrangement of the product rule,
$$f(x)g'(x)=[f(x)g(x)]'-g(x)f'(x).$$
Because the product rule for derivatives is what makes the
technique of integration by parts valid, many authors describe
it as a kind of analog of the product rule, though again,
it is not a direct formula for the integral of a product
like we had for the derivative of a product.

Still, it is a very useful technique which we will spend some time
developing in a later chapter, when we have other methods to draw upon
for the inevitable intermediate computations.
%%%%%%%%%%%%%%  END FOOTNOTE
} \footnote{
There will be still several other techniques which are not
at all simple reverses of derivative rules, and for which checking
by differentiating (computing the derivative of) the answer
is as difficult as, or more difficult than, the integration
technique itself.  Those sophisticated techniques are for later chapters.
}

The formulas presented in this subsection are immediate
consequences of our trigonometric derivative formulas.
For instance, we have the following pair of formulas:
\begin{align*}
\frac{d\sin x}{dx}=\cos x&\iff \int\cos x\,dx=\sin x+C,\\
\frac{d\cos x}{dx}=-\sin x&\iff \int(-\sin x)\,dx=\cos x+C.\end{align*}
This second integration formula is more awkward than necessary,
since it is more likely we would like an antiderivative for 
$\sin x$ directly.  We could multiply both sides by $-1$, and
rename the new constant $C$, or just notice that
$\frac{d}{dx}(-\cos x)=\sin x$, to come to the formula
$$\int \sin x\,dx=-\cos x+C.$$
As before, we can always check these by taking the derivative
of the right-hand side.  Recalling our six basic trigonometric
derivative formulas, and making adjustments for negative sign
placements as above, we have the following pairs of 
derivative/integral formulas:
\begin{alignat}{5}
\frac{d\sin x}{dx}&=\cos x&\qquad&\iff\qquad&&\int\cos x\,dx&&=\sin x+C,
      \label{IntOfCosine}\\
\frac{d\cos x}{dx}&=-\sin x&&\iff&&\int\sin x\,dx&&=-\cos x+C,
      \label{IntOfSine}\\
\frac{d\tan x}{dx}&=\sec^2x&&\iff&&\int\sec^2x\,dx&&=\tan x+C,
      \label{IntOfSecSquared}\\
\frac{d\cot x}{dx}&=-\csc^2x&&\iff&&\int\csc^2x\,dx&&=-\cot x+C,
      \label{IntOfCscSquared}\\
\frac{d\sec x}{dx}&=\sec x\tan x&&\iff&&\int\sec x\tan x\,dx&&=\sec x+C,
      \label{IntOfSecTan}\\
\frac{d\csc x}{dx}&=-\csc x\cot x&&\iff&&\int\csc x\cot x\,dx
                                    &&=-\csc x+C.
      \label{IntOfCscCot}
\end{alignat}
With these and our previous rules, we have some limited
ability to compute integrals involving trigonometric functions.
\bex Consider the following integrals:
\begin{itemize}
\item $\ds{\int\left[x^2+\sin x-\frac1x\right]\,dx=\frac{x^3}3-\cos x+\ln|x|+C
            }$
\item $\ds{\int\frac{\sin x}{\cos^2x}\,dx
      =\int\left[\frac1{\cos x}\cdot\frac{\sin x}{\cos x}\right]\,dx
      =\int\sec x\tan x\,dx=\sec x+C}$
\item $\ds{\int\cos w\,dw=\sin w+C}$
\item $\ds{\int\tan^2x\,dx=\int\left(\sec^2x-1\right)\,dx=\tan x-x+C}$.
\end{itemize}
\eex
In fact we are fortunate if an integral has the kind of trigonometric
form which is just the derivative of one of the six basic trigonometric
functions.  When it is the case, it often requires some rewriting,
as in two of the integrals above.
\subsection{Integrals Yielding Inverse Trigonometric Functions}


\begin{align}
\int\frac1{\sqrt{1-x^2}}\,dx&=\sin^{-1}x+C,\label{IntGivesArcsine}\\
\int\frac1{x^2+1}\,dx&=\tan^{-1}x+C,\\
\int\frac1{x{\sqrt{x^2-1}}}\,dx&=\sec^{-1}|x|+C.\label{IntGivesArcsecant}
\end{align}
Note how we only employ three of the six arctrigonometric functions
in (\ref{IntGivesArcsine})--(\ref{IntGivesArcsecant}).  In fact
these are sufficient.
Recall for instance
that the arccosine and arcsine have derivatives which differ by the 
factor $-1$.  For simplicity, it is much more commonly written 
$\int\frac{-1}{\sqrt{1-x^2}}\,dx=-\sin^{-1}x+C$, rather than
using the arccosine function as the antiderivative,
i.e.,  $\int\frac{-1}{\sqrt{1-x^2}}\,dx=\cos^{-1}x+C$
though the latter is certainly legitimate.  Indeed,
since $\sin^{-1}x+\cos^{-1}x=\frac{\pi}2$, we see
$\cos^{-1}x$ and $-\sin^{-1}x$ differ by a constant.  In fact
one could rewrite (\ref{IntGivesArcsine}) as
$\int\frac1{\sqrt{1-x^2}}\,dx=-\cos^{-1}x+C$.  The choice can
sometimes depend upon which range of angles we wish the antiderivative
fuction to output, though for this case
we can adjust that with the constant $C$.



Similarly one usually writes $\int\frac{-1}{x^2+1}\,dx=-\tan^{-1}x+C$, 
though $\cot^{-1}x+C$ (for a ``different'' $C$) is also legitimate.
Analogously for arcsecant and arccosecant; we usually avoid the arccosecant
function as an antiderivative.
  
But for these last two there is another small complication.  Note how
Equation (\ref{IntGivesArcsecant}) has the absolute value on the 
antiderivative rather than the $x$-term of the denominator in the integrand,
so it does not appear to be just a restatement of the derivative rule
for the arcsecant: $\frac{d}{dx}\sec^{-1}x=\frac1{|x|\sqrt{x^2-1}}$.
To see that (\ref{IntGivesArcsecant}) is still correct,
note that $|x|=x$ if $x>0$, and $|x|=-x$ if $x<0$.  Taking the derivative of
$\sec^{-1}|x|$ for those two cases, as we did in the computation
of $\frac{d}{dx}\ln|x|$  (see page~\pageref{ProofOfDerivLn|X|Stuff})
we can see that we do get 
$\frac1{x\sqrt{x^2-1}}$ both times.  But it should also be noted that,
while not often seen, it would be legitimate to have the 
absolute value inside, rather than outside, the integral, as 
in
$\int\frac1{|x|\sqrt{x^2-1}}\,dx=\sec^{-1}x+C$.\footnote{%
%%% FOOTNOTE
To further complicate things, we could notice that
$\sec^{-1}x=\cos^{-1}\frac1x$, so it can occur that computational
software will output an arccosine function, as 
in $\int\frac1{x\sqrt{x^2-1}}\,dx=\cos^{-1}\frac1{|x|}+C$,
and then this can be rewritten (with a ``different $C$'')
$\int\frac1{x\sqrt{x^2-1}}\,dx=-\sin^{-1}\frac1{|x|}+C$.
The software might also 
omit the absolute values, theoretically assuming $x>0$.
Still, the standard written
computation would output the expected $\sec^{-1}|x|+C$.
%%% END FOOTNOTE
}

Before listing some examples, we last make note of the convention
mentioned earlier (see page~\pageref{UsingdxAsAFactor}), of
using $dx$ as a factor.  So our new integration formulas
are often written:
\begin{align*}
\int\frac{dx}{\sqrt{1-x^2}}&=\sin^{-1}x+C,\\
\int\frac{dx}{x^2+1}&=\tan^{-1}x+C,\\
\int\frac{dx}{x\sqrt{x^2-1}}&=\sec^{-1}|x|+C.
\end{align*}

The above formulas will become much more important in future sections.
For now it is important to realize that these particular function forms 
do have (relatively) simple antiderivatives.  At this point in the 
development we are not prepared to make full use of these forms, 
but we need to be aware of them.  Sometimes a simple manipulation
produces a function containing one of these forms.

\bex Compute $\ds{\int\frac{x^2}{x^2+1}\,dx}$.

\underline{Solution}: With the aid of long division, we can see that\footnote{%
%%% FOOTNOTE
A popular alternative 
technique for a fraction like that in our integrand is to 
strategically add and subtract a term in the numerator, which
produces a term in the numerator identical to (or a multiple of)
the denominator,
and the extra term, from which we can make two fractions:
$$\frac{x^2}{x^2+1}=\frac{x^2+1-1}{x^2+1}
      =\frac{x^2+1}{x^2+1}-\frac1{x^2+1}
      =1-\frac1{x^2+1}.$$
While this is a very useful technique for such a simple case, 
it does not easily extend to more complicated cases.  Long division---when
the degree of the numerator is at least that of the denominator---is more
straightforward.  However, we will eventually have need of a technique
similar to that given above in this footnote, though that setting will be 
much more complicated and we will need all other advanced integration
techniques to make full use of it.
%%% END FOOTNOTE
}
$$\frac{x^2}{x^2+1}=1-\frac1{x^2+1},$$
and so
$$\int\frac{x^2}{x^2+1}\,dx
=\int\left[1-\frac1{x^2+1}\right]\,dx
=x-\tan^{-1}x+C.$$

\eex

\subsection{Integrals Yielding Exponential Functions}

To finish our list of integrals which arise from differentiation
formulas, we list those yielding exponential functions.  Below
$a\in(0,1)\cup(1,\infty)$.

\begin{alignat}{2}
&\int e^x\,dx&&=e^x+C,\label{IntOfE^X}\\
&\int a^x\,dx&&=\frac{a^x}{\ln a}+C.\label{IntOfA^X}\end{alignat}
The first of these, (\ref{IntOfE^X}) is the more obvious.
To see (\ref{IntOfA^X}), recall that $\frac{d\,a^x}{dx}=a^x\ln a$,
and that $\ln a$ is a constant.
Thus
$$\frac{d}{dx}\left[\frac{a^x}{\ln a}\right]
=\frac1{\ln a}\cdot\frac{d\,a^x}{dx}
=  \frac1{\ln a}\cdot a^x\ln a=a^x,\qquad\text{q.e.d.}$$
\bex We compute some antiderivatives involving these.
\begin{itemize}
\item $\ds{\int\left[1+x+e^x\right]\,dx=x+\frac{x^2}2+e^x+C}$,
\item $\ds{\int 2^x\,dx=\frac{2^x}{\ln2}+C}$,
\item $\ds{\int\frac{2^x-3^x}{5^x}\,dx
        =\int\left[\frac{2^x}{5^x}-\frac{3^x}{5^x}\right]\,dx
        =\int\left[\left(\frac25\right)^x-\left(\frac35\right)^x\right]\,dx
        =\frac{\left(\frac25\right)^x}{\ln\frac25}
         -\frac{\left(\frac35\right)^x}{\ln\frac35}+C}$,
\item $\ds{\int\!\left(1+3^x\right)^2\,dx
        =\int\left[1+2\cdot3^x+3^{2x}\right]\,dx
        =\int\left[1+2\cdot3^x+9^x\right]\,dx
        =x+\frac{2\cdot3^x}{\ln 3}+\frac{9^x}{\ln 9}+C}$.
\end{itemize}
In the last integral, we used the fact that
$\left(3^x\right)^2=3^{x\cdot2}=3^{2\cdot x}=\left(3^2\right)^x=9^x$.
\eex











\newpage
\section{Substitution With Power Rule\label{FirstSubstitutionSection}}
Substitution in general is the most important of the integrating
techniques, finding its way into the other techniques as well.
While we introduce it here, for now we limit the scope to power rules.

Before looking at this method formally, consider the following
antiderivative statements, each of which refer to the 
same power rule (perhaps most familiar in the first case):
\begin{align}
\int x^2\,dx&=\frac{x^3}3+C,\notag\\
\int u^2\,du&=\frac{u^3}3+C,\notag\\
\int (\sin x)^2\,d(\sin x)&=\frac{(\sin x)^3}3+C.
     \label{DumbSubExampleSin^2X--W.R.T.SinX}
\end{align}
The last integral is simply asking for an antiderivative of
$(\sin x)^2$ with respect to $\sin x$.  Indeed, we can check the
answer as before:
$$\frac{d}{d\sin x}\left[\frac13(\sin x)^3\right]
    =\frac13\cdot 3(\sin x)^2=(\sin x)^2,$$
as we expect.  Of course we usually take derivatives and antiderivatives
with respect to a variable, and not a function.  However the
integral in (\ref{DumbSubExampleSin^2X--W.R.T.SinX})
is not so unlikely to be occur as one might think. 
Recall that $df(x)=f'(x)\,dx$
is the definition of the differential (see
(\ref{EquationDefiningDifferentials}), 
page~\pageref{EquationDefiningDifferentials}).  Thus
$d\sin x=\cos x\,dx$, and the integral
in (\ref{DumbSubExampleSin^2X--W.R.T.SinX}) can be written instead
$$\int(\sin x)^2\cos x\,dx=\int(\sin x)^2\frac{d\sin x}{dx}\,dx
                          =\int(\sin x)^2\,d\sin x
                          =\frac13(\sin x)^3+C.$$
Indeed, it is not hard to see that the chain rule gives
us $\frac{d}{dx}\left[
\frac13(\sin x)^3\right]=\frac13\cdot3(\sin x)^2\cos x
=\sin^2x\cos x$. 

In this section we will concentrate on integrals of the form
\begin{equation}\int u^n\,du=\left\{\begin{array}{lclcl}
             \frac1{n+1}\cdot u^{n+1}&+&C&\qquad&\text{if }n\ne1,\\ \\
             \ln|u|&+&C&&\text{if }n=-1.\end{array}\right.
             \label{PowerRuleForIntegrationInU}
       \end{equation}
As anticipated in the discussion above, the content of the differential
$du$ may be more expansive than what we may expect from a single variable.
The point of this section is to recognize when we have the form
(\ref{PowerRuleForIntegrationInU}), and how to go about rewriting the
integral into the proper form.  

The reader should be forwarned:  this method requires a fair amount 
of practice.  It is not a simple algorithm. For each problem, the
reader has to decide which substitution will produce an integral
which can be computed with known rules.  Here we will limit ourselves
to the power rule (\ref{PowerRuleForIntegrationInU}), but
in subsequent sections we will delve into many other rules, and
it is not always obvious which rule should be used for a given integral.
With practice one learns to look for clues, and anticipate what
will occur several steps ahead, to see if there is indeed an
integration rule which can apply.\footnote{%%%
%%% FOOTNOTE
In fact, often there is no rule which will produce an antiderivative,
and then some approximation scheme will be necessary.  Still, it is most
desirable to have an exact antiderivative, and we can find one
often enough that it is well worth studying these techniques.}

\subsection{The Technique}
Here we will look at some of the simpler problems of integration by
substitution.  As we proceed, several observations will be made
regarding the method.

\bex Compute the indefinite integral $\ds{\int (x^2+1)^7\cdot2x\,dx}$.

\underline{Solution}: The technique is to introduce a new variable,
$u$, with which we can write the original integral in a simpler form.
We also have to take into account what will be the new differential,
namely $du$:
\begin{align*}
u&=x^2+1\\
\implies\qquad du&=2x\,dx.
\end{align*}
(Recall that if $u$ is a function of $x$, then $du=u'(x)\,dx$,
consistent with $\frac{du}{dx}=u'(x)$.)  Using this information, we can
replace all the terms in the original integral: the $(x^2+1)^7$
becomes $u^7$, and the terms $2x\,dx$ collectively become $du$
(see the above implication arrow).  Thus
$$\int(x^2+1)^7\cdot\underline{2x\,dx}=\int u^7\,\underline{du}
=\frac18\,u^8+C.$$
This is all true, but {\bf we} introduced $u$, while the
original question asked for an antiderivative with respect to $x$.
We only need to replace $u$ in the final answer, using again
$u=x^2+1$.  Summarizing,
$$\int(x^2+1)^7\cdot{2x\,dx}
=\int u^7\,{du}=\frac18\,u^8+C=\frac18(x^2+1)^8+C.$$
\label{FirstUSubExample}\eex

Note that we can check our answer in the above example by
computing the derivative of the answer (using the 
chain rule), yielding $\frac{d}{dx}\left[\frac18(x^2+1)^8\right]=
\frac18\cdot8(x^2+1)^7\cdot2x=
(x^2+1)^7\cdot2x$ as hoped.  
In fact, integration by substitution, at least in its
simplest forms, is often called a type of reverse chain rule.
Indeed, we can rewrite (\ref{PowerRuleForIntegrationInU}) as follows:
\begin{equation}\int u^n\,du=\int u^n\cdot\left(\frac{du}{dx}\right)\,dx
            =\int \left[u^n\cdot\frac{du}{dx}\right]\,dx
=\left\{\begin{array}{lcll}
             \frac1{n+1}\cdot u^{n+1}&+&C\quad&\text{if }n\ne1,\\ \\
             \ln|u|&+&C&\text{if }n=-1.\end{array}\right.
\label{IntPowerRuleWithUSubstitution}\end{equation}
To see that this is correct, if we take the derivative of 
the answers {\it with respect to $x$}, we see that we do indeed
get $u^n\cdot\frac{du}{dx}$ from the chain rule.    

In short, with integration by substitution we try to pick some
function we call $u$, so that
\begin{enumerate}
\item a main part of the integrand can be written as a simple function
      of $u$---one for which we know the antiderivative with respect to 
      $u$---and, equally crucial, so that
\item the remaining variable terms of the integral can be
      safely absorbed in $du$ (except for multiplicative constants,
      which we will see add only a slight complication).
\end{enumerate}
If these are both satisfied, our substitution of $u$ and $du$ terms
gives us a new, simple integral (entirely in terms of $u$ and $du$!).

When working such a problem (as opposed to, say, {\it publishing} a problem
and solution for professional consumption),
a useful format is to (1) write the original integral, (2) write the
substitution function $u$, with its differential $du$ both on different
lines than the original integral, (3) write the new form of
the original integral, i.e.,  in $u$ 
and $du$,  (4) compute the antiderivative of this new integral
in $u$ as a continuation of the first step,
and (5) resubstitute to arrive at the antiderivative in $x$.
Hence a typical homework-style presentation of 
Example~\ref{FirstUSubExample}
might look like the following (with the choice of
$u$ and resulting $du$ offset and below the original integral):

\begin{align*}
\int(x^2+1)^7\cdot\underline{2x\,dx}
         &=\int u^7\,\underline{du}=\frac18u^8+C=\frac18(x^2+1)^8+C\\
\left.\begin{aligned}
u&=x^2+1\\ du&=\underline{2x\,dx}\end{aligned}\right|&\end{align*}
That part of the integral which we hope to absorb into $du$ is underlined
in the original integral, the computation of $du$, and the corresponding
term in the new integral.  The rest of the integral was just 
$(x^2+1)^7=u^7$. In fact, when we choose $u$ so that a major
portion of the integral can be written $u^n$, then any
other factors which are variable, along with the differential
$dx$, must be absorbed in $du$ or the substitution will fail
(because the resulting integral will contain both $x$ and $u$
and no antiderivative rules will apply).
We will continue to use this kind of spatial organization 
when we integrate by substitution in the examples below.
\bex Compute the indefinite integral $\ds{\int\sin^2x\cos x\,dx}$.

\underline{Solution}: Note that this integral can be written
$\ds{\int(\sin x)^2\cos x\,dx}$.  Now we proceed:
\begin{align*}
\int\sin^2x\ \underline{\cos x\,dx}&=\int u^2\,\underline{du}=\frac13u^3+C
                                          =\frac13\sin^3x+C.\\
\left.\begin{alignedat}{2}
&&u&=\sin x\\
&\implies&du&=\underline{\cos x\,dx}
\end{alignedat}\right|&\end{align*}
\eex

Before continuing we will make a very minor change to the integral
in the first example (Example~\ref{FirstUSubExample},
page~\pageref{FirstUSubExample}), 
and show a simple way to extend our method to handle this.

\bex Compute $\ds{\int x(x^2+1)^7\,dx}$.

\underline{Solution}: Here we will make the same substitution
as before, but the $du$ will have an extra factor of 2.  Since
constant factors are relatively easy to handle in derivative
and antiderivative problems in general, we should not expect
this extra factor of 2 to cause much difficulty.  It 
will simply mean one extra step in the  substitution
computations.\footnote{%
%%% FOOTNOTE
There is another method used by some texts to handle a 
problem such as this, which is to simply introduce the
needed factor of 2 in the integral to complete the
differential $du=2x\,dx$, and compensate for the insertion of
the new factor by
simultaneously inserting a factor of $\frac12$, which
is simply carried through the rest of the calculation:
$$\int x(x^2+1)^7\,dx=\int\frac12
 \underbrace{(x^2+1)^7}_{u^7}\cdot\underbrace{2x\,dx}_{du}
 =\frac12\int u^7\,du=\frac12\cdot\frac18u^+C=\frac1{16}(x^2+1)^8+C,$$
where again $u=x^2+1$, $du=2x\,dx$.

This method is appealing because one rewrites the integrand into a form
where it is, more or less, clearly a derivative of a chain rule function
(perhaps multiplied by a constant, as with $\frac12$ here).

We will avoid this method because, though it is not so challenging for
simpler problems, it quickly becomes unreasonably difficult if an
integral is complicated.  Furthermore, the method presented in this
text---in the author's opinion---makes for much 
better preparation for more advanced methods,
such as trigonometric substitution and integration by parts.
%%%% END FOOTNOTE
}

\begin{align*}
\int \underline{x}(x^2+1)^7\,\underline{dx}
         &=\int u^7\cdot\underline{\frac12\,du}=\frac12\cdot
                \frac18u^8+C=\frac1{16}(x^2+1)^8+C\\
\left.\begin{alignedat}{2}
&&u&=x^2+1\\
&\implies& du&=2\,\underline{x\,dx}\\ &\implies&\frac12du&=\underline{x\,dx}
\end{alignedat}\right|\qquad&\end{align*}
\label{IntX(X^2+1)^7dxExample}\eex

This time the extra nonconstant and differential terms of the original
integral were, collectively, $x\,dx$.  Though that product is not
exactly $du$, it is a constant times $du$. In our substitution
we took an extra step and solved, again collectively, for $x\,dx=\frac12du$.


The preceding example shows that we need to be flexible when looking
for a possible power rule application.  Not every integral where
we can use the power rule will be of the strict form
(\ref{IntPowerRuleWithUSubstitution}), 
page~\pageref{IntPowerRuleWithUSubstitution}.  Indeed, we need
to be especially vigilant to notice that an integral may be
of the form
\begin{equation}
\int k\cdot u^n\,du=\int k\cdot u^n\cdot\left(\frac{du}{dx}\right)\,dx.
\label{IntsWhichAreConstantsTimesChainRuleFormOfPowerRule}
\end{equation}
So when we make a substitution, we try not to be distracted by
extra or missing multiplicative 
constants, as they will work themselves out in
the substitution and final integration steps.




\bex Compute $\ds{\int x^3\cos^5x^4\sin x^4\,dx}$.

\underline{Solution}: It is perhaps more obvious how to proceed
if we rewrite the integral in the form
$\ds{\int x^3(\cos x^4)^5\sin x^4\,dx}$.  Then we see that
the $u^n$ term will be $(\cos x^4)^5=u^5$, where $u=\cos x^4$.
Next we need to see if $du$ can absorb the other nonconstant terms:
\begin{align*}
\int \underline{x^3}\cos^5x^4\underline{\sin x^4\,dx}
 &=\int u^5\underline{\left(-\frac14\right)\,du}
 =-\frac14\cdot\frac16u^6+C=-\frac1{24}(\cos x^4)^6+C.\\
\left.\begin{alignedat}{2}
&&u&=\cos x^4\\
&\implies&du&=-\underline{\sin x^4}\cdot 4\,\underline{x^3\,dx}\\
&\implies&-\frac14du&=\underline{x^3\sin x^4\,dx}\end{alignedat}\right|&
\end{align*}
\eex
We should point out that the method above would not have
worked without {\it both} the $x^3$ and the $\sin x^4$
terms in the integral, for the $du$ would have 
variable terms not in the orginal integral.

Also, it is possible to compute the integral in the previous
by using two substitution steps instead of one.
For instance, a student recognizing that $x^4$, and a multiple
of its derivative in the form of $x^3$, both appear, might 
first make a substitution of the form $u=x^4$:
\begin{align*}
\int \underline{x^3}\cos^5x^4\sin x^4\,\underline{dx}&=
         \int\cos^5u\,\sin u\,\underline{\frac14\,du}
 =\frac14\int\cos^5u\,\sin u\,du.\\
\left.\begin{alignedat}{2}
&& u&=x^4\\
&\implies&du&=4\,\underline{x^3\,dx}\\
&\implies&\frac14du&=\underline{x^3\,dx}\end{alignedat}\right|&
\end{align*}
At this point, we have a simpler integral which itself requires a
substitution:
\begin{align*}
\frac14\int\cos^5u\,\underline{\sin u\,du}&=\frac14\int w^5\underline{(-dw)}
 =-\frac14\cdot\frac16 w^6+C=-\frac1{24}(\cos u)^6+C.\\
\left.\begin{alignedat}{2}
&&w&=\cos u\\
&\implies&dw&=-\underline{\sin u\,du}\\
&\implies&-dw&=\underline{\sin u\,du}\end{alignedat}\right|&\end{align*}
Of course this gives the answer in terms of $u$, so we substitute
back again, in terms of $x$.  Summarizing,
$$\int x^3\cos^5x^4\sin x^4\,dx=\cdots=
    -\frac1{24}w^6+C=-\frac1{24}\cos^6u+C
                =-\frac1{24}\cos^6x^4+C.$$
The second approach is longer, but it has the advantage that we
are not trying to rewrite the integral in one, all-encompassing
(and thus more complicated)
substitution step. Indeed it is sometimes desirable to simplify an 
integral with substitution even if the resulting integral cannot be 
evaluated immediately.  With most examples we will use the first
method, but the student working problems should be aware that
the option of successive substitutions is perfectly valid.

Next we look at a few very common types of examples where the
power of $n$ is $1/2$, $-1$ and $-2$.  These appear often enough that
it is worth some effort to remember them specifically.

\bex Compute $\ds{\int\frac{x}{\sqrt{x^2-9}}\,dx}$.

\underline{Solution} Here we will take $u=x^2-9$, since the $du=2x\,dx$
can absorb both the $dx$ and the extra factor of $x$:
\begin{align*}
\int\frac{\underline{x}}
  {\sqrt{x^2-9}}\,\underline{dx}&=\int u^{-1/2}\cdot\underline{\frac12\,du}
 =\frac12\cdot2{u}^{1/2}+C=\sqrt{x^2+1}+C.  \\
\left.\begin{alignedat}{2}&&u&=x^2-9\\ &\implies&du&=2\,\underline{x\,dx}\\
       &\implies&\frac12\,du&=\underline{x\,dx}\end{alignedat}\right|&
\end{align*}
\eex
\bex Compute $\ds{\int\frac{\sin x}{\cos x}\,dx}$.

\underline{Solution}:  Here we will take $u=\cos x$, since
$du=-\sin x\,dx$ will absorb the other terms.  

\begin{align*}
\int\frac{{\sin x}\,dx}{\cos x}
 &=\int\frac1{u}(-du)=-\ln|u|+C=-\ln|\cos x|+C.\\
\left.\begin{alignedat}{2}
&&u&=\cos x\\
&\implies&du&=-\underline{\sin x\,dx}\\
&\implies&-du&=\underline{\sin x\,dx}\end{alignedat}\right|&
\end{align*}


Note that
if we instead took $u=\sin x$, then $du=\cos x\,dx$, but $\cos x$ is not
a {\bf multiplicative} factor in the original integral; the desired factor is 
$\frac1{\cos x}$, which is not contained in the $du$ term
if $u=\sin x$.
\eex
It should be remembered that checking these antiderivatives is as
simple as computing the derivative of the answer.  Here
$$\frac{d}{dx}\left[-\ln|\cos x|\right]=
   -\frac{1}{\cos x}\cdot\frac{d}{dx}\cos x
   =-\frac{1}{\cos x}(-\sin x)=\frac{\sin x}{\cos x},$$
as we hope. Of course our original integrand, and the
derivative above, can both be written $\tan x$.

 Note that we can write
$-\ln|\cos x|=\ln\left|(\cos x)^{-1}\right|=\ln|\sec x|$,
so many calculus books contain the integration formula
\begin{equation}\int\tan x\,dx=\ln|\sec x|+C.\label{FirstAntiForTanGiven}
\end{equation}
(It is also interesting to ``fill in the dots''
for the computation $\frac{d}{dx}|\sec x|=\cdots=\tan x$,
verifying (\ref{FirstAntiForTanGiven}).
See also Exercise~\ref{IntOfSecXTanX/SecXForFun}, 
page~\pageref{IntOfSecXTanX/SecXForFun}.)

\bex Compute $\ds{\int\frac{e^{3x}}{(e^{3x}+4)^2}\,dx}$.

\underline{Solution}:
Here we note that the numerator of the integrand, namely $e^{3x}$, is
the derivative of $e^{3x}+4$, except for a multiplicative constant.
Thus we will let $u=e^{3x}+4$:
\begin{align*}
\int\frac{e^{3x}}{(e^{3x}+4)^2}\,dx&
  =\int\frac1{u^2}\,\cdot\frac13\,du
  =\frac13\int u^{-2}\,du =\frac13\cdot(-1)u^{-1}+C\\
\left.\begin{alignedat}{2}
&&u&=e^{3x}+4\\
&\implies&du&=\underline{e^{3x}}\cdot3\underline{dx}\\
&\implies&\frac13\,du&=\underline{e^{3x}\,dx}
\end{alignedat}\right|&=-\frac13(e^{3x}+4)^{-1}+C 
                        =\frac{-1}{e^{3x}+4}+C.
\end{align*}


\eex

At this point we notice three common forms of integration by
substitution:
\begin{align}
\int\frac{u'(x)}{\sqrt{u(x)}}\,dx&=2\sqrt{u(x)}+C,\\
\int\frac{u'(x)}{u(x)}\,dx&=\ln|u(x)|+C,\\
\int\frac{u'(x)}{[u(x)]^2}\,dx&=\frac{-1}{u(x)}+C.
\end{align}
In all three cases, $u'(x)\,dx=du$, and we have simple power
rules.  In the first and third cases there are
multiplicative constants which occur.  There is no real
need to memorize these, but they occur often enough that 
their ``mechanics'' should become familiar.  For that
reason these three results can become, if not memorized, 
then at least easily cited.

The method also works for cases where the $du$ term is 
just a constant multiple of $dx$:
%\bex Compute $\ds{\int\sec5x\tan5x\,dx}$.
%
%\underline{Solution}: Here we will let $u=5x$:
%\begin{align*}
%\int\sec5x\tan5x\,dx&=\int\sec u\tan u\,\frac15\,du
%                    =\frac15\sec u+C=\frac15\sec5x+C.\\
%  \left.\begin{alignedat}{2}
%    && u&=5x\\
%    &\implies&du&=5\,\underline{dx}\\
%    &\implies&\frac15\,du&=\underline{dx}
%   \end{alignedat}\right\}&
%\end{align*}
%
%\eex

\bex Compute $\ds{\int\frac1{(6-2x)^5}\,dx}$.

\underline{Solution}: 
\begin{align*}
\int\frac1{(6-2x)^5}\,dx&=\int u^{-5}\cdot\frac{-1}2\,du
   =-\frac12\cdot\frac1{-4}u^{-4}+C
   =\frac18(6-2x)^{-4}+C \\
\left.\begin{alignedat}{2}
&&u&=6-2x\\
&\implies&du&=-2\,\underline{dx}\\
&\implies&\frac{-1}2\,du&=\underline{dx}\end{alignedat}\right|&
=\frac1{8(6-2x)^4}+C.
\end{align*}
\eex

In the case that $du=dx$, this can often be anticipated and the
experienced calculus student might omit the middle steps:
%\bex Compute $\ds{\int\csc^2(x-\pi)\,dx}$.
%
%\underline{Solution}:
%\begin{align*}
%\int\csc^2(x-\pi)\,\underline{dx}&=\int\csc^2u\,\underline{du}
%       =-\cot u+C=-\cot(x-\pi)+C.\\
%\left.\begin{alignedat}{2}
%  &&u&=x-\pi\\
%&\implies&du&=\underline{dx}\end{alignedat}
%\right\}&\end{align*}
%\eex
\bex Compute $\ds{\int(x+9)^4\,dx}$.

\underline{Solution}:
\begin{align*}
\int(x+9)^4&=\int u^4\,du=\frac15u^5+C=\int\frac15(x+9)^5+C.
\\
\left.\begin{aligned}
u&=x+1\\
\implies du&=dx\end{aligned}\right|&
\end{align*}
\eex





Another way to look at the example above is to realize
that $d(x+9)=dx$, so we can write
$$\int(x+9)^4\,dx=\int(x+9)^4d(x+9)
                       =\frac15(x+9)^5+C.$$
In other words, $dx$ is the same as $d(x+9)$, so we get the same
if we interpret the original integral as an antiderivative with 
respect to $(x+9)$.\footnote{Note that the {\it change} in $x+9$ is the
same as the change in $x$.}  Indeed this is a shortcut one learns with
practice---thinking but perhaps not writing the second step---but 
at first it is still best to write out the full substitution,
as in the example above, at least until one is proficient in the 
method as presented here.  Of course this is the analog to 
a chain rule where the ``inner'' derivative is 1:
$$\frac{d}{dx}\left[\frac15(x+9)^5\right]
  =\frac15\cdot5(x+9)^4\cdot\frac{d(x+9)}{dx}
  =\frac15\cdot5(x+9)^4\cdot1=(x+9)^4,\text{ q.e.d.}$$



\subsection{A Slight Twist on the Method}

Recall our second example, namely Example~\ref{IntX(X^2+1)^7dxExample}
on page~\pageref{IntX(X^2+1)^7dxExample}: $\int x(x^2+1)^7\,dx$.
We used a substitution $u=x^2+1$ because $du=2x\,dx$ contained the
extra factor of $x$ in the integrand.  The substitution
eventually gave us $\int u^7\cdot\frac12\,du$, which was a simple
power rule.  Of course we could have ``simply'' expanded the
original function
\begin{align*}
x(x^2+1)^7&=x(x^2+7x^4+21x^6+35x^8+35x^{10}+21x^{12}+7x^{14}+1)\\
          &=x^3+7x^5+21x^7+35x^9+35x^{11}+21x^{13}+7x^{15}+x,
\end{align*}
and integrated ``term by term.''  However the substitution method was
arguably easier, and the answer's simple form, 
$\frac1{16}(x^2+1)^{8}+C$ would
probably not be recongnizable from a strategy which expands the
integrand first.

Now consider the integral $\ds{\int x(x-1)^{3/2}\,dx}$.
Here we can not simply ``expand'' the integrand (even
by brute force, as above), because
of the fractional power term $(x-1)^{3/2}$, which is algebraically
more difficult to deal with than positive integer powers.  Furthermore,
if we let $u=x-1$, then $du=dx$, but this differential
term cannot absorb the extra factor $x$. The key is to then
notice that the original substitution offers a way out:
that extra factor $x$ can be rewritten $u+1$ (since
$u=x-1\iff u+1=x$).
Below we show how this can be utilized.
Indeed we will expand the new integrand, but what is interesting 
is how the algebraic difficulties of the $(x-1)^{3/2}$ term
(namely that this is of the form $({a+b})^{r}$,
$r\not\in\mathbb{N}$) is transfered
to the $x$ term which, being a positive integer power, is 
then easier to handle.  Below we write this out in the standard
example format:\newpage
\bex Compute $\ds{\int x(x-1)^{3/2}\,dx}$.

\underline{Solution}: Here we substitute for $dx$ {\bf and} $x$.
Both substitutions are calculated below, but separately.
(This time we underline the substitution for $x$, instead of the
differential part.) Once the substitutions are completed, we can 
perform the multiplication to get two simple power rules:
\begin{align*}
\int \underline{{x}}(x-1)^{3/2}\,dx&=
\int\underline{{(u+1)}}u^{3/2}\,du
=\int\left(u^{5/2}+u^{3/2}\right)\,du
=\frac27u^{7/2}+\frac25u^{5/2}+C\\
\left.
{\begin{alignedat}{2}
&&u&=x-1\\
&\implies&du&=dx\\
\hline
\text{Also, } &&u&=x-1\\
&\iff& u+1&=\underline{{x}}\end{alignedat}}\right|
&=\frac27(x-1)^{7/2}-\frac25(x-1)^{5/2}+C.\end{align*}
Though the answer above is correct, one often factors the final 
answer:
$$=\frac2{35}(x-1)^{5/2}[5(x-1)-7]+C=\frac2{35}(x-1)^{5/2}(5x-12)+C.$$
\eex
\bex Compute $\ds{\int\frac{x}{\sqrt{2x+1}}\,dx}$.

\underline{Solution}: We will work this problem
twice  using two different
substitutions.  The first is perhaps the most obvious, but the
second has some appeal as well.
\begin{align*}
\int\frac{x}{\sqrt{2x+1}}\,dx&=\int\frac{\frac12(u-1)}{u^{1/2}}\cdot\frac12\,du
=\frac14\int\left(u^{1/2}-u^{-1/2}\right)\,du\\
\left.\begin{alignedat}{2}
&&u&=2x+1\\
&\implies&du&=2\,dx\\
&\implies&\frac12du&=dx\\
\hline
&&u&=2x+1\\
&\implies&x&=\frac12(u-1)
\end{alignedat}\right|&=\frac14\left(\frac23u^{3/2}-2u^{1/2}\right)+C
=\frac16(2x+1)^{3/2}-\frac12(2x+1)^{1/2}+C.
\end{align*}
Again one might factor, simplify and rearrange the variable parts of the answer
to arrive at
$$=\frac16(2x+1)^{1/2}\left[(2x+1)-3\right]+C
  =\frac16\sqrt{2x+1}(2x-2)+C
  =\frac13(x-1)\sqrt{2x+1}+C.$$
For the alternative substitiution we let $u=\sqrt{2x+1}$.  Note how
much of the integrand is then absorbed into $du$ (due to the
relationship between the square root and its derivative).
\begin{align*}
\int\frac{x}{\sqrt{2x+1}}\,dx&=\int\underbrace{\frac12(u^2-1)}_{x}
\underbrace{\,du\,\vphantom{\frac12}}_{\frac{dx}{\sqrt{2x+1}}}
=\frac12\left[\frac13u^3-u\right]+C
=\frac16u^3-\frac12u+C
\\
\left.\begin{alignedat}{2}
&&u&=\sqrt{2x+1}\\
&\implies&du&=\frac1{2\sqrt{2x+1}}\cdot2\,dx\\
&\implies&du&=\frac1{\sqrt{2x+1}}\,dx\\
\hline
&&u&=\sqrt{2x+1}\vphantom{\frac22}\\
&\implies&u^2&=2x+1\\
&\implies&\frac12(u^2-1)&=x\end{alignedat}\right|
&=\frac16\left(\sqrt{2x+1}\right)^3-\frac12\sqrt{2x+1}+C
\text{ (as before).}\end{align*}
\eex


It is important to notice that we used {\it the same} equation
for $u$ to calculate $du$, and to calculate $x$, in all of the above.
Also the reader should begin to see that we can make some rather 
interesting substitutions, so long as we are consistent when
replacing every term inside the integral.  In doing so, it will
become apparent if (1) it is even possible to use a given substitution
to rewrite the integral, and (2) even if so, is the new integral
one which we can actually compute.


\bex Compute $\ds{\int\frac{x^3}{\sqrt{x^2+1}}\,dx}$.


\eex




\subsection{Other Miscellaneous Power Rule Substitutions}
So far we have concentrated on algebraic (polynomial and
rational-power), exponential and trigonometric functions in our 
substitution problems.  It is also worth examining how power
rules can arise from integrals involving logarithmic and
arctrigonometric functions, which we do in this subsection.
\bex Compute $\ds{\int\frac{(\ln x)^5}{x}\,dx}$.

\underline{Solution}: He we see a factor $\frac1x$, which is
the derivative of $\ln x$, so the latter will be $u$:
\begin{align*}
\int\frac{(\ln x)^5}{x}\,dx&=\int u^5\,du=\frac16u^6+C
       =\frac{(\ln x)^6}{6}+C.\\
\left.\begin{aligned}
u&=\ln x\\
du&=\frac1x\,dx\end{aligned}\right|&\end{align*}
\eex
Note that, as a general rule, if we have a function
$f(x)$ with antiderivative $F(x)$, then we have\footnotemark
\begin{align}
\int\frac{f(\ln x)}{x}\,dx&=\int f(u)\,du=F(u)+C=F(\ln x)+C.\\
\left.
\begin{aligned}
u&=\ln x\\
\implies du&=\frac1x\,dx\end{aligned}\right|&\notag
\end{align}
\footnotetext{%%% FOOTNOTE
In fact we can replace $\ln x$ with $\ln|x|$ in throughout the above.
%%% END FOOTNOTE
}
Similar formulas apply to the arctrigonometric functions.
Rather than list and commit to memorize them, it is better
to look at the general idea that if, say,
$\sin^{-1}x$ occurs in an integral, we would look immediately
to see if its derivative, $\frac1{\sqrt{1-x^2}}$, also appears.
Similarly for all functions.
\bex Compute $\ds{\int\frac1{\sqrt{1-x^2}\sin^{-1}x}\,dx}$.

\underline{Solution}: Note here that if $u=\sin^{-1}x$ then
our $du$ below will account for
$\frac1{\sqrt{1-x^2}}\,dx$:
\begin{align*}\int\frac1{\sqrt{1-x^2}\sin^{-1}x}\,dx
&=\int\frac1{u}\,du=\ln|u|+C=\ln\left|\sin^{-1}x\right|+C.\\
\left.\begin{aligned}
u&=\sin^{-1}x\\
\implies du&=\frac1{\sqrt{1-x^2}}\,dx
\end{aligned}\right|&\end{align*}
\eex
\bex Compute $\ds{\int\frac{\sec^{-1}x}{x\sqrt{x^2-1}}\,dx}$.  Assume $x>0$
(or more precisely, $x\ge1$).

\underline{Solution}:
\begin{align*}
\int\frac{\sec^{-1}x}{x\sqrt{x^2-1}}\,dx&=\int u\,du
  =\frac12u^2+C=\frac{\left(\sec^{-1}x\right)^2}2+C.\\
\left.\begin{aligned}
u&=\sec^{-1}x\\
du&=\frac1{x\sqrt{x^2-1}}\,dx\end{aligned}\right|&\end{align*}
\eex
In the example above, if instead $x<0$ (actually $x<-1$), 
we would replace $x$ by $-|x|$ in the denominator of the integrand,
giving eventually $-\frac12(\sec^{-1}x)+C$ for the antiderivative.


\newpage
\begin{center}
\underline{\Large{\bf Exercises}}\end{center}

\begin{multicols}{2}
\begin{enumerate}
\item $\ds{\int\frac{\sec x\tan x}{\sec x}\,dx}$
      \label{IntOfSecXTanX/SecXForFun}
\end{enumerate}
\end{multicols}

\newpage
\section{Second Trigonometric Rules\label{SecondTrigRules}}
We first looked at the simplest trigonometric integration rules---those
arising from the derivatives of the trignometric functions---in
Section~\ref{IndefiniteIntegrals}
(Subsection~\ref{FirstTrigRules}, page~\pageref{FirstTrigRules}
to be more precise).  Here we will complete the trigonomic rules
in which one of the six basic trigonometric functions is the ``outer''
function.  In fact we have four of the six antiderivatives we need:
sine and cosine come quickly from the derivative formulas, and
tangent and cotangent come from substitution arguments.  As it turns
out, secant and cosecant require a little more cleverness, and while
we will not derive these from first principles, we will show that
checking them is a quick and interesting derivative computation.
Unfortunately (or fortunately, whatever your perspective) there
are variations of the antiderivatives of tangent, cotangent, secant
and cosecant.  We will choose one form for each, but the well-informed
student must be aware of the others to be prepared to discuss calculus
topics among students with different backgrounds.\footnote{%
%%% FOOTNOTE
In fact there are no strongly compelling reasons not to use 
\begin{align*}
\int\tan x\,dx&=-\ln|\cos x|+C,\\
\int\cot x\,dx&=\ln|\sin x|+C,\end{align*}
which afterall have slightly simpler verifications by differentiation than
(\ref{AntiDerivTanX}) and (\ref{AntiDerivCotX}).  Here we have opted to
use the latter, 
slightly more difficult formulas for a few reasons.
First, they are themselves
quite popular.  Second, the reader used to (\ref{AntiDerivTanX}) and
(\ref{AntiDerivCotX}) will be less likely to be confused when presented
the simpler alternatives by a colleague (or future professor) with a different
background, while the reader used to those simpler
alternatives may have some initial difficulty if similarly
presented our forms here.  Finally, there is so much added
structure, both calculus and algebraic, found in the context of the
secant and cosecant
functions so  it is important to be familiar and comfortable with them.

Admittedly, however, if (\ref{AntiDerivTanX}) and (\ref{AntiDerivCotX})
were not so common we would likely opt for the simpler forms.
%%% END FOOTNOTE
}

\subsection{Antiderivatives of the Six Trigonometric Functions}
The antiderivatives of the six basic trigonometric functions
are as follow:
\begin{alignat}{2}
\int\sin x\,dx&\ =\ &-&\cos x+C,\label{AntiDerivSinX}\\
\int\cos x\,dx&\ =\ &&\sin x+C,\label{AntiDerivCosX}\\
\int\tan x\,dx&\ =\ &&\ln|\sec x|+C,\label{AntiDerivTanX}\\
\int\cot x\,dx&\ =\ &-&\ln|\csc x|+C,\label{AntiDerivCotX}\\
\int\sec x\,dx&\ =\ &&\ln|\sec x+\tan x|+C,\label{AntiDerivSecX}\\
\int\csc x\,dx&\ =\ &-&\ln|\csc x+\cot x|+C.\label{AntiDerivCscX}
\end{alignat}
The first four of these can be verified mentally through quick derivative
computations if the student is well enough versed in differentiation.
The last two require some more care, but are somewhat interesting to check.
For instance, we can verify (\ref{AntiDerivSecX}) as 
follows:
\begin{align*}
\frac{d\,\ln|\sec x+\tan x|}{dx}
&=\frac1{\sec x+\tan x}\cdot\frac{d\,(\sec x+\tan x)}{dx}\\
&=\frac1{\sec x+\tan x}\cdot(\sec x\tan x+\sec^2x)\\
&=\frac{\sec x(\tan x+\sec x)}{\sec x+\tan x}\\
&=\frac{\sec x(\sec x+\tan x)}{\sec x+\tan x}=\sec x,\text{ q.e.d.}
\end{align*}\label{ProofOfIntegralOfSecant}


It is not entirely obvious how one would derive antiderivatives of 
the secant and cosecant functions, and so it is important to memorize
those especially. Indeed it is likely these were discovered through
experimentation, and such results are often very time consuming
to reproduce from first principles if one has to re-invent ``the trick,''
one of which will be explored in the exercise.
%(The others can be derived by inspection or through
%fairly simple substitution, recalling for instance that
%$\tan x=\sin x/\cos x$ and that $-\ln|\cos x|=\ln|\sec x|$, and so
%on.)  
In fact we will later show a popular alternative antiderivative
for the cosecant, and a not-so-popular alternative for the secant.
The alternatives for the tangent and cotangent are similar in
popularity to those we will use for our standards.

There is little we can do with just 
(\ref{AntiDerivSinX})--(\ref{AntiDerivCscX}) as they stand,
but we nonetheless explore a few examples quickly.

\bex Below are two quick antiderivative computations involving our
basic trigonmetric integral formulas.
\begin{itemize}
\item $\ds{\int\frac{\sin^2 x+\cos x}{\sin x}\,dx
  =\int\left(\sin x+\cot x\right)\,dx=-\cos x-\ln|\csc x|+C}$.
\item $\ds{\int\left(x+\sec x\right)\,dx
  =\frac{x^2}2+\ln|\sec x+\tan x|+C}$.
\end{itemize}
\eex

\bex Suppose $v(t)=1+\tan t$, and $s(\pi/6)=7$.  Find $s(t)$, and
the range of $t$ for which the solution is valid.

\underline{Solution}: We know that $s(t)$ is an antiderivative
of $v(t)$, so we write the following, realizing that we will use
our one datum ($s(\pi/6)=7$) to find the additive constant later.
$$s(t)=\int v(t)\,dt=\int(1+\tan t)\,dt=t+\ln|\sec t|+C.$$
So far $s(t)=t+\ln|\sec t|+C$, and $s({\pi}/6)=7$, so
$$7=\frac{\pi}6+\ln\left|\sec\frac{\pi}6\right|+C\iff 
7=\frac{\pi}6+\ln2+C\iff 7=\frac{\pi}6+\ln2+C,$$
and so $C=7-\frac{\pi}6-\ln2$.  Hence 
$$v(t)=t+\ln|\sec t|+7-\frac{\pi}6-\ln2.$$
\eex

\subsection{Substitution and the Basic Trigonometric Functions, Part I}

A student who uses calculus extensively is likely to eventually
encounter an antiderivative problem where the form is
particularly difficult or obscure, in which case it is common
to refer to so-called tables of integrals.  These usually
contain all the basic forms as well as those which would be
difficult enough to warrant a search through such a reference.\footnote{%
%%% FOOTNOTE
In a typical calculus class, the professor usually
has to answer the question of
why students have to learn all of the difficult integration techniques
when there are references available.  The answer is several-fold,
and we make just a few points addressing it here.

First, many problems are simple enough to not require reference,
and using tables for every problem becomes akin to looking up every word
in a dictionary as one reads a newspaper, for instance. 
Second, it is not even possible to use tables for 
every integration problem simply because 
many problems which can be accomplished through the techniques
will not match what is in the tables.  Third, on occasion there will
be a technicality which the editors of the tables did not anticipate
for a particular problem, or a mistake they did not catch,
and so reliance on tables can be problematic.
(On this last point, the same is true of mathematical software packages
which claim to compute integrals.)

We will eventually make use of tables to a limited extent, to see what
technicalities need to be addressed when using them, and to give some
flavor of what kinds of integrals one can find in standard lists.  Indeed,
every serious calculus student should eventually posess such a reference, 
though again it should be used sparingly.
%%% END FOOTNOTE
}
It is interesting to note that most modern tables of integrals do
not use the common variable $x$ in the formulas, but instead use
$u$, which is the most common variable for substitution type problems.
This is because substitution is so ubiquitous that it is assumed
the reader might not need a form exactly as it is
in the table, but rather needs one which becomes one of the forms 
(or a constant multiple of one of the forms) found in the table
only after a substitution.  In that spirit, the standard method
of listing the antiderivatives of the basic six trigonometric
functions is as follows:
\begin{alignat}{2}
\int\sin u\,du&\ =\ &-&\cos u+C,\label{AntiDerivSinUdU}\\
\int\cos u\,du&\ =\ &&\sin u+C,\label{AntiDerivCosUdU}\\
\int\tan u\,du&\ =\ &&\ln|\sec u|+C,\label{AntiDerivTanUdU}\\
\int\cot u\,du&\ =\ &-&\ln|\csc u|+C,\label{AntiDerivCotUdU}\\
\int\sec u\,du&\ =\ &&\ln|\sec u+\tan u|+C,\label{AntiDerivSecUdU}\\
\int\csc u\,du&\ =\ &-&\ln|\csc u+\cot u|+C.\label{AntiDerivCscUdU}
\end{alignat}
Of course these are just our previous formulas
(\ref{AntiDerivSinX})--(\ref{AntiDerivCscX}), 
from page~\pageref{AntiDerivSinX}, but with the entire
integral written in the variable $u$
instead of $x$.  However, each of these properly interpretted
contains a reverse chain
rule, also known as a substitution-type, form.  So for instance,
if $u=u(x)$, then $du=u'(x)\,dx$ and so we can read
(\ref{AntiDerivTanUdU}) as
$$\int\underbrace{\tan u(x)}_{\tan u}\,\underbrace{u'(x)\,dx}_{du}=
  \ln|\sec u(x)|+C,$$
verified by differentiation:
$$\frac{d}{dx}\ln|\sec u(x)|
 =\frac1{\sec u(x)}\cdot\frac{d\,\sec u(x)}{dx}
 =\frac1{\sec u(x)}\cdot\sec u(x)\tan u(x)\cdot\frac{d\,u(x)}{dx}
 =\tan u(x)\cdot u'(x),$$
q.e.d.  So forms (\ref{AntiDerivSinUdU})--(\ref{AntiDerivCscUdU}) are all
forms in which a basic trigonometric function of some function
$u(x)$, and the derivative $u'(x)$, and the differential $dx$
are the nonconstant factors of the integral.  We now look at several
examples.
%\newpage
\bex Compute $\ds{\int x\sin x^2\,dx}$.

\underline{Solution}: As often occurs, the form is not exact but
a constant multiple of one of our forms, this time (\ref{AntiDerivSinUdU}),
and furthermore the order of the factors is changed.  Here
we see that the factor $x$ is a constant multiple of $u'(x)$ if
$u(x)=x^2$, so the extra factor of $x$ can be ``absorbed'' in the 
differential $du$ after the substitution.  This ultimately leaves
us with the problem of finding the antiderivative of a sine function.
\begin{align*}
\int \underline{x}\sin x^2\,\underline{dx}&=
      \int \sin u\cdot\underline{\frac12\,du}
  =-\frac12\cos u+C=-\frac12\cos x^2+C.\\
\left.\begin{alignedat}{2}
&&u&=x^2\\
&\implies &du&=2\,\underline{x\,dx}\\
&\iff&\frac12\,du&=\underline{x\,dx}\end{alignedat}\right|&\end{align*}
\eex
The above example can be quickly checked by differentiation.
\bex Compute $\ds{\int e^x\cot e^x\,dx}$.

\underline{Solution}:  Here we see the derivative of the argument
$e^x$ of the cotangent function is also present as a multiplicative 
factor.
\begin{align*}
\int \underline{e^x}\cot e^x\,\underline{dx}&=
      \int \cot u\,\underline{du}
  =-\ln|\csc u|+C=-\ln\left|\csc e^x\right|+C.\\
\left.\begin{alignedat}{2}
&&u&=e^x\\
&\implies &du&=\underline{e^x\,dx}\end{alignedat}
\right|&\end{align*}
\eex
\bex Compute $\ds{\int\frac{\sec\sqrt{x}}{\sqrt x}\,dx}$.

\underline{Solution}: Here the factor $\frac{1}{\sqrt{x}}$
is in fact a constant multiple of the derivative of 
$\sqrt{x}$, the argument of the trigonometric function.
Thus we take $u=\sqrt{x}$, and then the resulting $du$ will
absorb the $\frac1{\sqrt{x}}$ term:
\begin{align*}
\int\frac{\sec\sqrt{x}}{\sqrt x}\,dx&=
  \int\sec\sqrt{x}\cdot\underline{\frac1{\sqrt{x}}\,dx}
  =\int \sec u\cdot\underline{2\,du}
  =2\ln|\sec u+\tan u|+C\\
\left.\begin{alignedat}{2}
&        &u&=\sqrt x\\
&\implies&du&=\frac1{2\sqrt{x}}\,dx\\
&\iff &2\,du&=\underline{\frac1{\sqrt{x}}\,dx}\end{alignedat}\right|
&=2\ln\left|\sec\sqrt{x}+\tan\sqrt{x}\right|+C.
\end{align*}
\eex

\bex Compute $\ds{\int\frac{\cos(1+4\ln x)}{x}\,dx}$.

\underline{Solution}: Here we see the derivative of $(1+4\ln x)$ appearing
as a factor as well, except for a constant factor.
\begin{align*}
\int\frac{\cos(1+4\ln x)}{x}\,dx&=\int\cos(1+4\ln x)\cdot
               \underline{\frac1x\,dx}
               =\int\cos u\cdot\underline{\frac14\,du}=\frac14\sin u+C\\
\left.\begin{alignedat}{2}
&         &u&=1+4\ln x\\
&\implies&du&=4\cdot\underline{\frac1x\,dx}\\
&\iff&\frac14\cdot du&=\underline{\frac1x\,dx}
\end{alignedat}\right|
&=\frac14\sin(1+4\ln x)+C.\end{align*}
\eex

\bex Compute $\ds{\int x^2\csc\left(\cos x^3\right)\sin x^3\,dx}$.

\underline{Solution}:  To be clear, first
we note that the integrand is the product of three
factors:
$$x^2\cdot\csc\left(\cos x^3\right)\cdot\sin x^3,$$
so the argument of the cosecant is $\cos x^3$.
Now we will compute this two different ways.
The first method requires two substitions, which is an option that
students must be aware is legitimate, assuming all computations are
made carefully and consistently.
\begin{description}
\item[Method 1.] Here we will first make a substitution $u=x^3$
to yield a simpler integral without any polynomial factors, 
though our new integral will still require some
work.
\begin{align*}
\int \underline{x^2}\csc\left(\cos x^3\right)\sin x^3\,\underline{dx}
 &=\int\csc(\cos u)\sin u\cdot\frac13\,du\\
\left.\begin{alignedat}{2}
&          &u&=x^3\\
&\implies &du&=3\,\underline{x^2\,dx}\\
&\iff&\frac13\,du&=\underline{x^2\,dx}
\end{alignedat}\right|&
\end{align*}
So at this point our problem reduces to computing
$\ds{\int\csc(\cos u)\sin u\cdot\frac13\,du}$.  
To do so we use another substitution, noting that
$\sin u$ is the derivative---up to a multiplicative constant---of
$\cos u$ (with respect to $u$ this time).
To remain consistent this second substitution must use a new
variable (lest we give one letter two different meanings within the
same problem, which would be contradictory!).
So we  call our new variable something other than $u$ or $x$.
A commonly used variable at this stage is $w$:
\begin{align*}
\frac13\int\csc(\cos u)\underline{\sin u\,du}
&=\frac13\int\csc w\underline{(-1)\,dw}
=-\frac13\left[-\ln|\csc w+\cot w|\right]+C\\
\left.\begin{alignedat}{2}
           &&w&=\cos u\\
 &\implies&dw&=-\underline{\sin u\,du}\\
 &\iff&(-1)dw&=\underline{\sin u\,du}
\end{alignedat}\right|
&=\frac13\ln|\csc w+\cot w|+C\\
&=\frac13\ln\left|\csc\left(\cos u\right)
      +\cot \left(\cos u\right)\right|+C\\
&=\frac13\ln\left|\csc\left(\cos x^3\right)
      +\cot \left(\cos x^3\right)\right|+C.\end{align*}
Note how we computed the antiderivative in $w$, which we
then replaced by its expression in $u$, and finally by
the definition of $u$ in terms of $x$.
\item[Method 2.] If we can see far enough ahead, we can combine
both substitutions into one.  For clarity we will use a different
variable---namely $z$---here (though by convention
one would usually use $u$).  We choose $z=\cos x^3$, noting
that its derivative, requiring the chain rule, will have
a $\sin x^3$ and a $x^2$ term (ignoring multiplicative constants),
which leaves us with a constant mulitiple of $\int\csc z\,dz$, for
which we have a formula.  
\begin{align*}
\int \underline{x^2}\csc\left(\cos x^3\right)\underline{\sin x^3\,dx}
 &=\int\csc z\cdot\underline{\frac{-1}3\,dz}
 =-\frac13\cdot[-\ln|\csc z+\cot z|]+C\\
\left.\begin{alignedat}{2}
& &z&=\cos x^3\\
&\implies &dz&=-\underline{\sin x^3}\cdot3\,\underline{x^2\,dx}\\
&\iff&-\frac13\,dz&=\underline{x^2\sin x^3\,dx}\end{alignedat}\right|
&=\frac13\ln\left|\csc\left(\cos x^3\right)
                  +\cot\left(\cos x^3\right)\right|+C.\end{align*}
\end{description}
\eex

In the previous example, the second method in fact just combines
the two substitutions from the first method into one.  Indeed,
the formula for $w$ in terms of $x$ is the same as that of 
$z$:\footnote{%
%%% FOOTNOTE
Similarly, though perhaps not so obviously, when we recall
these variables are all functions of $x$ we also have $dw=dz$:
\begin{align*}
dw&=\frac{dw}{du}\cdot\frac{du}{dx}\cdot dx\\
  &=(-\sin u)\cdot\left(3x^2\right)\,dx\\
  &=-\sin x^3\cdot 3x^2\,dx\\
  &=dz.\end{align*}
Again we see the power of the Leibniz notation in what is essentially
a chain rule.  Of course we should expect that $w=z\implies dw=dz$.
But also we see that while there are obvious algebraic consistencies
in our substitution method, there are also consequent
calculus consistencies which, while more subtle, are still correct
when we perform all the computations correctly.
%%% END FOOTNOTE
}
$$w=\csc(u)=\csc\left(\cos x^3\right)=z.$$
When one is well practiced in substitution the second method will
likely be chosen.  However, it is important also for the student
to realize that even if a substitution does not achieve an
integral that can be immediately computed, that does not mean
that the particular substitution need be abandoned.  If the
new integral is simpler, then the
first substitution can be worthwhile.  In fact in later sections
we will on occasion {\it require} multiple substitutions.
Of course it is important that all steps be carried out 
carefully, accurately and consistently.  

In this section we concentrated on those integrals which reduce
to integrals of a single trigonometric function, perhaps with the
aid of a substitution. In 
Chapter~\ref{AdvancedIntTechniques}
(and more sepcifically Section~\ref{TrigIntsSection})
we will look at the many techniques for computing those integrals 
which contain several factors of trigonometric functions, and no
other factors.  The techniques of our present section will
be called upon often, but these are only a small part of the 
needed knowledge for computing the  ``trignometric integrals''
of the later sections.  But in fact we have some other techniques
already.  For instance, there were the first trigonometric integral
formulas we had in Section~\ref{IndefiniteIntegrals},
Subsection~\ref{FirstTrigRulesSubsection} which arose from
the derivative rules for the six basic trigonometric functions.
In fact we had one other technique for dealing with some trigonometric
integrals, which was substitution in the case we could rewrite
the trigonometric integral as a power-rule type integral.\footnote{%
%%% FOOTNOTE
In fact there will be several other substitution type arguments we
will make for trigonometric integrals besides those which yield power
rules.
%%% END FOOTNOTE
}
\bex Compute $\ds{\int\frac{\sin3x}{\cos^53x}\,dx}$.

\underline{Solution}:  Here we see that we have
the derivative of the cosine function is present as a factor,
and we are left with a power of the cosine:
\begin{align*}
\int\frac{\sin3x}{\cos^53x}\,dx
 &=\int(\cos3x)^{-5}\underline{\sin3x\,dx}
  =\int u^{-5}\cdot\underline{\frac{-1}3\,du}\\
\left.\begin{alignedat}{2}
&&u&=\cos3x\\
&\implies&du&=-\underline{\sin3x}\cdot3\,\underline{dx}\\
&\iff&\frac{-1}3\,du&=\underline{\sin3x\,dx}
\end{alignedat}\right|
&=-\frac13\cdot\frac{-1}4\,u^{-4}+C\\
&=\frac1{12}\cos^{-4}3x+C\\
&=\frac1{12}\sec^{4}3x+C.
\end{align*}
\eex
The integration techniques we encounter throughout the book are
many and varied.  We will see later how a slight change in a problem
can substantially change the result, its difficulty,
or the technique used to achieve it.  We have seen this phenomenon
before.  Consider for instance
\begin{align*}
\int\frac{x}{x^2+1}\,dx&=\frac12\ln\left(x^2+1\right)+C,\\
\int\frac{1}{x^2+1}\,dx&=\tan^{-1}x+C,\\
\\
\int\frac{x}{\sqrt{1-x^2}}\,dx&=-\sqrt{1-x^2}+C,\\
\int\frac1{\sqrt{1-x^2}}\,dx&=\sin^{-1}x+C,\\
\\
\int\sec x\,dx&=\ln|\sec x+\tan x|+C,\\
\int\sec^2x\,dx&=\tan x+C,\\
\int\sec^3x\,dx&=\frac12(\sec x\tan x+\ln|\sec x+\tan x|)+C.
\end{align*}
In fact this last problem will have to wait until 
Chapter~\ref{AdvancedIntTechniques}, and is quite long and technical.
Even the verification by differentiation is nontrivial, and
requires one to employ some trigonometric identity along the way.
Suffice for now 
to simply note that the techniques, and results,
for even these first three powers of
the secant are all very different.

Computing antiderivatives in good time requires 
the ability to recognize which technique will work for a 
particular problem.  That in turn requires 
a fairly complete knowledge of the techniques, even to the
extent that one can anticipate the outcomes of several later steps.
Of course practice is one key to gaining this understanding.
For that reason this chapter will contain one
section in which the exercises' required
techniques are purposely randomized, by method as well as difficulty.
\begin{center}
\underline{\Large{\bf Exercises}}\end{center}
\bigskip
\begin{multicols}{2}
\begin{enumerate}
\item By differentiation, verify each of our basic six trigonometric
 integrals in the forms we use,
 (\ref{AntiDerivSinX})--(\ref{AntiDerivCscX}).  For reference
 see the proof for the secant, page~\pageref{ProofOfIntegralOfSecant}.
 \begin{enumerate}
 \item $\ds{\int\sin x\,dx=-\cos x+C}$
 \item $\ds{\int\cos x\,dx=\sin x+C}$
 \item $\ds{\int\tan x\,dx=\ln|\sec x|+C}$
 \item $\ds{\int\cot x\,dx=-\ln|\csc x|+C}$
 \item $\ds{\int\sec x\,dx=\ln|\sec x+\tan x|+C}$
 \item $\ds{\int\csc x\,dx=-\ln|\csc x+\cot x|+C}$
 \end{enumerate}
\item Compute the following integrals.
 \begin{enumerate}
 \item $\ds{\int x\sec\left(x^2+1\right)\,dx}$
 \item $\ds{\int \frac{\tan(\ln x)}{x}\,dx}$
 \item $\ds{\int \frac{\cos\left(\frac1x\right)}{x^2}\,dx}$
 \item $\ds{\int \sqrt{x}\csc\left(x\sqrt{x}\right)\,dx}$
 \item $\ds{\int x^3e^{5x^4}\cot \left(6e^{5x^4}\right)\,dx}$
 \end{enumerate}



\item Derive our formula (above) for the integral of
$\sec x$ by the following algebraic device, namely multiplying
and dividing by $\sec x+\tan x$ within the integral, i.e.,
$$\int\sec x\,dx
=\int\sec x\cdot\frac{\sec x+\tan x}{\sec x+\tan x}\,dx,$$
and then using an appropriate substitution argument.
\item By differentiation, verify each of the following alternative
 integration formulas.
 \begin{enumerate}
 \item $\ds{\int\tan x\,dx=-\ln|\cos x|+C}$.
 \item $\ds{\int\cot x\,dx=\ln|\sin x|+C}$.
 \item $\ds{\int\sec x\,dx=-\ln|\sec x-\tan x|+C}$.
 \item $\ds{\int\csc x\,dx=\ln|\csc x-\cot x|+C}$.
 \end{enumerate}

\end{enumerate}
\end{multicols}



\newpage
\section{Substitution with All Basic Forms%
\label{SubWithAllForms}}
In this section we will add to our forms for substitution
and recall some rather general guidelines for substitution.
Except for our four new trigonometric forms
from Section~\ref{SecondTrigRules},
all forms in this chapter derive directly from 
derivative rules.  These comprise
what we call here the {\it basic} integration rules.  Each
is based upon a single function specific to the rule. So for
instance, we will have in our list the following:
$$\int\frac1{u^2+1}\,du=\tan^{-1}u+C.$$
As before, the usual variable of integration in the given problem
will likely be $x$, but the {\it form} may be ultimately as
above, except for multiplicative constants, where $u=u(x)$ and then
$du=u'(x)\,dx$ contains another factor from the original integral.
So for instance we might see
\begin{align*}
\int\frac{x}{x^4+1}\,dx&=\int\frac1{\left(x^2\right)^2+1}
                            \cdot\underline{x\,dx}
                        =\int\frac{1}{u^2+1}\cdot\underline{\frac12\,du}
                        =\frac12\tan^{-1}u+C\\
\left.\begin{alignedat}{2}
 &&u&=x^2\\
&\implies&du&=2\,\underline{x\,dx}\\
&\iff&\frac12\,dx&=\underline{x\,dx}\\
\end{alignedat}\right|
&=\frac12\tan^{-1}x^2+C.\end{align*}
One clue that we might try $u=x^2$ was that its derivative was a factor
in the integrand in the form of the factor $x$,
again excepting multiplicative constants, and so we wrote that
factor separately next to the differential $dx$.
As it turned out, the rest of the integrand could indeed be written
as a function of $u=x^2$.  

Reading the problem above backwards,
the arctangent is the ``outer function'' of a chain
rule differentiation problem, and $x^2$ was the ``inner function.''
Put in terms of integration, the {\it form} was $\int\frac1{u^2+1}\,du$,
excepting multiplicative constants, with the ``inner function'' 
$u=x^2$.  The arctangent appeared because of the ultimate form
of the integral, in terms of $u=x^2$.

But note that the arctangent can also appear as the ``inner function,''
which we may wish to set equal to $u$. So for instance,
\begin{align*}
\int\frac{\left(\tan^{-1}x\right)^2}{x^2+1}\,dx
 &=\int\left(\tan^{-1}x\right)^2\cdot\underline{\frac1{x^2+1}\,dx}
 =\int u^2\,\underline{du}=\frac{u^3}3+C\\
\left.\begin{alignedat}{2}
 &&u&=\tan^{-1}x\\
&\implies&du&=\underline{\frac1{x^2+1}\,dx}\end{alignedat}\right|
&=\frac{\left(\tan^{-1}x\right)^3}3+C.\end{align*}

In all these cases, we are looking for some $u=u(x)$ so that
\begin{itemize}
\item one major (nonconstant) factor of the integral can be simply 
 written $f(u)$, i.e., $u$ is an ``inner function'' of some
 composite function $f(u(x))$ which appears in the integrand,
\item the remaining factors of the integrand will collectively
  be a constant multiple of $du=u'(x)\,dx$, 
\item and finally, so that $\int f(u)\,du$ is an integral we can
  compute, i.e., we know the antiderivative of $f$.
\end{itemize}
So of course identifying $u$ is the key, and in doing so we have to
be sure its derivative is also present, and finally that we are
left with an integral---albeit in $u$---which we can handle.\footnote{%
%%% FOOTNOTE
This all assumes that there is a substitution which will make
the integral into one of these simple forms.  It is not always the
case.  One which occurs in probability and other subjects
is $\int e^{x^2}\,dx$, which can not be changed by substitution into
a useful form.  In fact it will not succumb to any of the methods in
this or the next chapter.  We will eventually find a way to 
deal with this integral, in Chapter~\ref{TaylorSeriesChapter}.
In the meantime it is actually a good exercise to see why this
integral can not be forced into any of our methods.  Indeed, 
seeing what goes wrong in such a case very well complements
seeing what goes right in the cases where substitution,
and later methods, do achieve an answer.%%%
%%% END FOOTNOTE
}
\subsection{List of Basic Forms}

\begin{align}
\int u^n\,du&=\frac{u^{n+1}}{n+1}+C,\qquad u\ne-1,\label{IntPwrRlInU}\\
\int\frac1u\,du&=\ln|u|+C,\label{Int1/URl}\\
\int\sin u\,du&=-\cos u+C,\label{IntSinURl}\\
\int\cos u\,du&=\sin u+C,\label{IntCosURl}\\
\int\tan u\,du&=\ln|\sec u|+C,\label{IntTanURl}\\
\int\cot u\,du&=-\ln|\csc u|+C,\label{IntCotURl}\\
\int\sec u\,du&=\ln|\sec u+\tan u|+C,\label{IntSecURl}\\
\int\csc u\,du&=-\ln|\csc u+\cot u|+C,\label{IntCscURl}\\
\int\frac1{\sqrt{1-u^2}}\,du&=\sin^{-1}u+C,\label{IntGivingArcSinURl}\\
\int\frac1{u^2+1}\,du&=\tan^{-1}u+C,\label{IntGivingArcTanURl}\\
\int\frac1{u\sqrt{u^2-1}}\,du&=\sec^{-1}|u|+C,\label{IntGivingArcSec|U|Rl}\\
\int e^u\,du&=e^u+C,\label{IntGivingExp(U)Rl}\\
\int a^u\,du&=\frac{a^u}{\ln a}+C,\label{IntA^URl}\\
\int \sec^2u\,du&=\tan u+C,\label{IntSec^2URl}\\
\int\csc^2u\,du&=-\cot u+C,\label{IntCsc^2URl}\\
\int\sec u\tan u\,du&=\sec u+C,\label{IntSecUTanURl}\\
\int\csc u\cot u\,du&=-\csc u+C.\label{IntCscUCotURl}\end{align}


It can not be stressed too much that each form given assumes that
a substitution may be required.  So again, the following formulas
say the same:
$$\int e^u\,du=e^u+C,\qquad\qquad \int e^{u(x)}u'(x)\,dx=e^{u(x)}+C.$$
Recognizing when we have such a form is again key to using these formulas.

\bex Compute $\ds{\int x e^{x^2}\,dx}$.

\underline{Solution}:  Here we see the derivative of $x^2$ appearing
as the factor $x$, except for a constant multiple.  Hence we let $u=x^2$,
and the $du$ will contain the other factor $x$, leaving
an integral in one of our standard forms, namely (\ref{IntGivingExp(U)Rl}),
and nothing else except
a multiplicative constant.
\begin{align*}\int \underline{x}\, e^{x^2}\,\underline{dx}
 &=\int e^u\cdot\frac12\,du
  =\frac12\,e^u+C\\
\left.\begin{alignedat}{2}
&&u&=x^2\\
&\implies&du&=2\,\underline{x\,dx}\\
&\iff&\frac12\,du&=\underline{x\,dx}
\end{alignedat}\right|
&=\frac12\,e^{x^2}+C.
\end{align*}
\label{IntXE^x^2Example}\eex
%\bex Compute $\ds{\int\frac{\sqrt{\tan^{-1}x}}{x^2+1}\,dx}$.%%
%
%\underline{Solution}:  Here we see the derivative of $\tan^{-1}x$,
%namely $\frac1{x^2+1}$, appearing as a factor, so letting $u=\tan^{-1}x$
%will result in $du$ accounting for this other factor.
%\begin{align*}
%\int\frac{\sqrt{\tan^{-1}x}}{x^2+1}\,dx
%&=\int\sqrt{\tan^{-1}x}\cdot\underline{\frac1{x^2+1}\,dx}
%=\int\sqrt{u}\,du=\int u^{1/2}\,du\\
%\left.\begin{alignedat}{2}
%&&u&=\tan^{-1}x\\
%&\implies&du&=\underline{\frac1{x^2+1}\,dx}\end{alignedat}
%\right|&=
%\end{align*}
%\eex
\bex Compute $\ds{\int\frac{x^3}{\sqrt{1-x^8}}\,dx}$.

\underline{Solution}: At first it is tempting to let $u=1-x^8$
and hope this will become a power rule,
except that such $u$ implies $du=-8x^7$,
 which is very different from a constant
multiple of the other factor here, namely $x^3$.  

In fact, the other factor can be a good source of information
about what to set equal to $u$.  Indeed the factor $x^3$ will
be part of the differential of $u=x^4$, and then we can recognize
a form which will yield an arcsine ultimately, i.e., form
(\ref{IntGivingArcSinURl}).
\begin{align*}
\int\frac{x^3}{\sqrt{1-x^8}}\,dx
&=\int\frac{1}{\sqrt{1-\left(x^4\right)^2}}\cdot\underline{x^3\,dx}
=\int\frac1{\sqrt{1-u^2}}\cdot\underline{\frac14\,du}
=\frac14\,\sin^{-1}u+C\\
\left.\begin{alignedat}{2}
&&u&=x^4\\
&\implies&du&=4\,\underline{x^3\,dx}\\
&\iff&\frac14,du&=\underline{x^3\,dx}
\end{alignedat}\right|&=\frac14\sin^{-1}\left(x^4\right)
                      +C.
\end{align*}
\eex

As with the power rule, there are occasions where the derivative
of our $u$ is a nonzero constant, and thus a constant multiple of every
other nonzero constant.  While these integrals are arguably easier than
the others we encounter here, their relative simplicity can be
a source of confusion.

\bex Compute $\ds{\int\csc5x\,dx}$.

\underline{Solution}: Here we simply let $u=5x$.
\begin{align*}
\int\csc5x\,\underline{dx}&=\int\csc u\cdot\underline{\frac15\,du}
      =-\frac15\ln|\csc u+\cot u|+C\\
\left.\begin{alignedat}{2}
&&u&=5x\\
&\implies&du&=d\,\underline{dx}\\
&\iff\frac15\,du&=\underline{dx}
\end{alignedat}\right|&=-\frac15\ln|\csc5x+\cot 5x|+C.
\end{align*}
\eex

\bex Compute $\ds{\int 2^{3^x}\cdot 3^x\,dx}$.

\underline{Solution}:  Here we will need (\ref{IntA^URl}) eventually,
but first we simply notice that the factor $3^x$ is 
a constant multiple of the exponent in the first factor,
so we let $u=3^x$.
\begin{align*}
\int 2^{3^x}\cdot\underline{3^x\,dx}
&=\int 2^u\cdot\underline{\frac{1}{\ln 3}\,dx}
=\frac{1}{\ln 3}\cdot\frac1{\ln 2}\cdot 2^u+C\\
\left.\begin{alignedat}{2}
&&u&=3^x\\
&\implies&du&=\underline{3^x}\,\ln 3\,\underline{dx}\\
&\iff&\frac1{\ln 3}\,du&=3^x\,dx\end{alignedat}\right|
&=\frac{2^{3^x}}{\ln3\cdot\ln2}+C.
\end{align*}
\eex
The example above shows the importance of following the various
constant factors through the integration.  Students who rely
upon guessing the answers, without performing the
formal substitution steps, are much more likely to misplace
one or more constant factors.

For further practice we consider more basic examples.

\bex Compute $\ds{\int\frac{\sec\sqrt x}{\sqrt{x}}\,dx}$.

\underline{Solution}: The key here is that the derivative
of the argument of the secant is also present as a factor.
Recall $\frac{d}{dx}\left(\sqrt{x}\right)=\frac1{2\sqrt{x}}$,
which is obvious when the radicals are written as 1/2-powers.
\begin{align*}
\int\frac{\sec\sqrt x}{\sqrt{x}}\,dx
&=\int\sec{\sqrt{x}}\cdot\underline{\frac1{\sqrt{x}}\,dx}
=\int\sec u\cdot\underline{2\,du}
=2\ln|\sec u+\tan u|+C\\
\left.\begin{alignedat}{2}
&&u&=\sqrt{x}\\
&\implies&du&=\frac12\cdot\underline{x^{-1/2}\,dx}
\end{alignedat}\right|&=2\ln\left|\sec\sqrt{x}+\tan\sqrt{x}\right|+C.
\end{align*}


\eex



















In the next section we will further explore the arctrigonometric
antiderivatives by considering further complications.
For now we only look at two such complications, first involving the
arctangent (though the arcsine has a similar potential complication),
and then the arcsecant, which has the same complications as the others
and then one more.
\bex Compute $\ds{\int\frac{x^2}{1+25x^6}\,dx}$.

\underline{Solution}: There are two clues directing our choice of $u$.
First, we see the factor $x^2$, which is a multiple of the derivative
of $x^3$.  Then we see that the denominator can be written
as $1+\left(5x^3\right)^2$, which we can put into the form
yielding the arctangent, namely (\ref{IntGivingArcTanURl}).
\begin{align*}
\int\frac{x^2}{1+25x^6}\,dx
&=\int\frac{1}{1+\left(5x^3\right)^2}\cdot\underline{x^2\,dx}
=\int\frac1{1+u^2}\cdot\underline{\frac1{15}\,du}
=\frac1{15}\,\tan^{-1}u+C\\
\left.\begin{alignedat}{2}
  &&u&=5x^3\\
&\implies&du&=15\,\underline{x^2\,dx}\\
&\iff&\frac1{15}\,du&=\underline{x^2\,dx}
\end{alignedat}\right|&=\frac1{15}\tan^{-1}\left(5x^3\right)+C.
\end{align*}
\eex
The complication in this example is fairly benign: that the $u$ term
contains a multiplicative constant.  Here we wanted $25x^6$ to be
$u^2$ for the form (\ref{IntGivingArcTanURl}), so we took 
$u=5x^3$.\footnote{%%%
%%% FOOTNOTE
Note that we could also have used $u=-5x^3$, but then $du=-15x^2\,dx$,
and so our
answer would ultimately be 
(as the reader should verify) 
$-\frac{1}{15}\tan^{-1}\left(-5x^3\right)+C$.
In fact that is the same as the answer we got, since the arctangent
is an ``odd'' function, that is, $\tan^{-1}(-z)=-\tan^{-1}z$.
%%% END FOOTNOTE
}
Fortunately this was consistent with the $du$ containing the $x^2$ term
of the integrand, and that form (\ref{IntGivingArcTanURl})
could actually be used.  In the next section we will
see how to deal with cases where the additive constant in the denominator
of the integrand is not $1$.  For now we look at another complication
which is somewhat specific to the arcsecant form.

\bex Compute $\ds{\int\frac1{x\sqrt{9x^2-1}}\,dx}$.

\underline{Solution}: Because a new complication needs to be explained
while the problem is solved, the organization will be slightly different
than previous exercises, but every technique used below has appeared
previously. Note that we are trying to fit this integral into
form (\ref{IntGivingArcSec|U|Rl}).

Here we want $u^2=9x^2$ so we have $\sqrt{u^2-1}$ as one factor in
the denominator of our integrand.  Thus will let 
$u=3x\implies du=3\,dx\iff\frac13\,du=dx$.  Our integral so far is then
$$\int\frac1{x\sqrt{u^2-1}}\,\cdot\frac13\,du.$$
Now, all of our integral formulas require just one variable, but
in fact the integral above makes sense because of the relationship 
between $x$ and $u$.  But to use a formula we have to put it all into the
new variable, namely $u$.  So there is one term left, which is the 
factor $x$ on the bottom, which has to be put into $u$-terms.
For that we go back to our original substitution and note that
$u=3x\iff x=\frac13\,u$.  Now we continue:
\begin{align*}
\int\frac1{x\sqrt{9x^2-1}}\,dx&=\int\frac1{x\sqrt{u^2-1}}\,\cdot\frac13\,du\\
 &=\int\frac1{\frac13\,u\sqrt{u^2-1}}\cdot\frac13\,du\\
 &=\int\frac1{u\sqrt{u^2-1}}\,du\\
 &=\sec^{-1}|u|+C\\
 &=\sec^{-1}|3x|+C.\end{align*}
\label{FirstWackySecantU-SubProblem}\eex


\newpage
\begin{center}
\underline{\Large{\bf Exercises}}\end{center}
\bigskip
\begin{multicols}{2}
\begin{enumerate}
\item Compute $\ds{\int x e^{x^2}\,dx}$ two ways.
 \begin{enumerate}
 \item Let $u=x^2$ (as in Example~\ref{IntXE^x^2Example}).
 \item Instead let $u=e^{x^2}$.
 \end{enumerate}
\item In the spirit of the previous exercise,
      compute $\ds{\int\frac{e^{1/x}}{x^2}\,dx}$
      two ways, i.e., using two different substitutions.
\item Compute $\ds{\int\sec^2x\tan x\,dx}$
      two different ways.
 \begin{enumerate}
 \item Let $u=\tan x$.
 \item Instead let $u=\sec x$.
 \item Explain why the two answers are equivalent.
 \end{enumerate}
\end{enumerate}\end{multicols}


\newpage
\section{Further Arctrigonometric Forms}

Here we will still use the same arctrigonometric forms
we had before, namely
\begin{align}
\int\frac1{\sqrt{1-u^2}}\,du&=\sin^{-1}u+C,\label{IntGivingArcSinUR2}\\
\int\frac1{u^2+1}\,du&=\tan^{-1}u+C,\label{IntGivingArcTanUR2}\\
\int\frac1{u\sqrt{u^2-1}}\,du&=\sec^{-1}|u|+C.\label{IntGivingArcSec|U|R2}
\end{align}
What is different in this section is that our integrals will
need to be algebraically rewritten into these forms, and this will
require more than the previous substitution.

Each of the integrals in (\ref{IntGivingArcSinUR2}), 
(\ref{IntGivingArcTanUR2}) and (\ref{IntGivingArcSec|U|R2})
need to be exactly as they are stated.  For instance, replacing
$1-u^2$ with $1+u^2$ or $u^2-1$ in  (\ref{IntGivingArcSinUR2})
will give completely different antiderivatives.  In fact, even
the domain of the integrand would be completely different
with any such changes!  Similar changes would substantially alter
the results in
(\ref{IntGivingArcTanUR2}) and (\ref{IntGivingArcSec|U|R2}).

In this section we will have integrands which we can algebraically
rewrite so they conform to one of the forms (\ref{IntGivingArcSinUR2}), 
(\ref{IntGivingArcTanUR2}) or (\ref{IntGivingArcSec|U|R2}).
In fact there are only a couple of algebraic ``tricks'' which we
introduce here.  The first of these is to force the denominators
to have the additive constant $1$, where originally there may
be another constant.  This is accomplished through simple factoring
techniques.  The second technique is ``completing the square,''
where appropriate, and then using the first technique to 
finish rewriting the integrand.  With substitution there will
often be further multiplicative constants to accommodate as well.

\subsection{Factoring to Achieve ``1''}
Each of our integrals (\ref{IntGivingArcSinUR2}), 
(\ref{IntGivingArcTanUR2}) and (\ref{IntGivingArcSec|U|R2})
have the number $1$ conspicuously appearing in the denominator,
near the $u^2$ term.  Any other nonzero {\it constant} there
will have an effect on the vertical and horizontal scaling
of the function in ways we can not ignore in the formula.
To compensate is fairly straightforward: factor the constant,
and see what should be called ``$u^2$.''

\bex Compute $\ds{\int\frac1{9+x^2}\,dx}$.

\underline{Solution}: Our first priority is to rewrite this so
we have a form $1+u^2$ in the denominator.
$$\int\frac1{9+x^2}\,dx=\int\frac1{9\left(1+\frac{x^2}9\right)}\,dx
                       =\int\frac19\cdot\frac1{1+\frac{x^2}9}\,dx.$$
The factor $\frac19$ can simply be carried along for the rest of the
computation.  The denominator of the other factor can be
written $1+u^2$  (same as $u^2+1$ in our formula) if we
take $u^2=\frac{x^2}9$, which can be accomplished letting $u=\frac{x}3$.
\begin{align*}
\int\frac1{9+x^2}\,dx&=\frac1{9}\cdot\int\frac1{1+\frac{x^2}9}\,\underline{dx}
                    =\frac1{9}\int\frac{1}{1+u^2}\cdot\underline{3\,du}\\
\left.\begin{alignedat}{2}
&&u&=\frac{x}3\\
&\implies&du&=\frac13\,\underline{dx}\\
&\iff&3du&=\underline{dx}
\end{alignedat}\right|
&=\frac39\tan^{-1}u+C=\frac13\tan^{-1}\left(\frac{x}3\right)+C.\end{align*}
\eex
The above example illustrated much of the process:  algebraically
manipulate by factoring to achieve ``$1$'' in the appropriate place,
and then pick $u$ so the other term is $u^2$.  It is slightly
more complicated with the forms yielding arcsine and arcsecant, 
due to the presence of the radical.  In the next example
we will show more detail than one might normally write.

\bex Compute $\ds{\int\frac1{\sqrt{25-4x^2}}\,dx}$.

\underline{Solution}: Here we must find a way to replace the 
constant 25 with 1 instead.  We factor as before, but respect
the operation of the radical as well.
$$\int\frac1{\sqrt{25-4x^2}}\,dx
=\int\frac1{\sqrt{25\left(1-\frac{4x^2}{25}\right)}}\,dx
=\int\frac1{\sqrt{25}\sqrt{1-\frac{4x^2}{25}}}\,dx
=\int\frac15\cdot\frac1{\sqrt{1-\frac{4x^2}{25}}}\,dx.$$
So the factor 25 under the radical becomes the factor 5 outside
the radical.  Otherwise it is the same process as the previous
example.  Now we continue, using a substitution which will
result in $u^2=\frac{4x^2}{25}$.  For simplicity we take $u=\frac{2x}5$.
\begin{align*}
\int\frac1{\sqrt{25-4x^2}}\,\underline{dx}
 &=\frac15\int\frac1{\sqrt{1-\frac{4x^2}{25}}}\,\underline{dx}
 =\frac15\int\frac1{\sqrt{1-u^2}}\cdot\underline{\frac52\,du}\\
\left.\begin{alignedat}{2}
&&u&=\frac{2x}5\\
&\implies&du&=\frac25\cdot\underline{dx}\\
&\iff&\frac52\,du&=\underline{dx}\end{alignedat}\right|
&=\frac12\sin^{-1}u+C=\frac12\sin^{-1}\left(\frac{2x}5\right)+C.
\end{align*}
\eex
This example above again illustrates the role of the number $1$
in the denominator, but also
suggests a couple of new points that we make here.
First, it is not obvious where the factors $2$ and $5$---being
the square roots of the $4$ and $25$ appearing in the original---will
be present in the final answer.  There is a pattern for the
arctangent form, and a different one for the arcsine form, but
patterns can be forgotten if not used often enough, where the logic
of manipulating the integral algebraically to get one of the
three basic arctrigonometric forms should still be reproducible
after the patterns---which we will explore at the end of this 
section---are forgotten.  Second, we are approaching the
boundary between integrals which are easily checked with 
differentiation, and those where the differentiation has at least
as many algebraic difficulties as the integration.  In such cases,
it is usually better to have carefully written each integration step so it
can be audited for accuracy, rather than risk algebraic error in
testing our answer.  Consider a verification of the answer in this
latest example (readers' steps may vary):
\begin{align*}
\frac{d}{dx}\left[\frac12\sin^{-1}\left(\frac{2x}5\right)\right]
&=\frac12\cdot\frac1{\sqrt{1-\left(\frac{2x}{5}\right)^2}}
 \cdot\frac{d}{dx}\left[\frac{2x}5\right]
=\frac12\cdot\frac1{\sqrt{1-\frac{4x^2}{25}}}\cdot\frac25\\
&=\frac15\cdot\frac1{\sqrt{1-\frac{4x^2}{25}}}
=\frac1{\sqrt{25}}\cdot\frac1{\sqrt{1-\frac{4x^2}{25}}}
=\frac1{\sqrt{25-4x^2}},\qquad\text{q.e.d.}
\end{align*}
While such a verification is certainly possible, it is not likely one
to be performed ``mentally'' with much confidence,
as we may have been able to do with
many previous computations.  Indeed 
there are enough constants to be accommodated
that this verification should be done in careful writing.
In most of 
Chapter~\ref{AdvancedIntTechniques} we will see much more complicated
rewritings of integrals, and  verification will usually be much better
accomplished by checking our individual steps in integration rather than
by differentiating of our answers.

Our next example just takes this theme one step further.  Recall
that substitution in the arcsecant form had a slight complication,
which was that the $u$-variable appeared both inside and outside
the radical.  This caused a minor complication in
Example~\ref{FirstWackySecantU-SubProblem}, 
page~\pageref{FirstWackySecantU-SubProblem} for instance.
A similar problem will occur in this next example.

\bex Compute $\ds{\int\frac1{x\sqrt{81x^2-16}}\,dx}$.

\underline{Solution}: As with the previous two examples,
it is necessary to have a $1$ in the place presently
occupied by $16$, so we will factor the $16$ from the
radical.  The other algebraic difficulties will be
taken care of by the substitution. Indeed, the remaining
term under the radical must be $u^2$, and the rest of the
form will follow, with residual multiplicative constants.
\begin{align*}
\int\frac1{x\sqrt{81x^2-16}}\,\underline{dx}
&=\int\frac1{4x\sqrt{\frac{81x^2}{16}-1}}\,\underline{dx}
=\int\frac1{4\cdot\frac{4u}9\sqrt{u^2-1}}\cdot\underline{\frac49\,du}\\
\left.\begin{alignedat}{2}
&&u&=\frac{9x}4\\
&\implies&du&=\frac94\cdot \underline{dx}\\
&\iff&\frac49\,du&=\underline{dx}\\
\hline
\text{also,}&&u&=\frac{9x}4\\
&\iff&x&=\frac{4u}9\end{alignedat}
\right|&=\int\frac1{4u\sqrt{u^2-1}}\,du
        =\frac14\sec^{-1}|u|+C
        =\frac14\sec^{-1}\left|\frac{9x}4\right|+C.
\end{align*}
Note how the term $\frac{81x^2}{16}$ under the radical became
simply $u^2$, and then the term $x$ outside the 
radical became $\frac{4u}{9}$, both consistent with
$u=\frac{9x}{4}$.  Also note that a factor of
$\frac49$ in the denominator canceled with the same
factor multiplying the differential $du$.
\eex
This latest example again illustrates the points made before:
that having the $1$-term 
in the denominator is the key to the whole process,
that the rest is taken care of by the $u$-substitution which follows
and finally, that checking by differentiation is nontrivial.

Another minor complication is that the numbers we must factor might
not be perfect squares.  The process is exactly the same,
though perhaps some more care is required.

\bex Compute $\ds{\int\frac1{\sqrt{5-2x^2}}\,dx}$.

\underline{Solution}: The process is exactly the same as before.
The key is to factor the denominator to have a $1$ in the place
of the $5$:
\begin{align*}
\int\frac1{\sqrt{5-2x^2}}\,dx
&=\int\frac1{\sqrt{5}\cdot\sqrt{1-\frac{2x^2}5}}\,dx
 =\frac1{\sqrt{5}}\int\frac1{\sqrt{1-u^2}}
     \cdot\underline{\sqrt{\frac52}\cdot du}\\
\left.\begin{alignedat}{2}
&&u&=\sqrt{\frac25}\cdot x\\
&\implies&du&=\sqrt{\frac25}\cdot\underline{dx}\\
&\iff&\sqrt{\frac52}\cdot du&=\underline{dx}
\end{alignedat}\right|
&=\frac{1}{\sqrt2}\,\sin^{-1}u+C
 =\frac1{\sqrt2}\,\sin^{-1}\left(\sqrt{\frac25}\cdot x\right)+C.
\end{align*}

\eex

\subsection{Completing the Square}
In the previous subsection our first concern after identifying our
target form was to rewrite the integrand to have the number $1$
in the appropriate place in the denominator.  In this subsection
our first concern is identifying, except for a multiplicative
constant, what will be $u^2$.  We do this by completing the 
square first, and then fixing the form to have the number
$1$ where we need it, and working from there as before.
As there are differing levels of difficulty in such problems,
we will begin with one of the simplest and continue from there.
It should be noted that the completing the square step is
sometimes needed before determining that one of our three forms,
(\ref{IntGivingArcSinUR2}), 
(\ref{IntGivingArcTanUR2}) or (\ref{IntGivingArcSec|U|R2}),
can even be achieved.  If not, and we are fortunate, another
earlier method may work, though usually we should notice that
before attempting the method here.  If no earlier method will work,
there may be a method available in Chapter~\ref{AdvancedIntTechniques} 
that will solve the problem.

Recall that when completing the square, one adds and subtracts
$(b/2)^2$, where the original polynomial is $x^2+bx$, or 
more generally $x^2+bx+c$:
\begin{align*}
x^2+bx+c&=x^2+bx+\left(\frac{b}2\right)^2-\left(\frac{b}2\right)^2+c\\
&=\left(x+\frac{b}2\right)^2-\left(\frac{b}2\right)^2+c.\end{align*}
As we will see, the fact that the coefficient of $x^2$ was $1$
was key to the computation above.  If not, the leading coefficient 
will be factored from the $x^2$ and $x$ terms. Our first 
examples will not require that inititial factoring.


\bex Compute $\ds{\int\frac1{x^2+2x+2}\,dx}$.

\underline{Solution}: The hope is that we can somehow write the
denominator as $1+u^2$, perhaps multiplied by some nonzero constant,
without introducing any more variable factors.  For this one we
are unusually fortunate. Note that here ``$b$'' is 2.
\begin{align*}
\int\frac1{x^2+2x+2}\,dx&
=\int\frac1{x^2+2x+\left(\frac22\right)^2-\left(\frac22\right)^2+2}\,dx
=\int\frac1{x^2+2x+1-1+2}\,dx\\
\left.\begin{alignedat}{2}
&&u&=x+1\\
&\implies&du&=dx
\end{alignedat}\right|
&=\int\frac1{(x+1)^2+1}\,dx
 =\int\frac1{u^2+1}\,du=\tan^{-1}u+C=\tan^{-1}(x+1)+C.
\end{align*}
\eex
What made this last example particularly simple was that the
additive constant outside of the perfect square was already $1$,
which is of course key to our arctrigonometric antiderivative forms.
If not, we have to perform some division.
\bex Compute $\ds{\int\frac1{x^2+6x+17}\,dx}$.

\underline{Solution}: Here $b=3$, so we add and subtract
$\left(\frac{b}3\right)^2=9$.
\begin{align*}
\int\frac1{x^2+6x+17}\,dx
&=\int\frac1{x^2+6x+9-9+17}\,dx=\int\frac1{(x+3)^2+8}\,dx
=\frac18\int\frac1{\frac{(x+3)^2}8+1}\,dx\\
\left.\begin{alignedat}{2}
&&u&=\frac{x+3}{\sqrt8}\\
&\implies&du&=\frac1{\sqrt8}\,dx\\
&\iff&\sqrt8\,du&=dx
\end{alignedat}\right|
&=\frac18\int\frac1{u^2+1}\cdot\sqrt8\,du
 =\frac1{\sqrt8}\tan^{-1}u+C
 =\frac1{\sqrt8}\tan^{-1}\left(\frac{x+3}{\sqrt8}\right)+C.
\end{align*}
\eex
As this last example illustrates, the final form of the antiderivative
can be more complicated when completing the square is required.  While it 
would be an interesting exercise to verify the answer by differentiation,
perhaps verifying each individual step in the solution process would
be a more efficient means of verifying the answer we derived.

For simplicity we will continue with arctangent forms for the moment,
as we look at the next complication, which is that the coefficient of 
$x^2$ is not equal to 1.  In such a case we factor that leading
coefficient out of the entire polynomial, or at least out of the
$x^2$ and $x$ terms.  It is then important to perform
the addition and subtraction steps of completing the square
within the factor with the $x^2$ and $x$ terms;  the addition
and subtraction of the $(b/2)^2$ in the process must occur
simultaneously and beside eachother.  Note that such a term
has a different effect inside parentheses (or brackets) compared
to outside, so we must have the addition and subtraction
steps together in order that numerically they have no net effect.

\bex Compute $\ds{\int\frac1{5x^2-4x+9}\,dx}$.

\underline{Solution}: Our first priority is to have the 
coefficient of $x^2$ to be $1$, after which we complete the
square and finish the problem.
\begin{align*}\int\frac1{5x^2-4x+9}\,dx
&=\int\frac1{5\left[x^2-\frac45\,x+\frac95\right]}\,dx
=\int\frac1{5\left[x^2-\frac45+\left(\frac25\right)^2-\left(\frac25\right)^2
              +\frac95\right]}\,dx\\
&=\int\frac1%
{5\left[\left(x-\frac25\right)^2-\frac4{25}+\frac{45}{25}\right]}\,dx
=\frac15\int\frac1{\left[\left(x-\frac25\right)^2+\frac{41}{25}\right]}\,dx
\end{align*}
(Note how we added and subtracted $(2/5)^2$ together, both within the 
brackets.)
Now we need to manipulate this integral so there is a $1$ 
in place of the fraction $\frac{41}{25}$, which we do as before,
by factoring. Continuing,
\begin{align*}
\int\frac1{5x^2-4x+9}\,dx&=\cdots\\
&=\frac15\int\frac1{\left[\left(x-\frac25\right)^2+\frac{41}{25}\right]}\,dx\\
&=\frac15\int\frac1{\frac{41}{25}\biggl[\,
 \underbrace{\frac{25}{41}\left(x-\frac25\right)^2}_{\text{``$u^2$''}}
 +1\,\biggr]}\,dx\\
\left.\begin{alignedat}{2}
&&u&=\frac5{\sqrt{41}}\left(x-\frac25\right)\\
&\implies&du&=\frac5{\sqrt{41}}\,dx\\
&\iff&\frac{\sqrt{41}}5\,du&=dx\end{alignedat}\right|
&=\frac15\cdot\frac{25}{41}\int\frac1{u^2+1}\cdot\frac{\sqrt{41}}5\,du\\
&=\frac1{\sqrt{41}}\tan^{-1}u+C
 =\frac1{\sqrt{41}}\tan^{-1}\left[
           \frac5{\sqrt{41}}\left(x-\frac25\right)\right]+C.
\end{align*}

\eex



\newpage
\section{Hyperbolic Functions}
The algebraic and differential structure embedded in the trigonometric
functions made for some surprising, but useful derivative and integral
formulas involving the arctrigonometric functions.  As it happens,
there is another genre of functions with a similar, yet distinct
structure, that genre being the so-called {\it hyperbolic functions}.

In fact the hyperbolic functions are somewhat redundant, in that
they are based upon exponential functions and the more interesting
integration formulas can be arrived at through other methods.  However,
exploiting these functions can greatly simplify certain types of 
integration problems, as we will see.

We will begin with definitions, derivatives and some identities 
involving the hyperbolic functions.  We will then 
look at their graphs, and consider what their ``inverses'' 
would look like, and how they play out in derivative and integral
formulas.  Along the way we will compare them to their trigonometric
counterparts and see how a sign ($\pm$) here or there can make
a crucial difference in an integration problem.

\subsection{Hyperbolic Functions and Their Basic Intrinsic Structures}
We begin with the hyperbolic sine and hyperbolic cosine functions.
These can be defined geometrically, but unlike their trigonometric
counterparts, these also have  straightforward definitions
in terms of our earlier, familiar functions (though it is not 
obvious from the geometry!):
\begin{align}
\sinh x&=\frac12\left(e^x-e^{-x}\right)=\frac{e^x-e^{-x}}{2},
      \label{DefSinh}\\
\cosh x&=\frac12\left(e^x+e^{-x}\right)=\frac{e^x+e^{-x}}2.
      \label{DefCosh}
\end{align}
The others are defined in terms of these, just as with the
trigonometric functions.  Before we define 
the others, we will notice a couple of relationships which are
similar, though distinct, from what occurs with the trigonometric
functions.  The first result is algebraic:

\begin{theorem}  For all $x\in\Re$, we have
\begin{equation}\cosh^2x-\sinh^2x=1.\label{Cosh^2X-Sinh^2X=1}
\end{equation}
\end{theorem}
For the proof, we just expand the left-hand side:
\begin{align*}
\cosh^2x-\sinh^2x&=\left(\frac{e^x+e^{-x}}2\right)^2
                   -\left(\frac{e^x-e^{-x}}2\right)^2\\
                  &=\frac{e^{2x}+2e^xe^{-x}+e^{-2x}}4
                   -\frac{e^{2x}-2e^xe^{-x}+e^{-2x}}4\\
                  &=\frac{\not{e^{2x}}+2e^0+\not{e^{-2x}}-
                   \not{e^{2x}}+2e^0-\not{e^{-2x}}}4\\
                  &=\frac{2+2}4\\
                  &=1,\text{ q.e.d.}\end{align*}
Of course this is the hyperbolic analog of the basic trigonometric
identity $\cos^2\theta+\sin^2\theta=1$.  As we will see, the 
difference in the signs between the trigonometric identity
and (\ref{Cosh^2X-Sinh^2X=1}) makes for analogous, but significantly
distinct results throughout the hyperbolic development.  We next
look at the derivatives of these:
\begin{theorem} For $x\in\Re$, we have
\begin{align}
\frac{d}{dx}\sinh x&=\cosh x, \label{DerivSinhX}\\
\frac{d}{dx}\cosh x&=\sinh x. \label{DerivCoshX}
\end{align}
\end{theorem}
These are simple chain rule computations.  For the first
case, we have
$$
\frac{d}{dx}\left[\frac12\left(e^x-e^{-x}\right)\right]
=\frac12\left(e^x-e^{-x}\frac{d(-x)}{dx}\right)
=\frac12\left(e^x-e^{-x}(-1)\right)
=\frac12\left(e^x+e^{-x}\right)$$
i.e., $\frac{d}{dx}\sinh x=\cosh x$.
That $\frac{d}{dx}\cosh x=\sinh x$ is similar.  Note
how these compare to derivative formulas for $\sin x$ and
$\cos x$.
