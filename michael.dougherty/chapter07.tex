%\setcounter{page}{800}
\chapter{Advanced Integration Techniques\label{AdvancedIntTechniques}}
Before introducing the more advanced techniques, we will look at
a shortcut for the easier of the substitution-type integrals.
Advanced integration techniques then follow:
integration by parts, trigonometric integrals, trigonometric
substitution, and partial fraction decompositions.

\section{Substitution-Type Integration by Inspection}
In this section we will consider integr>als which we 
would have done earlier by substitution, but which are
simple enough that we can guess the approximate form
of the antiderivatives, and then insert any factors needed
to correct for discrepancies detected by (mentally) computing
the derivative of the approximate form and comparing it to 
the original integrand.  Some general forms will be mentioned
as formulas, but the {\it idea} is to be able to compute
integrals without resorting to writing the usual $u$-substitution
steps.

\bex Compute $\ds{\int\cos5x\,dx}$.

\underline{Solution}: We can anticipate that the 
{\bf approximate form\footnotemark} of the answer is
$\sin5x$, 
but then
$$\frac{d}{dx}\sin5x=\cos5x\cdot\frac{d}{dx}(5x)=\cos5x\cdot5=5\cos5x.$$
Since we are looking for a function whose derivative is
$\cos5x$, and we found one whose derivative is $5\cos5x$,
we see that our candidate antiderivative $\sin5x$ gives
a derivative with an extra factor of $5$, compared with
the desired outcome.  Our candidate antiderivative's
derivative is 5 times too large, so
this candidate $\sin 5x$ must be 5 times too large.  To compensate
and arrive at a function with the proper derivative, 
we multiply our candidate  $\sin5x$
by $\frac15$.  This give us a new candidate antiderivative $\frac15\sin5x$,
whose derivative is of course $\frac15\cos5x\cdot5=\cos5x$,
as desired.  Thus we have
$$\int\cos5x\,dx=\frac15\sin5x+C.$$
\eex
%%%%%  FOOTNOTE
\footnotetext{In this section, by {\it approximate form} we mean
a form which is correct except for {\bf multiplicative constants}.}
%%%%%  END FOOTNOTE

It may seem that we wrote more in the example above than with 
the usual $u$-substitution method, but what we
wrote could be performed mentally without resorting to writing
the details.

In future sections, an integral such as the above may occur as a
relatively small step in the execution of a more advanced and
more complicated method
(perhaps for computing a much more difficult integral).  
This section's purpose is to point out how such an integral can be 
quickly dispatched, to avoid it becoming a needless distraction
in the more advanced methods.

Some formulas which should be quickly verifiable by inspection
(that is, by reading and mental computation rather than 
with paper and pencil, for instance) follow:
\begin{align}
\int e^{kx}\,dx&\hphantom{-}=\hphantom{-}\frac1ke^{kx}+C,\label{IntE^KX}\\
\int \cos kx\,dx&\hphantom{-}=\hphantom{-}\frac1k \sin kx+C,\\
\int\sin kx\,dx&\hphantom{-}=-\frac1k\cos kx+C,\\
\int \sec^2kx\,dx&\hphantom{-}=\hphantom{-}\frac1k\tan kx+C,\\
\int\csc^2kx\,dx&\hphantom{-}=-\frac1k\cot kx+C,\\
\int\sec kx\tan kx\,dx&\hphantom{-}=\hphantom{-}\frac1k\sec kx+C,\\
\int\csc kx\cot kx\,dx&\hphantom{-}=-\frac1k\csc kx+C,\\
\int\frac1{ax+b}\,dx&\hphantom{-}=\hphantom{-}\frac1a\ln|ax+b|+C.
            \label{Int1/AX+B}
\end{align}

\bex The following integrals can be computed with $u$-substitution,
but also are computable by inspection:
\begin{align*}
\int\frac1{5x-9}\,dx&=\frac15\ln|5x-9|+C,\\
\int\sin5x\,dx&=-\frac15\cos5x+C,\\
\int\cos\frac{x}2\,dx&=2\sin\frac{x}2+C,\\
\int\sec^2\pi x\,dx&=\frac1{\pi}\tan\pi x+C,\\
\int\csc6x\cot6x\,dx&=-\frac16\csc6x+C.\end{align*}
\eex

While it is true that we can call upon the formulas
(\ref{IntE^KX})--(\ref{Int1/AX+B}), the more flexible
strategy is to anticipate the form of the antiderivative
and adjust accordingly.  For instance, we have the following
antiderivative form, written two ways:
\begin{align*}
\int \frac{1}u\,du&=\ln|u|+C,\\
\int\frac{f'(x)}{f(x)}\,dx&=\ln|f(x)|+C.\end{align*}
(As usual, the second form is the same as the first where 
$u=f(x)$.) So when we see an integrand which is a fraction,
the numerator being the derivative of the denominator
except for multiplicative constants, we know the 
antiderivative will be, approximately, the natural 
log of the absolute value of that denominator.

\bex Consider $\ds{\int\frac{x}{x^2+1}\,dx}$

Here we see that the derivative of the denominator is also a factor
in the integrand.
Our candidate approximate form can then be $\ln|x^2+1|=\ln(x^2+1)$.
Now we differentiate to see what we need to include to get the
correct derivative:
$$\frac{d}{dx}\ln(x^2+1)=\frac1{x^2+1}\cdot2x=2\cdot\frac{x}{x^2+1}.$$
To correct for the extra factor of 2 and thus get the correct derivative,
we insert the factor $\frac12$:
$$\frac{d}{dx}\left[\frac12\ln(x^2+1)\right]
 =\frac12\cdot\frac1{x^2+1}\cdot2x=\frac{x}{x^2+1},$$
as desired.  Thus 
$$\int\frac{x}{x^2+1}\,dx=\frac12\ln(x^2+1)+C.$$
To be sure, a quick (mental?) check by differentiation verifies the answer.
\eex

Of course there are many other forms.


\bex Consider $\ds{\int\frac1{\sqrt{5x-9}}\,dx}$.

Of course this can be rewritten
$\int(5x-9)^{-1/2}\,dx$.  Now it is crucial that a
complete substitution, $u=5x-9\implies du=5\,dx$, etc., 
shows that $du$ and $dx$ agree except for a multiplicative
constant, so we know that the integral---up to multiplicative
constants---is of form $\int u^{-1/2}\,du$, which is a power rule.

The {\it approximate} form of the antiderivative is thus
$u^{1/2}=(5x-9)^{1/2}$,  which we write in $x$ and then differentiate,
$$\frac{d}{dx}(5x-9)^{1/2}=\frac12(5x-9)^{-1/2}\cdot5,$$
which has {\bf extra} factors (compared to our original integrand)
of collectively  $\frac52$. To cancel their effects we
include a factor $\frac25$ in our actual, reported  antiderivative.
Thus
$$\int\frac1{\sqrt{5x-9}}\,dx=\frac25(5x-9)^{1/2}+C=\frac25\sqrt{5x-9}+C.$$
\eex

Note that a quick derivative computation, albeit involving a
(simple) chain rule, gives us the correct function $1/\sqrt{5x-9}$. 

\bex Consider $\ds{\int 7x\,\sin^5x^2\cos x^2\,dx}$.

For such an antiderivative, our ability to guess the form 
depends upon our exptertise with the original substitution
methods.  In all of these was a form $\int f(u)\,K\,du$,
where we could anticipate both $u$ and $f$, with $du$
accounting for remaining terms, and $K\in\Re$
which we can ignore by taking our shortcut path
described in this section.  Looking ahead, the student well-versed
in substitution will expect $u=\sin x^2$, and the integral
being of the approximate form $\int u^5\,du$.  Thus we will
have an approximate antiderivative of $u^6$ (times a constant),
i.e., the approximate form should be $\sin^6x^2$.  Now we differentiate
this and see what compensating factors must be included
to reconcile with the original integrand:\footnotemark
$$\frac{d}{dx}(\sin x^2)^6
=6(\sin x^2)^5\cdot \cos x^2\cdot 2x
=12x\,\sin^5x^2\cos x^2.$$
Of course we want $7$ in the place of the $12$ (or separately,
$2\cdot6$), so we multiply
by $\frac7{12}$ (or again, $7\cdot\frac1{6\cdot2}$).  With this we have
$$\int7x\,\sin^5x^2\cos x^2\,dx=\frac7{12}\sin^6x^2+C.$$
\eex
\footnotetext{%%%
%%% FOOTNOTE
Notice that we are assuming fluency in the chain rule as we
compute the derivative of $\sin^6x^2$, rather than writing out
every step as we did in Chapter~\ref{DerivativeChapter}.
Each student must gage personal ability to omit steps.
%%% END FOOTNOTE
}


It would be perfectly natural to forego this method of ``guess and
adjust'' in favor of the old-fashioned substitution method.
Indeed the full substitution method has some advantages (see the next
subsection).  For instance, it
is more ``constructive,'' and thus
less error-prone; one is less tempted to skip steps
while employing substitution, while one might 
attempt a mental derivation of the answer here and
thus easily be off by a factor.  
It is important that each student find the comfortable
level of brevity for himself.\footnote{%%
%%% FOOTNOTE
It is the author's experience that students in engineering
and physics programs are more interested in arriving at the
answer quickly, while mathematics and other science students prefer the
presentation of the full substitution method.  The latter
are somewhat less likely to be wrong by a multiplicative
constant, though the former tend to progress through the
topics faster.  There are, of course, spectacular exceptions,
and each group benefits from camaraderie with the other.
%%% END FOOTNOTE
}



The method used in the above examples can be summarized as follows:
\begin{enumerate}
\item Anticipate the form of the antiderivative by an {\it approximate form}
      (correct up to a multiplicative constant).
\item Differentiate this approximate form and compare to the original
      function (to be integrated);
\item If Step 1 is correct, and thus the approximate form's derivative 
      differs from the original (integrand) function by a multiplicative 
      constant, insert a compensating,
      reciprocal multiplicative constant in the approximate form to arrive
      at the actual antiderivative;
\item For verification, differentiate the answer to see if the
      original function emerges.
\end{enumerate}

\bex Compute $\ds{\int x^3\sin x^4\,dx}$.

\underline{Solution}: This is of the approximate form
$\int\sin u\,du$, with $u=x^4$. The approximate form of the solution
is thus $\cos x^4+C$ (or $-\cos x^4+C$, but these differ by 
a multiplicative constant $-1$), which has derivative
$-\sin x^4\cdot4x^3$.  We introduce a factor of $-\frac14$
to compensate for the extra factor of $-4$:
$$\int x^3\sin x^4\,dx=-\frac14\cos x^4+C,$$
which can be quickly verified by differentiation.
\eex

\bex Compute $\ds{\int x\sqrt{9-x^2}\,dx}$.

\underline{Solution}: It is advantageous to read this
integral $\int x(9-x^2)^{1/2}\,dx$, which is of
approximate form $\int u^{1/2}\,du$ (where $u=9-x^2$).  These observations,
and the approximate form $(9-x^2)^{3/2}$ of the integral, can be 
gotten by mental observation (referred to earlier as
``by inspection'').  Its derivative
is $\frac32(9-x^2)^{1/2}\cdot(-2x)$, which
has an extra factor of $-3$ (after cancellation).
Thus
$$\int x\sqrt{9-x^2}=-\frac13\left(9-x^2\right)^{3/2}+C.$$


\eex






\subsection{Limitations of the Method}
There are two very important points to be made about the
limitiations of the method.  The first point is argued
by making several related points, and the second is illustrated in
an example.

\begin{enumerate}[{\bf(I)}]
\item {\bf This method can not totally replace the earlier substitution
      method.} 
  \begin{enumerate}[(a)]
      \item The skills used in the substitution method will be
            needed for later methods.  In particular, the idea
            that the entire integral in $x$ is replaced
            by one in $u$ (for instance), including the $dx$
            and, if a definite integral, the interval of integration.
      \item If an integral is difficult enough, the more constructive
            substitution method is less error-prone than
            this ``guess and adjust'' style here.
      \item The idea of the substitution method is the same as
            this method; anticipating what to set equal to $u$ is
            equivalent to guessing the approximate form of the
            integral in $u$, and thus the approximate form of
            the antiderivative.
      \item When using numerical and other methods with definite
            integrals, a substitution can sometimes make for a
            much simpler integral to be approximated or otherwise 
            analyzed, even if the antiderivative is never computed. 
            For instance, with $u=x^2$, giving then $du=2x\,dx$, we can write
            $$\int_{-1}^2 xe^{x^4}\,dx=\frac12\int_{1}^4e^{u^2}\,du.$$
       \end{enumerate}
\item {\bf It is imperative that the derivative of the approximate form
differs from the original function to be integrated by at most a
{multiplicative constant}.}  In particular, an extra
variable function cannot be compensated for.  
\end{enumerate}
To illustrate this point, and simultaneously warn against a 
common mistake, consider 
$$\int\frac1{x^2+1}\,dx.$$
The mistake to avoid here is to take the approximate solution to be
$\ln(x^2+1)$, which we then notice has derivative
$$\frac{d}{dx}\ln(x^2+1)=\frac1{x^2+1}\cdot2x.$$
Unfortunately we cannot compensate by dividing by 
the extra factor $2x$, because\footnotemark
$$\frac{d}{dx}\left[\frac{\ln(x^2+1)}{2x}\right]=
\frac{2x\cdot\frac{d\ln(x^2+1)}{dx}-\ln(x^2+1)\cdot\frac{d(2x)}{dx}}{(2x)^2},$$
which is guaranteed (by the presence of the logarithm in
the result) to be something other than our original 
function $\frac1{x^2+1}$.  The method does not work because multiplicative
{\it functions} do not ``go along for the ride'' in derivative (or 
antiderivative) problems the way multiplicative constants do.
%%% FOOTNOTE
\footnotetext{Alternatively, a product rule computation can be used:
$$\frac{d}{dx}\left[\frac1{2x}\ln(x^2+1)\right]
=\frac1{2x}\cdot\frac{d\ln(x^2+1)}{dx}+
  \ln(x^2+1)\cdot\frac{d}{dx}\left[\frac1{2x}\right],$$
which eventually gives the original function for the first product, but 
the second part of the product rule is a complication
we cannot rid ourselves of easily.

It should be pointed out that the method of the next section
does utilize the fact that the first product above is
the desired original function, and an algorithm can be fashioned
to compensate for the presence of the second product.  The application of 
that method is not universally useful, and even 
when it is helpful 
it takes considerable work to develop the theory as well as
fluency in its application.
%%% END FOOTNOTE
}

Of course we knew from before that 
$$\int\frac1{x^2+1}\,dx=\tan^{-1}x+C,$$
so this integral is not really suitable for a substitution argument, 
but is rather a special case in and of itself.  











\newpage
\section{Integration By Parts\label{IntByPartsSection}}

While integration by substitution in its elementary form
takes advantage of the chain rule, by contrast integration by
parts exploit the product rule. The application is a bit
more complicated than with substitution, and there are 
perhaps more variations on the theme than with substitution.
Furthermore, to be truly fluent in this method requires one to be able
to see more steps ahead than with substitution, as the method
is arguably twice as long and complicated as most substitution
problems.  Still, it can be similarly mastered with practice.

\subsection{The Idea by an Example\label{IntByPartsSubsec''TheIdea''}}
Suppose that we need to find an antiderivative of the function
$f(x)=x\sec^2 x$.  It is not hard to see that normal substitution
in not going to easily yield our desired antiderivative $F$:
$$\int x\sec^2x\,dx=F(x)+C.$$  
However, a clever student might notice that $x\sec^2 x$
contains terms that could have arisen from
a product rule derivative computation:
\begin{align*}\frac{d}{dx}\left[x\tan x\right]
  &=x\cdot\frac{d\tan x}{dx}+\tan x\cdot\frac{dx}{dx}\\
  &=x\sec^2x+\tan x.\end{align*}
If we rearrange the terms above, we can summarize as follows:
$$x\sec^2x=\frac{d}{dx}\left[x\tan x\right]-\tan x.$$
In fact the line immediately
above is perhaps where the spirit of the method is
most on display:  that the given function is indeed one {\it part}
of a product rule derivative.  If we are fortunate, the other
{\it part} of the product rule formula is easier to integrate,
because the derivative term, namely $\frac{d}{dx}\left[x\tan x\right]$
is trivial to integrate, since we are just asking for the antiderivative
of the derivative of a (differentiable) function, which ultimately 
returns the function itself, and an additive constant we can absorb
in the second integral.
Indeed, if we take antiderivatives of both sides, we then get
\begin{align*}\int x\sec^2x\,dx&=\int\left[\left(\frac{d[x\tan x]}{dx}\right)
               -\tan x\right]\,dx\\
             &=x\tan x-\int\tan x\,dx\\
             &=x\tan x-\ln|\sec x|+C.\end{align*}
From such as the above emerges a method whereby we identify
our given function (here $x\sec^2x$) as a {\it part} of a product rule 
computation ($\frac{d}{dx}[x\tan x]$),
and integrate our original function by instead (trivially) integrating the
product rule derivative term (again $\frac{d}{dx}[x\tan x]$), 
and then integrating the other {\it part} ($\tan x$)
of the product rule output.  Often the other, hidden part of the
underlying product rule is easier to integrate than the original
function, and therein lies much of the success of the method.


\subsection{The Technique In Its Simpler Applications}
Recall that when we completely developed the substitution method, 
the underlying 
principle---the chain rule---was not written out in complete
derivative form, but rather in {\it differential} form.
That was partly because of the compactness of the differential
notation.  Having supposed that $F$ was an antiderivative of $f$,
we eventually settled on writing the argument below without
the first two integrals:
$$\int f(u(x))u'(x)\,dx=\int f(u(x))\cdot\frac{du(x)}{dx}\,dx
 =\int f(u)\,du=F(u)+C=F(u(x))+C.$$
At first we did write the first steps because the proof was in the chain
rule: $\frac{d}{dx}F(u(x))=F'(u(x))u'(x)=f(u(x))u'(x).$
However, we eventually opted for the differential form, though for
most it takes some practice.  We will adopt differential notation
in integration by parts as well.  For instance, recall the 
product rule can be rewritten in differential form:
\begin{equation}
d(uv)=u\,dv+v\,du.\label{ProdRuleInDifferentialsForParts}
\end{equation}
Of course this came from multiplying a derivative product rule
by, say, $dx$, assuming $u$ and $v$ are in fact functions of $x$:
$$\frac{d(uv)}{dx}=u\cdot\frac{dv}{dx}+v\cdot\frac{du}{dx}.$$
Now we rearrange (\ref{ProdRuleInDifferentialsForParts}) as follows:
\begin{equation}
u\,dv=d(uv)-v\,du.\label{Rearr.ProdRuleInDifferentialsForParts}
\end{equation}
Equation (\ref{Rearr.ProdRuleInDifferentialsForParts}) is perhaps
the best equation to visualize the principle behind the eventual
integration formula, because it is an easy step from the product rule.
The actual formula quoted in most textbooks is still two steps away.
First we integrate both sides:
\begin{equation}\int u\,dv=\int d(uv)-\int v\,du.
\label{OneStepBeforeIntByParts}\end{equation}
Next we  notice that the first integral on the right hand side of 
(\ref{OneStepBeforeIntByParts}) is simply $uv$, and so we arrive
at our final working formula for our integration by parts technique:
\begin{equation}
\boxed{\int u\,dv=uv-\int v\,du.}\label{IntByPartsFormula}\end{equation}
Most textbooks and instructors use the formula above in exactly that
form.  It is best memorized, though its derivation---particuarly
from (\ref{Rearr.ProdRuleInDifferentialsForParts})---should
not be forgotten.  

Next we look at an example of the actual application of
(\ref{IntByPartsFormula}).  In the example below, the arrangement of terms
is as one would work the problem with pencil and paper, except for
the implication arrows and the underbraces (which we explain briefly below,
and exclude from then on).

\bex Compute $\ds{\int xe^x\,dx}$.

\underline{Solution}:
\begin{alignat*}{3}
u&=x&&\qquad\qquad& dv&=e^x\,dx\\
&\Downarrow&&&&\Uparrow\\
du&=dx&&&v&=e^x
\end{alignat*}
$$\int\underbrace{x}_{u}\ \underbrace{e^x\,dx}_{dv}
=uv-\int v\,du
=(x)(e^x)-\int(e^x)\,dx
=xe^x-e^x+C.$$
\eex
 It is interesting to note
that we {\it choose} $u$ and $dv$, and then {\it compute}
$du$ and $v$, with one qualification.  That is that
$v$ is not unique;  the computation from $dv$ to $v$ is
of an antidifferentiation nature and so we really only know $v$
up to an additive constant.  In fact {\it any} $v$ so that $dv=e^x\,dx$
(we took $v=e^x\implies dv=e^x\,dx$)
will work in (\ref{IntByPartsFormula}).  Any additive constant,
while legitimate, will eventually cancel in the final computation.
For instance, if we had chosen $v=e^x+100$, we would have had
\begin{align*}
uv-\int v\,du&=x(e^x+100)-\int(e^x+100)\,dx\\
               &=xe^x+100x-e^x-100x+C\\
              &=xe^x-e^x+C,
\end{align*}
as before.  For most cases, we will just assume that the additive
constant is zero and we will use the simplest antiderivative for $v$.
(We will not continue to write the implication arrows as they 
are technical and perhaps confusing.)

Now we will revisit the example we gave in
Subsection~\ref{IntByPartsSubsec''TheIdea''}, using what will be our
basic style for this method.

\bex Compute $\ds{\int x\sec^2x\,dx}$.

\underline{Solution}:
\begin{alignat*}{3}
u&=x&&\qquad\qquad&dv&=\sec^2x\,dx\\
du&=dx&&&v&=\tan x
\end{alignat*}
$$
\int x\sec^2x\,dx =\int u\,dv
=uv-\int v\,du=x\tan x-\int\tan x\,dx=x\tan x-\ln|\sec x|+C.$$
\eex
Since this method is more complicated than substitution, there
are more complicated considerations in how to
apply it. First of course, one should attempt an
earlier, simpler method.  But if those fail, and integration
by parts is to be attempted,\footnote{%%%
%%% FOOTNOTE
Of course with practice one can see ahead whether or not integration by
parts is likely to achieve an answer for a particular integral.}
 the following guidelines for choosing $u$ and $dv$
should be considered:
\begin{enumerate}
\item $u$ and $dv$ {\bf must} account for exactly all factors of 
      the integral.
  \begin{enumerate}[1.5.]
  \item Of course, 
        $dv$ {\bf must} contain the differential term (for example, $dx$)
      as a factor,
      but can contain more terms.\end{enumerate}
\item $v=\int dv$ should be computable with relative ease.
\item $du=u'(x)\,dx$ (assuming the original integral was in $x$)
      should not be overly complicated.
\item The integral $\int v\,du$ should be simpler than the original
      integral $\int u\,dv$.\footnote{%%%%
%%% FOOTNOTE
Later, in a twist on the method,
we will see that the we do not necessarily require
$\int v\,du$ be easier than the original,  $\int u\,dv$.%
%%% END FOOTNOTE
}
\end{enumerate}
The next example illustrates the importance of the second consideration
(numbered 2.) above.

\newpage
\bex Compute $\ds{\int x^3\sin x^2\,dx}$.

\underline{Solution}: 
We  do not want to make $u=\sin x^2$, because 
then $dv=x^3\,dx$, giving $du=2x\cos x^2$ and $v=\frac14x^4$,
and our $\int v\,du$ will be $\int\frac12x^5\cos x^2\,dx$,
which is worse than our original integral.

We will instead take $u$ to be some power of $x$,
but not all of $x^3$, else the terms remaining for $dv$
would be $dv=\sin x^2\,dx$, which we cannot integrate with ordinary
methods.\footnote{%%% 
%%% FOOTNOTE
In fact, we cannot compute $\int\sin x^2\,dx$ using any kind of
substitution or parts, or any other method of this text for that
matter, and arrive at an antiderivative in simple terms of the
functions we know so far such as powers, exponentials, logarithms,
trigonometric or hyperbolic functions or their inverses.  
However, when we
study series we will find other expressions with which we can
fashion an antiderivative of $\sin x^2$.%%
%%% END FOOTNOTE
}



What we will settle on is $dv=x\sin x^2\,dx$, for its integral
is an easy substitution.  We leave the remaining terms, $x^2$,
for $u$:
\begin{alignat*}{3}
u&=x^2&&\qquad\qquad&dv&=x\sin x^2\,dx\\
du&=2x\,dx&&&v&=-\frac12\cos x^2\end{alignat*}
\begin{align*}
\int x^3\sin x^2\,dx&=\int \underbrace{x^2}_{u}
                       \cdot \underbrace{x\sin x^2\,dx}_{dv}\\
                    &=uv-\int v\,du\\
                    &=(x^2)\left(-\frac12\cos x^2\right)
                        -\int\left(-\frac12\cos x^2\right)2x\,dx\\
                    &=-\frac{x^2}{2}\cos x^2+\frac12\int \cos x^2\cdot x\,dx\\
                    &=-\frac{x^2}{2}\cos x^2+\frac12\sin x^2+C.
\end{align*}
We omitted the details of computing $v$ from $dv$, and computing the
last integral, as most students at this point can anticipate the
approximate forms of those antiderivatives, mentally compute
the derivatives of the approximate forms, and then insert constant factors
required to make the antiderivatives correct.  (This was 
the purpose of the last section.)
\eex

This last example shows how the requirement that $v=\int dv$ (up
to an additive constant) be computable helps to guide us to 
the proper choice of $u$ and $dv$.  It was lucky that the second integral
was easily computable (which would not have been the case
if the original integral were, say, $\int x^2\sin x^2\,dx$
or $\int x^4\sin x^2\,dx$), but anyhow we cannot even get
to the second integral if we cannot compute $v$.

 The next example shows a different lesson:
that it is sometimes appropriate to integrate by parts more than once
in a given problem.

\bex Compute $\ds{\int x^2\cos 3x\,dx}$.

\underline{Solution}: For reasons that will be clear later, 
we will call this integral $\I$.  Now we proceed to the 
integration by parts.
\begin{alignat*}{3}
u&=x^2&&\qquad\qquad&dv&=\cos3 x\,dx\\
du&=2x\,dx&&&v&=\frac13\sin3x\end{alignat*}
$$\I=\int \underbrace{x^2}_{u}\underbrace{\cos 3x\,dx}_{dv}
=uv-\int v\,du
=\frac13x^2\sin 3x-\int\frac23x\sin3x\,dx.
$$
While we still cannot compute this last integral directly with old methods,
it is better than the original in the sense that
our trigonometric function is multiplied by a first-degree polynomial,
where in the original the polynomial was second degree.
The work which is left more closely resembles our earliest examples
of integration by parts.  

A strict use of the language would force us to introduce
two new variables other than $u$ and $v$, but since they
have ``disappeared'' in the present form of our answer,
namely $\frac13x^2\sin3x-\frac23\int x\sin3x\,dx$, it is
not considered such bad form to ``reset'' (or ``recycle'') $u$ and $v$
for another integration by parts step, this time 
involving the integral $\int x\sin 3x\,dx$:
\begin{alignat*}{3}
u&=x&&\qquad\qquad&dv&=\sin3x\,dx\\
du&=dx&&&v&=-\frac13\cos3x\end{alignat*}
$$\int \underbrace{x}_{u}\underbrace{\sin3x\,dx}_{dv}
=uv-\int v\,du
=-\frac{x}3\cos3x+\frac13\int\cos3x\,dx
=-\frac{x}3\cos3x+\frac13\cdot\frac13\sin 3x+C_1.$$
Now we insert this last result into our original computation:
\begin{align*}
\int x^2\cos3x\,dx&=\frac{x^2}3\sin3x-\frac23\left[-\frac{x}3\cos3x
           +\frac19\sin3x+C_1\right]\\
  &=\frac{x^2}3\sin3x+\frac{2}{9}x\cos3x-\frac2{27}\sin3x+C.\end{align*}
\eex

Several lessons can be gleaned from the example above.  (1)
it is very important for proper ``bookkeeping,'' as these
problems can beget several ``subproblems,'' and the proper
placement of resulting terms is crucial to getting the correct answer;
(2) it is not unknown to use integration by parts more
than once in a problem; (3) if we can take as many 
antiderivatives of a function $f(x)$ as we like
(i.e., the antiderivative, the antiderivative of the
antiderivative, etc.), then for an integral
$\int x^nf(x)\,dx$ we can let $u=x^n$, and integration
by parts will yield a second integral with a reduction
in the power of $x$, namely 
\begin{equation}
\int x^nf(x)\,dx=x^nF(x)-\int nx^{n-1}F(x)\,dx\end{equation}
where $F'=f$. 

Other cases concern choices of $u$ where we choose a 
function whose derivative we know, but whose antiderivative
might not be standard knowledge for the average student.
\bex Compute $\ds{\int(x^2+1)\ln x\,dx}$.

\underline{Solution}: We cannot let $dv=\ln x\,dx$, since
as yet we do not know the antiderivative of $\ln x$.
(Even if we did, such a choice for $dv$ would not be advantageous,
as Example~\ref{ComputingIntLnXExample} helps to show.)
So we have little choice but to let $\ln x$ be $u$.
\begin{alignat*}{3}
u&=\ln x&&\qquad\qquad&dv&=(x^2+1)\,dx\\
du&=\frac1x\,dx&&&v&=\frac{x^3}3+x\end{alignat*}
\begin{align*}
\int(x^2+1)\ln x\,dx&=uv-\int v\,du\\
                    &=(\ln x)\left(\frac{x^3}3+x\right)
                     -\int\left(\frac{x^3}3+x\right)\frac1x\,dx\\
               &=\frac13(\ln x)(x^3+3x)-\int\left(\frac{x^2}3+1\right)\,dx\\
            &=\frac13(x^3+3x)\ln x-\frac19x^3-\frac13x+C.
                   \end{align*}
\label{IntPoly*LnExample}\eex





\subsection{A Simple Twist on the Method}
Here we show how to find antiderivatives of interesting functions
whose derivatives we already know.  
\bex Compute $\ds{\int \ln x\,dx}$.

\underline{Solution}: Here we cannot let $dv=\ln x\,dx$, for
computing $v=\int dv$ would be the same as computing the whole,
original integral.  As in Example~\ref{IntPoly*LnExample},
we also note that placing $\ln x$ in the $u$-term makes for
a simpler derivative.  Thus we write
\begin{alignat*}{3}
u&=\ln x&&\qquad\qquad&dv&=dx\\
du&=\frac1x\,dx&&&v&=x,\end{alignat*}
so that
\begin{align*}
\int \ln x\,dx&=uv-\int v\,du\\
              &=(\ln x)(x)-\int(x)\left(\frac1x\,dx\right)\\
              &=x\ln x-\int 1\,dx\\
              &=x\ln x-x+C.\end{align*}
\label{ComputingIntLnXExample}\eex
In fact the same type of computation will be used for 
finding antiderivatives of arctrigonometric functions.

\bex Compute $\ds{\int\sin^{-1}x\,dx}$.

\underline{Solution}: Again we have no choice but to let $u=\sin^{-1}x$,
$dv=dx$.
\begin{alignat*}{3}
u&=\sin^{-1}x&&\qquad\qquad&dv&=dx\\
du&=\frac1{\sqrt{1-x^2}}\,dx&&&v&=x\end{alignat*}
For brevity, we will begin to label the desired integral $\I$,
so here $\I=\int\sin^{-1}x\,dx$.
(The second interval is computed ``by inspection.'')
\begin{align*}
\I=uv-\int v\,du&=x\sin^{-1}x-\int\frac{x}{\sqrt{1-x^2}}\,dx\\
                &=x\sin^{-1}x+\left(1-x^2\right)^{1/2}+C.\end{align*}

\eex


\newpage
\subsection{An Indirect Method}
The following method, which we describe in the end, is
useful in surprisingly many settings.
\bex Compute $\ds{\int e^{2x}\cos3x\,dx=\I}$.

\underline{Solution}: Here we again name the desired integral
$\I$ for brevity in later steps.

It should be obvious (especially after a
few attempts) that simple substitution methods will not work.
So we attempt an integration by parts.
\begin{description}
\item[Step 1.]  We will let the trigonometric function be part of $dv$:
\begin{alignat*}{3}
u&=e^{2x}&&\qquad\qquad&dv&=\cos3x\,dx\\
du&=2e^{2x}\,dx&&&v&=\frac13\sin3x
\end{alignat*}
So far, after some rearrangement and simplifying, we have
\begin{align}\I&=uv-\int v\,du\notag\\
&=\frac13e^{2x}\sin3x-\frac23
\underbrace{\int e^{2x}\sin 3x\,dx}_{\II}.\label{(I)Step1}\end{align}
This does not seem any easier than the first integral, so 
perhaps we might continue, but this time let the trigonometric 
function
be $u$ and the exponential (along with $dx$) be contained in $dv$.
\item[Step 2.] Compute $\II=\int e^{2x}\sin3x\,dx$ 
 in light of the comments at the end of the first step.
\begin{alignat*}{3}
u&=\sin3x&&\qquad\qquad&dv&=e^{2x}\,dx\\
du&=3\cos3x\,dx&&&v&=\frac12e^{2x}\end{alignat*}
$$\II=uv-\int v\,du=\frac12e^{2x}\sin3x-\frac32\int e^{2x}\cos3x\,dx.$$
Combining this with the conclusion (\ref{(I)Step1}) of Step 1 gives us:
\begin{align*}
\I&=\frac13e^{2x}\sin3x-\frac23\left[
        \frac12e^{2x}\sin3x-\frac32\int e^{2x}\cos3x\,dx\right]\\
  &=\frac13e^{2x}\sin3x-\frac13e^{2x}\sin3x+\int e^{2x}\cos3x\,dx\\
  &=\int e^{2x}\cos3x\,dx.
\end{align*}
\end{description}
Unfortunately that puts us right back where we started.
However, a minor change in our effort above will eventually lead us to 
the solution.  While keeping Step 1,
our next step towards a solution is to replace Step 2
by the same strategy as used in Step 1, namely that we 
use the exponential function for $u$ and the trigonometric with  $dv$.
\begin{description}
\item[Step 2---Second Attempt] Again we compute $\II=\int e^{2x}\sin3x\,dx$:
\begin{alignat*}{3}
u&=e^{2x}&&\qquad\qquad&dv&=\sin3x\,dx\\
du&=2e^{2x}\,dx&&&v&=-\frac13\cos3x\end{alignat*}
\begin{align}
\II&=uv-\int v\,du\notag\\
   &=-\frac13e^{2x}\cos3x+\frac23\int e^{2x}\cos3x\,dx.
   \label{(II)Step2'}\end{align}
It may seem that (\ref{(II)Step2'}) is also a dead end,
since it contains the original integral. But this
attempt is different. In fact,
when we combine (\ref{(II)Step2'}) with (\ref{(I)Step1}) we get
\begin{align*}
\I&=\frac13e^{2x}\sin3x-\frac23\II\\
  &=\frac13e^{2x}\sin3x-\frac23\left[-\frac13e^{2x}\cos3x
                                 +\frac23\int e^{2x}\cos3x\,dx\right]\\
  &=\frac13e^{2x}\sin3x+\frac29e^{2x}\cos3x-\frac49
\underbrace{\int e^{2x}\cos3x\,dx}_{\I},
\end{align*}
which we can summarize by the following equation:
\begin{equation}
\I=\frac13e^{2x}\sin3x+\frac29e^{2x}\cos3x-\frac49\I.
\label{EquationToBeSolvedToFindI}
\end{equation}
Now we are ready to derive $\I$, not by another calculus computation,
but in fact by simple algebra.
\item[Step 3.] Solve (\ref{EquationToBeSolvedToFindI}) for $\I$.
First we add $\frac49\I$ to both sides.
$$\frac{13}{9}\I=\frac13e^{2x}\sin3x+\frac29e^{2x}\cos3x+C_1.$$
Here we add $C_1$ because in fact each $\I$ in 
(\ref{EquationToBeSolvedToFindI}) represents
all antiderivatives which of course differ from eachother by
additive constants.  Now (\ref{EquationToBeSolvedToFindI})
makes sense because of the fact that there are additive constants on both
sides of that equation (though on the right side they are 
mulitplied by $-\frac49$, but that still yields additive constants).\footnote{%
%%% FOOTNOTE
In fact, two simlutaneous appearances of $\I$ do not have to have
the same additive constants, so $\I-\I=C_2$, not zero.%
%%%% END FOOTNOTE
}  
Solving for $\I$ we now have
\begin{align}
\I&=\frac9{13}\left[\frac13e^{2x}\sin3x+\frac29e^{2x}\cos3x+C_1\right]\notag\\
&=\frac3{13}e^{2x}\sin3x+\frac2{13}e^{2x}\cos3x+C,
\end{align}
where $C=\frac9{13}C_1$.
\end{description}
\label{ExampleIntEx(2x)Sin3x;SolveFor(I)}\eex

What is important to understand about the example above is that sometimes,
though we cannot perhaps directly compute a particular integral, it may
happen that an indirect method gives us the answer.  Here we found an
equation, namely (\ref{EquationToBeSolvedToFindI}), which our desired
integral satisfies, and for which it could be solved algebraically.
We must be open to the possibility of finding a desired quantity
by indirect methods, as well as direct computations.

It should be pointed out that we could have computed the integral
in Example~\ref{ExampleIntEx(2x)Sin3x;SolveFor(I)} by instead
letting $u$ be the trigonometric function, and $dv=e^{2x}\,dx$
in {\it both} steps.  In fact it is usually best to pick similar
choices for $u$ and $dv$ when an integration by parts will take
more than one step.  (Recall the discussion for $\int x^nf(x)\,dx$.)

The method of Example~\ref{ExampleIntEx(2x)Sin3x;SolveFor(I)},
namely solving for $\I$ after an integration by parts step, 
is available perhaps more often than one would think.

\bex Compute $\ds{\int \sin^2x\,dx=\I}$.

\underline{Solution}:
\begin{alignat*}{3}
u&=\sin x&&\qquad\qquad&dv&=\sin x\,dx\\
du&=\cos x\,dx&&&v&=-\cos x
\end{alignat*}
$$\I=\int uv-\int v\,du=-\sin x\cos x+\int\cos^2x\,dx.$$
We could perform the same integration by parts with the second
integral, but instead we will use the fact that $\cos^2x=1-\sin^2x$:
\begin{align*}\I&=-\sin x\cos x+\int(1-\sin^2x)\,dx\\
                &=-\sin x\cos x+x-\int\sin^2x\,dx\\
                &=x-\sin x\cos x-\I.\end{align*}
Here we see that we can add $\I$ to both sides, divide by $2$ and get
$$\I=\frac12(x-\sin x\cos x)+C.\footnotemark$$
\footnotetext{%%
%%%% FOOTNOTE
The intermediate step would be 
  $2\I=x-\sin x\cos x+C_1$.  See the discussion
for Example~\ref{ExampleIntEx(2x)Sin3x;SolveFor(I)},
page~\pageref{ExampleIntEx(2x)Sin3x;SolveFor(I)}.

In fact many textbooks do not bother writing the $C_1$ term, preferring
to remind the student at the end that an indefinite integral problem 
necessitates a ``$+C.$''%
%%% END FOOTNOTE
}\label{FirstPartsIntegrationOfSinSquaredX}
\eex

The example above can be computed directly if we use the trigonometric fact
that $\sin^2x=\frac12(1-\cos 2x)$, giving
\begin{align*}\int\sin^2x\,dx&=\int\frac12(1-\cos2x)\,dx\\
        &=\frac12x-\frac14\sin2x+C\\
        &=\frac12x-\frac14\cdot2\sin x\cos x+C,\end{align*}
which is the same as before.  
For the last step we used the (double-angle)
trigonometric identity $\sin2x=2\sin x\cos x$.
In Section~\ref{TrigIntsSection} we will opt for
this alternative method, and indeed will make quite an
effort to exploit the algebraic properties of the trignometric
functions wherever possible, but some integrals there will
also {\it require} integration by parts.






\subsection{Miscellaneous Considerations}
First we look at a definite integral arising from integration by
parts.  It should be pointed out that the general formula
will look like the following:\footnotemark
\begin{equation}
\int_{x=a}^{x=b}u\,dv=\left.\vphantom{\frac11}uv\right|_{x=a}^{x=b}
-\int_{x=a}^{x=b}v\,du.\label{PartsForDefIntsEq}\end{equation}
\footnotetext{%
%%% FOOTNOTE
Some texts leave out the ``$x=$'' parts, assuming they are 
understood, but we will continue to use the convention that,
unless otherwise stated, the ``limits of integration''
should match the differential's variable.  Another popular way
to write (\ref{PartsForDefIntsEq}) avoids the issue:
$$\int_a^b u(x)v'(x)\,dx=\left.\vphantom{\frac11}u(x)v(x)\right|_a^b
   -\int_a^bv(x)u'(x)\,dx.$$
%%% END FOOTNOTE
}
\bex Compute $\ds{\int_{-\pi}^{\pi}x\sin x\,dx}$.

\underline{Solution}:
(Note how the negative signs cancel in the parts formula step.)
\begin{alignat*}{3}
u&=x&&\qquad\qquad&dv&=\sin x\,dx\\
du&=dx&&&v&=-\cos x\end{alignat*}
\begin{align*}
\int_{-\pi}^{\pi}x\sin x\,dx
  &=\left.\vphantom{X_X^X}(-x\cos x)\right|_{-\pi}^{\pi}
    +\int_{-\pi}^{\pi}\cos x\,dx\\
  &=[-\pi\cos\pi]-[-(-\pi)\cos(-\pi)]+\left.
   \vphantom{\frac11}\sin x\right|_{-\pi}^{\pi}\\
  &=(-\pi)(-1)-(\pi)(-1)+\sin\pi-\sin(-\pi)\\
  &=\pi+\pi+0-0\\
  &=2\pi.\end{align*}
\eex

In the example above,
We could also have  noticed that $\int_{-\pi}^{\pi}\cos x\,dx$
is zero because we are integrating over a whole period 
$[-\pi,\pi]$ of $\cos x$, and both $\sin x$ and $\cos x$
have definite integral zero over any full period.

It is typical to compute that part $\left.\vphantom{\frac22}
  u(x)v(x)\right|_{z}^b$ separately, but one could instead and
separately compute the entire antiderivative, and then 
evaluate at the two limits and take the difference:
$$\int_{-\pi}^{\pi} x\sin x\,dx
=\left.\vphantom{\frac11}\left(-x\cos x+\sin x\right)\right|_{-\pi}^{\pi}.$$
Which method to use is a matter of bookkeeping preferences,
and perhaps whether or not part of the 
right-hand side of (\ref{PartsForDefIntsEq}) is particularly simple.


The next example gives us several options along the way, though
in each the original choices of $u$ and $dv$ are the same.
\bex Compute $\ds{\int x\tan^{-1}x\,dx=\I}$.

\underline{Solution}: Again we have little choice on our selection of
$u$ and $dv$.
\begin{alignat*}{3}
u&=\tan^{-1}x&&\qquad\qquad&dv&=x\,dx\\
du&=\frac1{x^2+1}\,dx&&&v&=\frac12x^2
\end{alignat*}
\begin{align*}
\I&=uv-\int v\,du\\
  &=\frac12x^2\tan^{-1}x-\frac12\int\frac{x^2}{x^2+1}\,dx.\end{align*}
Now this last integral can be found by rewriting
the integrand using either polynomial long division, or 
by a little cleverness:
$$\frac{x^2}{x^2+1}=\frac{x^2+1-1}{x^2+1}=\frac{x^2+1}{x^2+1}
                       -\frac1{x^2+1}=1-\frac1{x^2+1}.$$
Thus
\begin{align*}
\I&=\frac12x^2\tan^{-1}x-\frac12\int\left(1-\frac1{x^2+1}\right)\,dx\\
 &=\frac12x^2\tan^{-1}x-\frac12x+\frac12\tan^{-1}x+C.\end{align*}

Though our choice of $u$ and $dv$ was limited, 
our choice of $v$ was not as limited.  Recall that we could have
chosen any $v=\frac12x^2+C_1$.  In this particular integral, 
we could have saved ourselves some effort if we had chosen
$v$ more strategically:
\begin{alignat*}{3}
u&=\tan^{-1}x&&\qquad\qquad&dv&=x\,dx\\
du&=\frac1{x^2+1}\,dx&&&v&=\frac12\left(x^2+1\right)\end{alignat*}
This gives
\begin{align*}
\I&=uv-\int v\,du\\
  &=\frac12\left(x^2+1\right)\tan^{-1}x-\int\frac{\frac12(x^2+1)}{x^2+1}\,dx\\
  &=\frac12\left(x^2+1\right)\tan^{-1}x-\int\frac12\,dx\\
  &=\frac12\left(x^2+1\right)\tan^{-1}x-\frac12x+C,
\end{align*}
as before (though rearranged).
\eex
Though rare, and not crucial, strategically adding a particular constant
to the natural choice for $v$ can on occasion make for easier computations.




\newpage
\begin{center}
\underline{\Large{\bf Exercises}}\end{center}

\begin{multicols}{2}

Compute the following integrals.
\begin{enumerate}
\item $\ds{\int x\sin x\,dx}$
\item $\ds{\int x\cos x\,dx}$
\item $\ds{\int x\sec x\tan x\,dx}$
\item $\ds{\int x\sec^2x\,dx}$
\item $\ds{\int x\ln x\,dx}$
\item $\ds{\int x\tan^{-1}x\,dx}$
\item $\ds{\int x\sec^{-1}x\,dx}$, $x>1$.
\item $\ds{\int x\sec^{-1}x\,dx}$, $x<1$.
\item $\ds{\int x\sqrt{1-x}\,dx}$.
\item $\ds{\int\frac{x}{\sqrt{1-x}}\,dx}$
\item $\ds{\int xe^x\,dx}$
\item $\ds{\int x\sin5x\,dx}$
\item $\ds{\int x e^{x/2}\,dx}$
\item $\ds{\int x^3e^{x^2}\,dx}$
\item $\ds{\int x^5\sin x^3\,dx}$
\item $\ds{\int x^2e^{3x}\,dx}$
\item $\ds{\int \ln x\,dx}$
\item $\ds{\int \tan^{-1}x\,dx}$
\item $\ds{\int\sin^{-1}x\,dx}$
\item $\ds{\int x\sqrt{1-x^2}\ \sin^{-1}x\,dx}$
\item $\ds{\int\sin^2x\,dx}$
\item $\ds{\int\cos^25x\,dx}$
\item $\ds{\int e^{5x}\cos2x\,dx}$
\end{enumerate}
\end{multicols}
\newpage
\section{Trigonometric Integrals\label{TrigIntsSection}}
We have already looked at two basic types of trigonometric
integrals:  those arising from the derivatives of the
trigonometric functions (Subsection~\ref{FirstTrigRulesSubsection}, 
page~\pageref{FirstTrigRulesSubsection}), and those of the elementary
substitution types in Section~\ref{SecondTrigRules}.
In this section we are mainly interested in
computing integrals
where the integrands are combinations of powers of
trigonometric functions.  In such cases, the 
angles of each trigonometric function appearing are all the same.
Another important topic considered here is how to deal
with trigonometric combinations where the angles differ,
and we will examine how to deal with several of those cases.

In the first examples where the angles agree, 
we rearrange the terms in the integrand  and use 
the three basic trigonometric identities to write the
entire integral as function of one trigonometric function,
and its differential as the final factor.  A substitution
step then leads to one or more power rules.  Unfortunately
this  only leads to a solution  if the combinations of powers 
are of a few simple forms.  Still, these combinations occur
often enough to warrant study.

After we look at those simplest forms, we look at 
other combinations of powers where the angles agree.  
Techniques include other
algebraic manipulations, as well as integration by parts.

In the final forms, where the angles do not agree, we look
at several trigonometric identities which help us to 
rewrite the integrals in simpler forms.

\subsection{Simplest Examples}

These first three examples illustrate an approach we
develop in Subsections~\ref{OddSineOrCosine},
\ref{EvenSecant/OddTangent} and \ref{EvenCosecant/OddCotangent}.


\bex Compute $\ds{\int\tan^2x\,dx}$.

\underline{Solution}: 
$\ds{\int\tan^2x\,dx=\int(\sec^2x-1)\,dx=\tan x-x+C.}$
\eex
The example above used the facts that 
$\tan^2x=\sec^2x-1$, and that we know
the antiderivative of $\sec^2x$ (where we
might not have known the antiderivative of 
$\tan^2x$ immediately).  The integral above does
not in itself contain a general method.  Indeed there
is no general method, but there are ways to 
rewrite many trigonometric integrals to make their
computations more elementary.
\bex 
Compute  $\ds{\int\sin^6x\,\cos^5x\,dx}$.

\underline{Solution}: Here we will use the fact that 
$\cos^2x=1-\sin^2x$, and so $\cos^{2k}x=\left(1-\sin^2x\right)^k$.
Eventually  we will take $u=\sin x$, implying $du=\cos x\,dx$:
\begin{align*}
\int\sin^6x\,\cos^5x\,dx&=\int\sin^6x\,\cos^4x\,\cos x\,dx\\
                     &=\int\sin^6x\left(1-\sin^2x\right)^2\cos x\,dx\\
                  &=\int u^6\left(1-u^2\right)^2\,du
                     =\int u^6\left(1-2u^2+u^4\right)\,du\\
                  &=\int\left[u^6-2u^8+u^{10}\right]\,du\\
                  &=\frac{u^7}7-\frac{2u^9}9+\frac{u^{11}}{11}+C\\
                  &=\frac17\sin^7x-\frac29\sin^9x+\frac1{11}\sin^{11}x+C.
\end{align*}
\eex

\bex Compute $\ds{\int\sec^5x\,\tan^3x\,dx}$.

\underline{Solution}: Here we will borrow a factor of secant,
and another of tangent, to form the functional part of $du$,
where $u=\sec x$:

\begin{align*}
\int\sec^5x\,\tan^3x\,dx&=\int\sec^4x\,\tan^2x\,\sec x\tan x\,dx\\
&=\int\sec^4x\left(\sec^2x-1\right)\sec x\tan x\,dx\\
&=\int u^4\left(u^2-1\right)\,du\\
&=\int\left[u^6-u^4\right]\,du\\
&=\frac{u^7}7-\frac{u^5}5+C\\
&=\frac17\sec^7x-\frac15\sec^5x+C.
\end{align*}
\eex

Now we look at these three specific techniques more closely
and generalize them.

\subsection{Odd Powers of Sine or Cosine\label{OddSineOrCosine}}
Here we are interested in the cases of integrals
\begin{equation}\int\sin^m\theta\,\cos^n\theta\,d\theta.\label{SineCosForm1}
\end{equation}
where either $m$ or $n$ is odd.  Suppose, for example, that
$m$ is odd, so that we can write $m=2k+1$ for some integer $k$.
Then we rewrite the form (\ref{SineCosForm1}) as 
$$\int\sin^m\theta\,\cos^{2k+1}\theta\,d\theta
  =\int\sin^m\theta\,\cos^{2k}\theta\,\cos\theta\,d\theta.$$
The $\cos\theta$ term which we ``peeled away'' becomes the functional
part of the $du$, where $u=\sin\theta$ (so $du=\cos\theta\,d\theta$).
We then write the rest of the integral in terms of $u=\sin\theta$.
To do so we use
\begin{alignat*}{2}
&&\sin^2\theta+\cos^2\theta&=0\\
&\iff&\qquad\cos^2\theta&=1-\sin^2\theta\\
&\implies&\cos^{2k}\theta&=\left(1-\sin^2\theta\right)^k.\end{alignat*}
Using this fact in the integral above, and setting $u=\sin\theta$, we get
\begin{align*}
\int\sin^m\theta\,\cos^{2k+1}\theta\,d\theta
 & =\int\sin^m\theta\,\cos^{2k}\theta\,\cos\theta\,d\theta\\
 &=\int\sin^m\theta\left(1-\sin^2\theta\right)^k\cos\theta\,d\theta\\
 &=\int u^m\left(1-u^2\right)^k\,du.\end{align*}
This yields a polynomial integrand, which
we may then wish to expand before computing (with a sequence of power
rules).

Simliarly, if there is an odd power of the sine, we can
use the fact that $\sin^2\theta=1-\cos^2\theta$,
and eventually using $u=\cos\theta$, to 
rewrite such an integral
\begin{align*}
\int\sin^{2k+1}\theta\,\cos\theta\,d\theta&=\int\sin^{2k}\theta\,\cos\theta
                       \sin\theta\,d\theta\\
  &=\int\left(1-\cos^2\theta\right)^k\cos^n\theta\,\sin\theta\,d\theta\\
  &=\int\left(1-u^2\right)u^n(-du)\\
  &=-\int\left(1-u^2\right)u^n\,du.
\end{align*}

In both of these it was crucial that we had an odd number of
factors of either the sine or cosine, since ``peeling off''
one factor then leaves an even number, which can be
easily written in terms of the other trigonometric function.
The peeled off factor is then the functional part of the differential
after substitution.

Note that while
any even power of a sine or cosine function can be written
entirely in terms of  the other, this is not the
case with odd powers.\footnote{%%
%%% FOOTNOTE
Consider the trigonometric identity $\sin^2\theta+\cos^2\theta=1$.
When solved for either the sine or cosine function, we
get one of the following:
\begin{align*}
\sin\theta&=\pm\sqrt{1-\cos^2\theta},\\
\cos\theta&=\pm\sqrt{1-\sin^2\theta}.\end{align*}
We see the ambiguity in the $\pm$, and the introduction of a radical
which itself can very much complicate an integral.
However, when we raise these to 
even powers the radicals {\it and} the $\pm$ both 
disappear, and we are left with sums of nonnegative, integer powers.
%%% END FOOTNOTE
}
This technique works because
removing a factor from an odd power of sine or cosine,
both provides the functional part of $du$ and leaves an even power,
which we write in terms of the other function which is then $u$
in the substitution.

\bex Compute $\ds{\int\sin^5x\,\cos^4x\,dx}$.

\underline{Solution}: Here we see an odd number of sine factors,
as so we peel one away to be part of the differential term,
and write the entire integral in terms of the cosine:
\begin{align*}
\int\sin^5x\,\cos^4x\,dx&=\int\sin^4x\,\cos^4x\,\sin x\,dx\\
                      &=\int\left(\sin^2x\right)^2\cos^4x\,\sin x\,dx\\
                      &=\int\left(1-\cos^2x\right)^2\cos^4x\,\sin x\,dx.
\end{align*}
(In most future computations we will skip the second line above.)
Now we take
\begin{alignat*}{2}
&&u&=\cos x\\
&\implies& du&=-\sin x\,dx\\
&\iff&    -du&=\sin x\,dx.
\end{alignat*}
With the substitution we will have a polynomial to integrate.
To summarize and finish the problem, we have:
\begin{align*}
\int\sin^5x\,\cos^4x\,dx&=\int\left(1-\cos^2x\right)^2\cos^4x\,\sin x\,dx\\
      &=\int\left(1-u^2\right)^2u^4(-du)\\
      &=-\int\left(1-2u^2+u^4\right)u^4\,du\\
      &=-\int\left(u^4-2u^6+u^8\right)\,du\\
      &=-\frac15\,u^5+\frac27\,u^7-\frac19\,u^9+C\\
      &=-\frac15\cos^5x+\frac27\cos^7x-\frac19\cos^9x+C.
\end{align*}
\eex

It should be clear that one cannot easily differentiate the final
answer and immediately recognize the original integrand.  This is because
some trigonometric identities were used to get an integrand form
which was computable using these methods.  Indeed, it is best
to check the validity of the steps from the beginning, rather
than to differentiate a tentative answer.  However, it is an
interesting exercise---left to the interested reader---in 
trigonometric identities to 
perform the differentiation, and then validate that the
answer there is the original integrand.


It is not necessary that the angle is always $x$.
However, for this technique we do require the angles inside
the trigonometric functions to always match, and for the
approximate differential of the variable of substitution to be present.

\bex Compute $\ds{\int\sin^45x\,\cos^35x\,dx}$.

\underline{Solution}: Here there is an odd number of cosine terms, and
we act accordingly.
\begin{align*}
\int\sin^45x\,\cos^35x\,dx
  &=\int\sin^45x\,\cos^25x\,\cos5x\,dx\\
  &=\int\sin^45x\left(1-\sin^25x\right)\cos5x\,dx.\end{align*}
Here we have
\begin{alignat*}{2}
&&u&=\sin5x\\
&\implies& du&=5\cos5x\,dx\\
&\iff&    \frac15\,du&=\cos5x\,dx.
\end{alignat*}
Now we begin again, incorporating this new information into 
our computation:
\begin{align*}
\int\sin^45x\,\cos^35x\,dx
&=\int\sin^45x\left(1-\sin^25x\right)\cos5x\,dx\\
&=\int u^4\left(1-u^2\right)\cdot\frac15\,du\\
&=\frac15\int\left(u^4-u^6\right)\,du\\
&=\frac15\cdot\frac15\,u^5-\frac15\cdot\frac17\,u^7+C\\
&=\frac1{25}\sin^55x-\frac1{35}\sin^75x+C.
\end{align*}
\eex

The technique works even if only one of the trigonometric
functions sine or cosine appears, as long as it is to an odd
power.

\bex Compute $\ds{\int\sin^37x\,dx}$.

\underline{Solution}:  Here we can still peel off a sine factor
to be the functional part of our differential, and then
write the remaining factors in terms of the cosine.

\begin{align*}
\int\sin^37x\,dx&=\int\sin^27x\,\sin7x\,dx\\
                &=\int\left(1-\cos^27x\right)\sin7x\,dx.
\end{align*}
Using the substitution $u=\cos7x$, so $du=-7\sin7x\,dx$,
implying $-\frac17\,du=\sin7x\,dx$, we get
\begin{align*}
\int\sin^37x\,dx&=
\int\left(1-\cos^27x\right)\sin7x\,dx\\
              &=\int\left(1-u^2\right)\cdot\frac{-1}7\,du\\
              &=-\frac17\left[u-\frac13\,u^3\right]+C\\
              &=-\frac17\cos7x+\frac1{21}\cos^37x+C.
\end{align*}
\eex

Furthermore, not all the powers need to be positive integer powers,
as long as one is odd.

\bex Compute $\ds{\int\frac{\cos^7x}{\sqrt{\sin x}}\,dx}$.

\underline{Solution}:  Here we have an odd number of cosine terms,
so we will peel one off to be the functional part of our differential.
That is, we will have $u=\sin x$, so $du=\cos x\,dx$. Thus
\begin{align*}
\int\frac{\cos^7x}{\sqrt{\sin x}}\,dx
&=\int\frac{\cos^6x}{\sqrt{\sin x}}\,\cos x\,dx\\
&=\int\frac{\left(1-\sin^2x\right)^3}{\sqrt{\sin x}}\,\cos x\,dx\\
&=\int\frac{\left(1-u^2\right)^3}{\sqrt{u}}\,du\\
&=\int\frac{1-3u^2+3u^4-u^6}{u^{1/2}}\,du\\
&=\int\left[u^{-1/2}-3u^{3/2}+3u^{7/2}-u^{11/2}\right]\,du\\
&=2u^{1/2}-3\cdot\frac25\,u^{5/2}+3\cdot\frac29\,u^{9/2}-\frac2{13}\,u^{13/2}
          +C\\
&=2u^{1/2}\left[1-\frac35u^2+\frac13u^4-\frac1{13}u^6\right]+C\\
&=2\sqrt{\sin x}\left[1-\frac35\sin^2x+\frac13\sin^4x-\frac1{13}\sin^6x\right]
   +C.
\end{align*}
\eex

It is possible that both powers are odd, and either
function can be peeled off, and the integral written in terms of the
other.  However, if one of these odd powers is greater than the other,
it is more efficient to peel off a factor from the lower power, as the
next example demonstrates.

\bex Compute $\ds{\int\sin^3x\,\cos^7x\,dx}$.

\underline{Solution}:  We will consider both methods for computing
this antiderivative.  First we peel off a sine to be part of
the differential, and let $u=\cos x$.
\begin{align*}
\int\sin^3x\,\cos^7x\,dx&=\int\sin^2x\,\cos^7x\,\sin x\,dx\\
                      &=\int\left(1-\cos^2x\right)\cos^7x\,\sin x\,dx\\
                      &=\int\left(1-u^2\right)u^7(-\,du)\\
                      &=-\int\left(u^7-u^9\right)\,du\\
                      &=-\frac18\,u^8+\frac1{10}\,u^{10}+C\\
                      &=-\frac18\cos^8x+\frac1{10}\cos^{10}x+C.
\end{align*}
Next we instead peel off a cosine factor, and let $w=\sin x$.
\begin{align*}
\int\sin^3x\,\cos^7x\,dx&=\int\sin^3x\,\cos^6x\,\cos x\,dx\\
        &=\int\sin^3x\left(1-\sin^2x\right)^3\cos x\,dx\\
        &=\int w^3\left(1-w^2\right)^3\,dw\\
        &=\int w^3\left(1-3w^2+3w^4-w^6\right)\,dw\\
        &=\int\left(w^3-3w^5+3w^7-w^9\right)\,dw\\
        &=\frac14\,w^4-\frac36\,w^6+\frac38\,w^8-\frac1{10}\,w^{10}+C\\
        &=\frac14\sin^4x-\frac12\sin^6x+\frac38\sin^8x-\frac1{10}\sin^{10}x+C.
\end{align*}
\eex

As we see in the above example, there can be different valid choices for
some integrals.  The answers may look very different, but 
that is a reflection of the wealth of trigonometric identities 
available.  In fact, the antiderivatives, excluding the arbitrary
constants, need not be equal, but the difference should be 
accounted for in the constants.\footnote{%%%
%%% FOOTNOTE
Recall the integral $\int2\sin x\cos x\,dx$, for which one
can let either $u=\sin x$ or $u=\cos x$, yielding
\begin{align*}
\int2\sin x\cos x\,dx&=\sin^2x+C_1,\qquad\text{or}\\
\int2\sin x\cos x\,dx&=-\cos^2x+C_2.\end{align*}
Since these differ by a constant, specifically
$\sin^2x=-\cos^2x+1$, both are valid.  But clearly
$\sin^2x\ne\cos^2x$.
%%% END FOOTNOTE
}




\subsection{Even Powers of Secant or Odd Powers of Tangent
\label{EvenSecant/OddTangent}}
This technique of peeling off some factors of a trigonometric
function to be part of the $du$ (after substitution) has
two workable versions for integrals of the type
\begin{equation}
\int\sec^m\,\theta\tan^n\theta\,d\theta.\label{SecantTangentForm1}
\end{equation}
These rely upon the following facts from trigonometry and calculus:
\begin{alignat*}{3}
\tan^2\theta+1&=\sec^2\theta,&\qquad\qquad
            \frac{d}{d\theta}\tan\theta&=\sec^2\theta,\\
\sec^2\theta-1&=\tan^2\theta,&\qquad\frac{d}{d\theta}
        \sec\theta&=\sec\theta\tan\theta.\end{alignat*}

The techniques we will employ are as follow:
\begin{enumerate}
\item If an integral of the form (\ref{SecantTangentForm1})
contains an odd power of tangent, we peel off a factor
$\sec\theta\tan\theta$ to be the functional part of the differential.
This leaves an even power of tangent, which can be written as
a power of $\left(\sec^2\theta-1\right)$.
\item If an integral of the form (\ref{SecantTangentForm1})
contains an even power of secant, we peel off a factor
$\sec^2\theta$ to be the functional part of the differential.
The remaining even power of secant is then written as 
a power of $\left(\tan^2\theta+1\right)$.
\end{enumerate}

\bex Compute $\ds{\int\sec^6x\,\tan^8x\,dx}$.

\underline{Solution}  Here we have an even number of secant factors,
and so we can peel off two.  Eventually we will let $u=\tan x$,
implying $du=\sec^2x\,dx$.
\begin{align*}
\int\sec^6x\,\tan^8x\,dx
 &=\int\sec^4x\,\tan^8x\,\sec^2x\,dx\\
 &=\int\left(\tan^2x+1\right)^2\tan^8x\,\sec^2x\,dx\\
 &=\int\left(u^2+1\right)^2u^8\,du\\
 &=\int\left(u^4+2u^2+1\right)u^8\,du\\
 &=\int\left(u^{12}+2u^{10}+u^8\right)\,du\\
 &=\frac1{13}\,u^{13}+\frac2{11}\,u^{11}+\frac19\,u^9+C\\
 &=\frac1{13}\tan^{13}x+\frac2{11}\tan^{11}x+\frac19\tan^9x+C.
\end{align*}
\label{SecEvenTanEven1}\eex

\bex Compute $\ds{\int\sec^72x\tan^52x\,dx}$

\underline{Solution}:
Here we have an odd number of tangent factors, so we peel off a 
$\sec2x\tan2x$ factor to be the functional part of the differential.
Eventually we then have $u=\sec2x$, giving $du=2\sec2x\tan2x\,dx$
and thus $\frac12\,du=\sec2x\tan2x\,dx$.
\begin{align*}
\int\sec^72x\,\tan^52x\,dx
 &=\int\sec^62x\,\tan^42x\,\sec2x\tan2x\,dx\\
 &=\int\sec^62x\left(\sec^22x-1\right)^2\sec2x\tan2x\,dx\\
 &=\int u^6\left(u^2-1\right)^2\cdot\frac12\,du\\
 &=\frac12\int u^6\left(u^4-2u^2+1\right)\,du\\
 &=\frac12\int\left(u^{10}-2u^8+u^6\right)\,du\\
 &=\frac12\left[\frac1{11}\,u^{11}-\frac29\,u^9+\frac17\,u^7\right]+C\\
 &=\frac1{22}\sec^{11}2x-\frac19\sec^92x+\frac1{14}\sec^72x+C.
\end{align*}
\eex
In fact this last example could be computed by first rewriting
the integral in terms of cosines and sines:
\begin{align*}
\int\frac{\sin^52x}{\cos^{12}2x}\,dx
&=\int\frac{\sin^42x}{\cos^{12}2x}\sin2x\,dx
=\int\frac{\left(1-\cos^22x\right)^2}{\cos^{12}2x}\sin2x\,dx\\
&=\int\frac{\left(1-u^2\right)^2}{u^{12}}\cdot\frac{-1}2\,du
=-\frac12\int u^{-12}\left(1-2u^2+u^4\right)\,du,\text{ etc.}
\end{align*}
Thus, the relationships involving the secant and tangent are
not required in this last example.  However, rewriting
the integral in the previous problem, Example~\ref{SecEvenTanEven1},
in terms of sines and cosines would not yield either to an odd
power.  Thus  Example~\ref{SecEvenTanEven1} illustrates
an integral which does benefit from the extra structure
(algebraic and calculus) of the secant-tangent relationship.

\bex Compute $\ds{\int\tan^4x\,dx}$.

\underline{Solution}:  Here we look at two solutions.
In the first, instead of exploiting the fact that there are an even
number of factors of secant (namely zero) present here,
we will repeatedly use the fact that $\tan^2\theta+1=\sec^2\theta$.
(In the second line, we let $u=\tan x$.)
\begin{align*}
\int\tan^4x\,dx&=\int\tan^2x\left(sec^2x-1\right)\,dx\\
               &=\int\underbrace{\tan^2x}_{u^2}\underbrace{\sec^2x}_{du}
                 \,dx -\int\tan^2x\,dx\\
               &=\frac13\tan^3x-\int\tan^2x\,dx\\
               &=\frac13\tan^3x-\int\left(\sec^2x-1\right)\,dx\\
               &=\frac13\tan^3x-\tan x+x+C.\end{align*}
Of course the other method is to ``peel off'' a factor of $\sec^2x$, which
we do even
though it does not really appear.  To have it appear, we will
multiply and divide the integrand by $\sec^2x$.  Then
we will let $u=\tan x$.
A long division will give us the sum of powers in our final integral
below.
\begin{align*}
\int\tan^4x\,dx&=\int\frac{\tan^4x}{\sec^2x}\sec^2x\,dx\\
               &=\int\frac{\tan^4x}{\tan^2x+1}\sec^2x\,dx\\
               &=\int\frac{u^4}{u^2+1}\,du\\
               &=\int\left(u^2-1+\frac1{u^2+1}\right)\,du\\
               &=\frac13\,u^3-u+\tan^{-1}u+C_1\\
               &=\frac13\tan^3x-\tan x+\tan^{-1}(\tan x)+C_1\\
               &=\frac13\tan^3x-\tan x+x+C.
\end{align*}
Here we did have the extra complication of long division.
Furthermore, to see that the two answers were the same we
had to notice
$\tan^{-1}(\tan x)=x+n\pi$, where $n\in\mathbb{Z}$, i.e., $n$ is
an integer.\footnote{
%%% FOOTNOTE
Recall knowing $\tan x$ does not mean we know
the angle $x$, but we do know it to an integer multiple of $\pi$,
which is the period of tangent.  Recall also that
tangent is one-to-one in each such period. (Of course the
integral is only defined on each period individually,
separated by the discontinuities---in fact vertical 
asymptotes---of the integrand.)  Furthermore, the arctangent
function only outputs angles in the period $-\pi/2<\theta<\pi/2$,
so $\tan^{-1}(\tan x)\in(-\pi/2,\pi/2)$, where $x$ is not so restricted.
Eventually, this is all taken care of by the arbitrary nature
of the constant $C$.%
%%% END FOOTNOTE
}  Thus the final constant $C$ takes into account $C_1-n\pi$, still
a constant.
\eex

\subsection{Even Powers of Cosecant or Odd Powers of Cotangent
\label{EvenCosecant/OddCotangent}}

Here we just point out that a similar relationship 
exists between the cosecant and cotangent, as exists
between the secant and tangent.  We briefly look
at two examples to illustrate this.   The integral type
is 
\begin{equation}
\int\csc^m\theta\,\cot^n\theta\,d\theta.\label{CosecantCotangentForm1}
\end{equation}
We begin with  the following facts from trigonometry and calculus:
\begin{alignat*}{3}
\cot^2\theta+1&=\csc^2\theta,&\qquad\qquad
            \frac{d}{d\theta}\cot\theta&=-\csc^2\theta,\\
\csc^2\theta-1&=\cot^2\theta,&\qquad\frac{d}{d\theta}
        \csc\theta&=-\csc\theta\cot\theta.\end{alignat*}

The techniques we will employ mirror those used for
the secant-tangent integrals:
\begin{enumerate}
\item If an integral of the form (\ref{CosecantCotangentForm1})
contains an odd power of cotangent, we peel off a factor
$\csc\theta\cot\theta$ to be the functional part of the differential.
This leaves an even power of cotangent, which can be written as
a power of $\left(\csc^2\theta-1\right)$.
\item If an integral of the form (\ref{CosecantCotangentForm1})
contains an even power of cosecant, we peel off a factor
$\csc^2\theta$ to be the functional part of the differential.
The remaining even power of cosecant is then written as 
a power of $\left(\cot^2\theta+1\right)$.
\end{enumerate}

\bex Compute $\ds{\int\csc^8x\,\cot^2x\,dx}$.

\underline{Solution}: We see an even number of cosecants, so
we peel off two to be part of the differential.
\begin{align*}
\int\csc^8x\,\cot^2x\,dx&=\int\csc^6x\,\cot^2x\,\csc^2x\,dx\\
    &=\int\left(\csc^2x\right)^3\cot^2x\,\csc^2x\,dx\\
    &=\int\left(\cot^2x+1\right)^3\cot^2x\,\csc^2x\,dx.
\end{align*} 
Taking $u=\cot x$, giving $du=-\csc^2x\,dx$, so $-du=\csc^2x\,dx$,
we get
\begin{align*}
\int\csc^8x\,\cot^2x\,dx&\int\left(\cot^2x+1\right)^3\cot^2x\,\csc^2x\,dx\\
         &=\int\left(u^2+1\right)^3u^2(-du)\\
         &=-\int\left(u^6+3u^4+3u^2+1\right)u^2\,du\\
         &=-\int\left(u^8+3u^6+3u^4+u^2\right)\,du\\
         &=-\frac19\,u^9-\frac37\,u^7-\frac35\,u^5-\frac13\,u^3+C\\
         &=-\frac19\cot^9x-\frac37\cot^7x-\frac35\cot^5x-\frac13\cot^3x+C.
\end{align*}
\eex
%\newpage
\bex Compute $\ds{\int\csc^3\frac{x}2\,\cot^3\frac{x}2\,dx}$.

\underline{Solution}: Here the cotangent appears to an odd
power, so we will peel of one cosecant and one cotangent.
\begin{align*}
\int\csc^3\frac{x}2\,\cot^3\frac{x}2\,dx
  &=\int\csc^2\frac{x}2\,\cot^2\frac{x}2\,\csc\frac{x}2\,\cot\frac{x}2\,dx\\
  &=\int\csc^2\frac{x}2\left(\csc^2\frac{x}2-1\right)
          \csc\frac{x}2\cot\frac{x}2\,dx.
\end{align*}
Now we let $u=\csc\frac{x}2$, implying
$du=-\csc\frac{x}2\cot\frac{x}2\cdot\frac12\,dx$,
whence
$-2\,du=\csc\frac{x}2\cot\frac{x}2\,dx$.
Our integral then becomes
\begin{align*}
\int\csc^3\frac{x}2\,\cot^3\frac{x}2\,dx
 &=\int\csc^2\frac{x}2\left(\csc^2\frac{x}2-1\right)
          \csc\frac{x}2\cot\frac{x}2\,dx\\
 &=\int u^2\left(u^2-1\right)(-2)\,du\\
 &=-2\int\left(u^4-u^2\right)\,du\\
 &=-\frac25\,u^5+\frac23\,u^3+C\\
 &=-\frac25\csc^5\frac{x}2+\frac23\csc^3\frac{x}2+C.
\end{align*}
\eex

\subsection{Even Powers of Sine and Cosine}
Now we turn our attention to the question of integration
when both powers of sine and cosine are even.  There
are two standard methods for handling this:
integration by parts, and ``half-angle formulas.''
The former is more useful when the powers are small
than when they are large, and the latter is perhaps
more general.\footnote{%
%%% FOOTNOTE
In today's calculus texts, integration by parts is less
prominently presented for such integrals, while half-angle
methods are more popular among authors.  We present
both here for the lower powers, as some of the phenomena 
found in the integration by parts 
for such integrals are found later in this section.%
%%%% END FOOTNOTE
}

\bex Compute $\ds{\int\sin^2x\,dx}$ using integration by parts.

\underline{Solution}: This exact computation was performed
in
Example~\ref{FirstPartsIntegrationOfSinSquaredX},
page~\pageref{FirstPartsIntegrationOfSinSquaredX}.
So that it is in front of us here, we summarize that 
computation:
\begin{align*}
\I&=\int\underbrace{\sin x}_{u}\underbrace{\sin x\,dx}_{dv}
   =\underbrace{(\sin x)}_v\underbrace{(-\cos x)}_{v}
         -\int\underbrace{(-\cos x)}_v\underbrace{\cos x\,dx}_{du}
   =-\sin x\cos x+\int\cos^2x\,dx\\
  &=-\sin x\cos x+\int\left(1-\sin^2x\right)\,dx
   =-\sin x\cos x+x-\int\sin^2x\,dx\\
  &=x-\sin x\cos x-\I.
\end{align*}
At this point we add $\I=\int\sin^2x\,dx$ to both sides to get
\begin{align*}
2\int\sin^2x\,dx&=x-\sin x\cos x+C_1\\
\implies
\int\sin^2x\,dx&=\frac12\left(x-\sin x\cos x\right)+C.\end{align*}
\label{IntOfSineSquaredForTrigIntegrals}\eex

The method above works well for integrating
$\sin^2x$ or $\cos^2x$, but higher, even powers
become more cumbersome.  For this reason it is 
common to opt for alternatives involving 
slightly more sophisticated trigonometric identities.
There is some redundancy in the list below, as 
(\ref{SineIsOdd}), (\ref{CosineIsEven}),
(\ref{Sine(A+B)}) and (\ref{Cosine(A+B)}) together imply 
the others.
\begin{align}
\sin(-\theta)&=-\sin\theta,\label{SineIsOdd}\\
\cos(-\theta)&=\cos\theta,\label{CosineIsEven}\\
\sin(A+B)&=\sin A\cos B+\sin B\cos A\label{Sine(A+B)}\\
\sin(A-B)&=\sin A\cos B-\sin B\cos A\label{Sine(A-B)}\\
\cos(A+B)&=\cos A\cos B-\sin A\sin B\label{Cosine(A+B)}\\
\cos(A-B)&=\cos A\cos B+\sin A\sin B\label{Cosine(A-B)}\\
\sin 2A&=2\sin A\cos A\label{Sine2A}\\
\cos 2A&=\cos^2A-\sin^2A\label{Cosine2A-Version1}\\
\cos 2A&=2\cos^2A-1\label{Cosine2A-Version2}\\
\cos 2A&=1-2\sin^2A.\label{Cosine2A-Version3}
\end{align}
It is left for the exercises to show that 
\begin{enumerate}
\item (\ref{Sine(A-B)})
follows from replacing $B$ with $-B$ in (\ref{Sine(A+B)}),
\item Similarly, (\ref{Cosine(A-B)})
follows from (\ref{Cosine(A+B)}).
\item (\ref{Sine2A}) follows from (\ref{Sine(A+B)}), if we let $B=A$.
\item Similarly (\ref{Cosine2A-Version1}) follows from (\ref{Cosine(A+B)}).
\item (\ref{Cosine2A-Version2}) and (\ref{Cosine2A-Version3})
      follow from (\ref{Cosine2A-Version1}) and the identity
      $\sin^2A+\cos^2A=1$.
\end{enumerate}
Now (\ref{Cosine2A-Version2}) and (\ref{Cosine2A-Version3}) can 
be rewritten as follow:
\begin{align*}
\cos2A+1&=2\cos^2A,\\
2\sin^2A&=1-\cos2A.\end{align*}
Replacing $A$ with $\theta$, we can divide these by 2 to get
so-called half-angle formulas:\footnote{%
%%% FOOTNOTE
Equations (\ref{Half-AngleForCosineSquared}) 
and (\ref{Half-AngleForSineSquared})
are called half-angle formulas because the 
angle $\theta$ on the left is half of the angle $2\theta$ on the
right.  In fact, knowing the location of the terminal side of an
angle does not tell us where its half is located.
Indeed, $90^\circ$ and $450^\circ$ are coterminal, but 
their half angles, $45^\circ$ and $225^\circ$ are not.
This is reflected in what are given as ``half-angle'' formulas
in most trigonometry texts (compare to
(\ref{Half-AngleForCosineSquared}) and (\ref{Half-AngleForSineSquared})):
\begin{align*}
\cos\frac{\alpha}2&=\pm\sqrt{\frac{1+\cos\alpha}2}\\
\sin\frac{\alpha}2&=\pm\sqrt{\frac{1-\cos\alpha}2}.\end{align*}
However, knowing where an angle terminates does determine where
twice the angle terminates, as is reflected in 
(\ref{Sine2A})--(\ref{Cosine2A-Version3}).
%%% END FOOTNOTE
}
\begin{align}
\cos^2\theta&=\frac12\left(1+\cos2\theta\right),
              \label{Half-AngleForCosineSquared}\\
\sin^2\theta&=\frac12\left(1-\cos2\theta\right).
             \label{Half-AngleForSineSquared}
\end{align}
Using (\ref{Half-AngleForSineSquared}), we see
$$\int\sin^2x\,dx=\int\frac12(1-\cos2x)\,dx
=\frac12\left[x-\frac12\sin2x\right]+C.$$
For reasons which will be clear in the next section, it is
often desirable that the angle in the final answer agree
with the original angle, in this case $x$.
For that we use the double-angle formula (\ref{Sine2A}),
to get
\begin{align*}\int\sin^2x\,dx&=\frac12\left[x-\frac12\sin2x\right]+C
 =\frac12\left[x-\frac12\cdot2\sin x\cos x\right]+C\\
 &=\frac12x-\frac12\sin x\cos x+C.\end{align*}
This agrees with the answer we obtained through
integration by parts, in 
Example~\ref{IntOfSineSquaredForTrigIntegrals},
page~\pageref{IntOfSineSquaredForTrigIntegrals}.

\newpage
\bex Compute $\ds{\int\sin^4x\,dx}$.

\underline{Solution}: Here we use the half-angle formulas 
repeatedly, until our integral has no positive, even powers of sine or cosine:
\begin{align*}
\int\sin^4x\,dx&=\int\left(\sin^2x\right)^2\,dx
                =\int\left[\frac12\left(1-\cos2x\right)\right]^2\,dx\\
               &=\frac14\int\left(1-2\cos2x+\cos^22x\right)\,dx\\
               &=\frac14\int\left[1-2\cos2x+\frac12(1+\cos4x)\right]\,dx\\
               &=\frac14\int\left[\frac32-2\cos2x+\frac12\cos4x\right]\,dx\\
          &=\frac14\left[\frac32\,x-\sin2x+\frac12\cdot\frac14\sin4x\right]+C\\
        &=\frac38x-\frac14\sin2x+\frac1{32}\sin4x+C.
\end{align*}
This answer is correct, but if we want to match the angles to the
original ($x$), we can use some double-angle formulas
(\ref{Sine2A}) and (\ref{Cosine2A-Version1}):
\begin{align*}
\int\sin^4x\,dx&=\frac38x-\frac14\sin2x+\frac1{32}\sin4x+C\\
&=\frac38\,x-\frac14\cdot2\sin x\cos x+\frac1{32}\cdot2\sin2x\cos2x+C\\
&=\frac38\,x-\frac12\sin x\cos x+\frac1{16}(2\sin x\cos x)(\cos^2x-\sin^2x)+C,
\end{align*}               
which can again be simplified and rewritten in several ways.
\eex

\bex Compute $\ds{\int\sin^23x\,\cos^23x\,dx}$.

\underline{Solution}: 
\begin{align*}
\int\sin^23x\,\cos^23x\,dx
 &=\int\frac12(1-\cos6x)\cdot\frac12(1+\cos6x)\,dx
  =\frac14\int\left(1-\cos^26x\right)\,dx\\
 &=\frac14\int\left[1-\frac12(1+\cos12x)\right]\,dx
  =\frac14\int\left[\frac12-\frac12\cos12x\right]\,dx\\
 &=\frac14\left[\frac{x}2-\frac12\cdot\frac1{12}\sin12x\right]+C
  =\frac{x}8-\frac1{96}\sin12x+C.
\end{align*}
(We could have used $1-\cos^26x=\sin^26x$ after the first line.)
As before, if we would like to have our answer in terms of the
original angle, we need to utilize the double angle formulas
(\ref{Sine2A}) and (\ref{Cosine2A-Version1}):
\begin{align*}
\int\sin^23x\,\cos^23x\,dx
  &=\frac{x}8-\frac1{96}\sin12x+C\\
  &=\frac{x}8-\frac1{96}\cdot2\sin6x\cos6x+C\\
  &=\frac{x}8-\frac1{48}(2\sin3x\cos3x)(\cos^23x-\sin^23x)+C\\
  &=\frac{x}8-\frac1{24}\sin3x\cos3x(\cos^23x-\sin^23x)+C.
\end{align*}
\eex

\subsection{Miscellaneous Problems and Methods}
There are many trigonometric integrals that
either do not fit one of the above categories, or
for which those methods are unwieldy.  We
will look at several such here.  The reader should 
realize, however, that we cannot exhaust all
possibilities here, and a particular problem may have
a particularly clever solution which does not generalize
well to other problems.

The methods of the earlier subsections are all standard and any
successful calculus student is expected to know them.
The first few examples here are of this class also, in
that the better students should be able to handle these
without resorting to references.  We will, however, 
eventually have methods in this subsection
which such a student should be
aware of, but is understandably less likely to be able to 
recite from memory.
All are derivable, but again, the latter are somewhat more 
obscure and even an excellent students might prefer to use 
a reference.  It is important, however,  that all students
be aware of these latter classes of problems, 
and the available methods of solution,
regardless of whether a reference is used ultimately.

\bex Compute $\ds{\int\sec^3x\,dx}$.

\underline{Solution}: Here we have an odd number of secants,
and an even (zero) number of tangents.  Unfortunately our
ealier methods called for an even number of secants or
an odd number of tangents.  We could notice that the integrand
represents an odd number ($-3$) of cosines, and
then with $u=\sin x$, we could write
$$\int\sec^3x\,dx=\int\frac1{\cos^3x}\,dx=\int\frac1{\cos^4x}\,\cos x\,dx
                 =\int\frac1{\left(1-\sin^2x\right)^2}\cos x\,dx
                 =\int\frac1{\left(1-u^2\right)^2}\,du,$$
but in fact we have yet to discuss how to integrate that final
form. (We will in Section~\ref{PartialFracSection},
and while it will be somewhat long, it will be a straightforward computation
there).
Instead we will next try integration by parts.  
Since the integrand contains an easily integrated $\sec^2x\,dx$ factor, we will
let that be $dv$:
\begin{alignat*}{3}
u&=\sec x&&\qquad\qquad&dv&=\sec^2x\,dx\\
du&=\sec x\tan xdx&&&v&=\tan x 
\end{alignat*}
$$\int\sec^3x\,dx=uv-\int v\,du=\sec x\tan x-\int \sec x\tan^2x\,dx.$$
Now it is tempting to do another parts step, with $dv=\sec x\tan x\,dx$,
but---as happened in some previous examples---we would then have
our original integral on the left, and the same on the right.
What works here is to instead use one of the basic trigonometric 
identities at this step:
\begin{align*}
\int\sec^3x\,dx&=\sec x\tan x-\int\sec x \tan^2x\,dx\\
           &=\sec x \tan x-\int\sec x\left(\sec^2x-1\right)+C\\
       &=\sec x\tan x -\int\sec^3 x\,dx +\int\sec x\,dx\\
       &=\sec x\tan x+\ln|\sec x+\tan x|-\int\sec^3x\,dx.
\end{align*}
Adding $\int\sec^3x\,dx$ to both sides gives
\begin{align*}
2\int\sec^3x\,dx&=\sec x\tan x+\ln|\sec x+\tan x|+C_1\\
\implies
\int\sec^3x\,dx&=\frac12(\sec x\tan x+\ln|\sec x+\tan x|)+C.
\end{align*}
\eex

When integrating arbitrary integer powers of the trigonometric
functions, a common technique is to make use of so-called
{\it reduction formulas}.  These are derived using integration
by parts, often incorporating the kind of computation 
above.  For instance, let us consider the general problem of
integrating $\sec^nx$, where $n\ge3$.  Such an  integral contains
within its integrand
the factor $\sec^2x$, which we use in the $dv$ term.
Integration by parts can proceed as follows:
\begin{alignat*}{3}
u&=\sec^{n-2} x&&\qquad\qquad&dv&=\sec^2x\,dx\\
du&=(n-2)\sec^{n-3}x\,\sec x \cdot\tan x\,dx&&&v&=\tan x\\
du&=(n-2)\sec^{n-2}x\,\tan x\,dx 
\end{alignat*}
giving us
\begin{align*}
\int\sec^nx\,dx
&=\int\underbrace{\sec^{n-2}x}_{u}\underbrace{\sec^2x\,dx}_{dv}\\
&=\sec^{n-2}x\,\tan x-\int(n-2)\sec^{n-2}x\,\tan^2x\,dx\\
&=\sec^{n-2}x\,\tan x-\int(n-2)\sec^{n-2}x\left(\sec^2x-1\right)\,dx\\
&=\sec^{n-2}x\,\tan x-(n-2)\int\sec^nx\,dx+(n-2)\int\sec^{n-2}x\,dx\\
\implies
(n-1)\int\sec^nx\,dx&=\sec^{n-2}x\,\tan x+(n-2)\int\sec^{n-2}x\,dx.
\end{align*}
Now we can divide by $(n-1)$ to get a general reduction formula
\begin{equation}
\int\sec^nx\,dx=\frac1{n-1}\,\sec^{n-2}x\,\tan x+\frac{n-2}{n-1}\int
\sec^{n-2}x\,dx.\label{ReductionFormulaForSecant}\end{equation}
This is called a reduction formula because the resulting integral
is of a lower power of secant.  A quick inspection reveals that
this formula is valid for $n=2$ as well, so it is if fact valid
for $n\ge2$.  

\bex Compute $\ds{\int\sec^5x\,dx}$.

\underline{Solution}: Here we will invoke the formula twice:
once for $n=5$, and then again for $n=3$ to deal with the
resulting integral.  That will give an integral of secant
to the first power, which is one which should be already
memorized.
\begin{alignat*}{2}
\int\sec^5x\,dx&=\frac14\sec^3x\,\tan x+\frac{3}{4}\int\sec^3x\,dx
 &\qquad&\text{($n$=5 in (\ref{ReductionFormulaForSecant}))}\\
 &=\frac14\sec^3x\,\tan x+\frac34\left[\frac12\sec x\tan x+\frac12
              \int\sec x\,dx\right]&
        &\text{($n$=3 in (\ref{ReductionFormulaForSecant}))}\\
 &=\frac14\sec^3x\,\tan x+\frac38\sec x\tan x+\frac38
         \ln|\sec x+\tan x|+C.\end{alignat*}
\eex

Other reduction formulas which can be arrived at similarly
include
\begin{align}
\int\sin^nx\,dx&=-\frac{\sin^{n-1}x\,\cos x}{n}
       +\frac{n-1}n\int\sin^{n-2}x\,dx,
        \label{ReductionFormulaForSine}\\
\int\cos^nx\,dx&=\frac{\cos^{n-1}x\,\sin x}{n}
       +\frac{n-1}n\int\cos^{n-2}x\,dx.
        \label{ReductionFormulaForCosine}
\end{align}

\bex Compute $\ds{\int\cos^65x\,dx}$.

\underline{Solution}: Here we cannot use the formula
(\ref{ReductionFormulaForCosine}) directly, because our angle does
not match our differential.  To compensate, we will
perform a substitution step first.  Specifically, we will
let $u=5x$, so $du=5dx$ and thus $\frac15\,du=dx$, giving
\begin{alignat*}{2}
\int\cos^65x\,dx&=\int\cos^6u\cdot\frac15\,du\\
&=\frac15\int\cos^6u\,du\\
&=\frac15\left[\frac{\cos^5u\,\sin u}6+\frac56\int\cos^4u\,du\right]
        &\qquad&\text{($n=6$ in (\ref{ReductionFormulaForCosine}))}\\
&=\frac{\cos^5u\,\sin u}{30}+\frac16\left[\frac{\cos^3u\,\sin u}{4}
           +\frac34\int\cos^2u\,du\right]
          &&\text{($n=4$ in (\ref{ReductionFormulaForCosine}))}\\
&=\frac{\cos^5u\,\sin u}{30}+\frac{\cos^3u\,\sin u}{24}
           +\frac18\left[\frac{\cos u\sin u}2+\frac12\int 1\,du\right]
          &&\text{($n=2$ in (\ref{ReductionFormulaForCosine}))}\\
&=\frac{\cos^5u\,\sin u}{30}+\frac{\cos^3u\,\sin u}{24}
           +\frac{\cos u\sin u}{16}+\frac1{16}\,u+C\\
&=\frac{\cos^55x\,\sin5x}{30}+\frac{\cos^35x\,\sin 5x}{24}
           +\frac{\cos 5x\sin 5x}{16}+\frac{5x}{16}+C.
\end{alignat*}
\eex

Clearly reduction formulas can be very useful.  Indeed they 
provide an iterative method for reducing an integral computation,
step by step, until---hopefully---a manageable integral appears.
In fact in 
both examples above, we used the reduction formula for one more step
than necessary, because it was easier than 
recomputing $\int\sec^3x\,dx$ or $\int\cos^2u\,du$ as before.
Furthermore, many of the technical details for finding these
integrals are built into the reduction formulas.

As useful as the reduction formulas are, they have a couple of
minor drawbacks.  First, when the angle does not match the differential,
some substitution needs to be performed to compensate.
Second---and more serious---is that any attempt to memorize these is likely
to result in error.  Thus the student of calculus needs to learn
the earlier methods and be able to perform such calculations
unaided, and also know that these reduction formulas (and others)
are available and know how to use them.\footnote{%
%%% FOOTNOTE
Such formulas can be found in most engineering/science calculus
texts, as well as books containing tables of integration formulas
such as the {\it CRC Standard Mathematical Tables and Formulae}.
%%% END FOOTNOTE
}

Now we consider integrals of following three forms, where $m\ne n$:
$$\int\sin mx\,\sin nx\,dx,
\qquad\qquad
\int\sin mx\,\cos nx\,dx,
\qquad\qquad
\int\cos mx\,\cos nx\,dx.$$
What distinguishes these is that the angles of the trigonometric functions
do not agree.  There are two methods for computing these:
integration by parts, and utilizing the following trigonometric identities:
\begin{align}
\sin A\cos B&=\frac12\left[\sin(A-B)+\sin(A+B)\right],\label{SinACosB}\\
\sin A\sin B&=\frac12\left[\cos(A-B)-\cos(A+B)\right],\label{SinASinB}\\
\cos A\cos B&=\frac12\left[\cos(A-B)+\cos(A+B)\right].\label{CosACosB}
\end{align}
For obvious reasons, these are called {\it product-sum formulas}.
These follow from adding or subtracting
(\ref{Sine(A+B)}), (\ref{Sine(A-B)}), (\ref{Cosine(A+B)}),
and (\ref{Cosine(A-B)}), which we repeat here for reference:
\begin{align*}
\sin(A+B)&=\sin A\cos B+\sin B\cos A,\\
\sin(A-B)&=\sin A\cos B-\sin B\cos A,\\
\cos(A+B)&=\cos A\cos B-\sin A\sin B,\\
\cos(A-B)&=\cos A\cos B+\sin A\sin B.\end{align*}
For instance, (\ref{SinACosB}) follows from adding the
first two of these and solving for $\sin A\cos B$.  All
three, (\ref{SinACosB})--(\ref{CosACosB}), are left 
to the exercises.

\bex Compute $\ds{\int\sin2x\cos5x\,dx}$.

\underline{Solution}: Here we use (\ref{SinACosB}),
with $A=2x$ and $B=5x$:
\begin{align*}
\int\sin2x\cos5x\,dx&=\int\frac12[\sin(2x-5x)+\sin(2x+5x)]\,dx\\
                    &=\int\frac12[\sin(-3x)+\sin7x]\,dx\\
                    &=\int\frac12[-\sin3x+\sin7x]\,dx\\
                    &=\frac13\cos3x-\frac17\cos7x+C.\end{align*}
\eex

The method above has the drawback that the solution does not
contain the same angles as the integrand. One can
get back to the original angles using the formulas
\begin{align*}
\cos3x&=\cos(5x-2x)=\cos5x\cos2x+\sin5x\sin2x,\\
\cos7x&=\cos(5x+2x)=\cos5x\cos2x-\sin5x\sin2x.\end{align*}
Alternatively, an integration by parts argument leaves intact
the angles.  It requires two integration by parts steps,
and we need to solve for the integral.  Furthermore, 
we have to make the analogous substitution for $u$ both
times, and for $dv$ both times.  By analogous, here
we mean using the same angle, $2x$ or $5x$, as the argument
of the trigonometric function both times.
If we always let the $u$-term have angle $2x$,
and the $dv$-term have angle $5x$, 
eventually the solution there will be
$$\int\sin2x\cos5x\,dx
  =\frac5{21}\sin2x\sin5x+\frac2{21}\cos2x\cos5x+C.$$
\newpage
\begin{center}
\underline{\Large{\bf Exercises}}\end{center}

\begin{multicols}{2}
Evaluate the following integrals.
\begin{enumerate}
\item $\ds{\int\sin x\cos x\,dx}$
\item $\ds{\int\sin^2x\cos x\,dx}$
\item $\ds{\int\sin x\cos^2x\,dx}$
\item $\ds{\int\sin^3x\cos^2x\,dx}$
\item $\ds{\int\sin^4x\cos^5x\,dx}$
\item $\ds{\int\frac{\sin^3x}{\cos^2x}\,dx}$
\item $\ds{\int\frac{\sin^3x}{\cos^2x+1}\,dx}$
\item $\ds{\int \sin^4x\cos^5x\,dx}$
\item $\ds{\int\sin^32x\cos^{15}2x\,dx}$
\item $\ds{\int\frac{\sin^2x}{\cos x}\,dx}$




\item $\ds{\int\sin^3x\ln|\sin x|\,dx}$
\item $\ds{\int\cos^3x\ln|\sin x|\,dx}$
\end{enumerate}
\end{multicols}


\newpage
\section{Trigonometric Substitution\label{TrigSubSection}}
In this section we explore how integrals can sometimes be
solved by making some clever substitutions involving
trigonometric functions, even though the original integrals
themselves do not involve such functions.  
Before developing the general mechanics, the example below
is offered for motivation:

\bex Compute $\ds{\int\frac{\sqrt{1-x^2}}x\,dx}$.

\underline{Solution}:  Note first that this integral will not
simply yield to earlier techniques.  (The reader is welcome to 
try, to see where those methods eventually fall short.)

Note also that, due to the square root, 
we require $-1\le x\le 1$.  In fact we also cannot have $x=0$, but
that constraint will be consistent with our new integral upon substitution.
In fact it is the radical which is giving us the most difficulty here.

The solution to that problem is to realize that $[-1,1]$ is exactly
the range of the function $\sin\theta$.  Moreover, if we
allow $\theta\in[-\pi/2,\pi/2]$, then we have the mappings
from $\theta$ to $x$ and back as follows:

\begin{center}
\begin{pspicture}(-6,-.5)(6,.5)
\rput(-5,0){$\ds{\left[-\frac{\pi}2,\frac{\pi}2\right]}$}
\psline{|->}(-4,0)(-1,0)
\rput(-2.5,.3){$\sin\theta$}
\rput(0,0){$[-1,1]$}
\psline{|->}(1,0)(4,0)
\rput(5,0){$\ds{\left[-\frac{\pi}2,\frac{\pi}2\right]}$}
\rput(2.5,.3){$\sin^{-1}x$}
\end{pspicture}
\end{center}

We will use the substitution $x=\sin\theta$ in the integral above,
with the understanding that $\theta\in[-\pi/2,\pi/2]$ (excluding zero
due to the denominator).  As usual, it is also crucial to write 
the differential in terms of the new variable:
\begin{align*}
x&=\sin\theta\\
\implies dx&=\cos\theta\,d\theta.\end{align*}
Thus
$$\int\frac{\sqrt{1-x^2}}x\,dx
=\int\frac{\sqrt{1-\sin^2\theta}}{\sin\theta}\,\cos\theta\,d\theta.$$
Since we have chosen the range $\theta\in[-\pi/2,\pi/2]$ to get
our range for $x\in[-1,1]$ (again excluding zero), we have
angles in the first and fourth quadrants.  In those quadrants, in
fact, the cosine is nonnegative.  Thus
$$\theta\in\left[-\frac{\pi}2,\frac{\pi}2\right]
\implies\sqrt{1-\sin^2\theta}=\sqrt{\underbrace{\cos^2\theta}_{\ge0}}
=\cos\theta.$$
Summarizing our work so far, and then continuing, we get
\begin{align*}
\int\frac{\sqrt{1-x^2}}x\,dx
&=\int\frac{\sqrt{1-\sin^2\theta}}{\sin\theta}\,\cos\theta\,d\theta
=\int\frac{\cos\theta}{\sin\theta}\cos\theta\,d\theta\\
&=\int\frac{\cos^2\theta}{\sin\theta}\,d\theta
=\int\frac{1-\sin^2\theta}{\sin\theta}\,d\theta
=\int\left[\csc\theta-\sin\theta\right]\,d\theta\\ \vphantom{\int}
&=\ln|\csc\theta-\cot\theta|+\cos\theta+C.
\end{align*}
Now we have a trigonometric form of the antiderivative, but of course
the original integral was in $x$ and not $\theta$.  While we will
develop a more visual method in the subsections, here we will note that
$\sin\theta=x$, and $\cos\theta=\sqrt{1-\sin^2\theta}=\sqrt{1-x^2}$.
Of course once we have the sine and cosine functions, we have all
the trigonometric functions.  Summarizing again, but this time
adding the final substitution in $x$, we have
$$\int\frac{\sqrt{1-x^2}}x\,dx
=\ln|\csc\theta-\cot\theta|+\cos\theta+C
=\ln\left|\frac1x-\frac{\sqrt{1-x^2}}x\right|+\sqrt{1-x^2}+C.$$
\eex

The technique in this section---assuming the original
variable of integration is $x$---can be somewhat summarized, though
the details are rather sophisticated.  A summary could be
written as
follows:
\begin{enumerate}
\item Use an appropriate {\it trigonometric substitution}, i.e., 
let $x$ be in terms of some trigonometric function of an
angle $\theta$.  Substitution for $dx$ follows.
\item Simplify the trigonometric expression in the resulting 
integral.  In particular, if there were radicals
in the original integral they should usually not be present in the 
simplified trigonometric integral.
\item Solve the trigonometric integral.  The solution will be
in terms of $\theta$.
\item Substitute back in terms of $x$, as dictated by the 
original substitution where $x$-terms were replaced by $\theta$-terms.
\end{enumerate}

It should be noted at the outset that the trigonometric integrals
which arise here may require some re-writing before they succumb to 
our trigonometric integral methods of Section~\ref{TrigIntsSection}.
Furthermore, a problem which naturally gives rise to trigonometric
substitution (as in the previous example) may or may not
yield a simple trigonometric integral.  However, all trigonometric
integrals we will encounter here are of classes we considered
in Section~\ref{TrigIntsSection}, so ultimately those techniques
equipped us for our work here.













\newpage
\section{Partial Fractions and Integration\label{PartialFracSection}}
In this section we are interested in techniques for computing
integrals of the form
\begin{equation}
\int\frac{P(x)}{Q(x)}\,dx,\label{GeneralRationalForPFD}
\end{equation}
where $P(x)$ and $Q(x)$ are polynomials. 
This is not in general a simple problem, unless 
the integral in (\ref{GeneralRationalForPFD}) is from
very particular classes.  However, with the techniques
we explore here, we can break $\frac{P(x)}{Q(x)}$ into
simpler fractions whose integrals are relatively easy.
To see an advantage of such an approach, consider the following example.
\bex Compute $\ds{\int\frac{5x-1}{x^2-x-2}\,dx}$.

\underline{Solution}: Note that the numerator is not a 
simple (constant number)
multiple of the derivative of the denominator, so 
substitution will not give a simple $\int\frac1u\,du$-form.

However, it  happens that
\begin{equation}\frac{5x-1}{x^2-x-2}
  =\frac{5x-1}{(x-2)(x+1)}
  =\frac2{x+1}+\frac3{x-2}.\label{FirstPFDExamplePFD}
\end{equation}
Thus
\begin{align*}
\int\frac{5x-1}{x^2-x-2}\,dx
 & =\int\left(\frac2{x+1}+\frac3{x-2}\right)\,dx\\
 & =2\ln|x+1|+3\ln|x-2|+C\\
 & =\ln\left|(x+1)^2(x-2)^3\right|+C.\end{align*}
\eex

In this section we will develop the methods needed to 
find expansions of fractions such as 
in (\ref{FirstPFDExamplePFD}). The idea is to reverse
the high school
algebra exercises, which would have us {\it combine} 
sums or differences of fractions into a single fraction.  For 
purposes of integral calculus, it is almost always
better to instead deal with several, simpler fractions than
their combination into a single, more complicated fraction.

A significant amount of our work in this section will be
algebraic, specifically, developing the method of decomposing
a fraction $P(x)/Q(x)$ into simpler, ``partial fractions.''
In order for the method to work, we will require $P(x)$ to
have lower degree $Q(x)$. (If the degree of $P$ is at least
that of $Q$, we can use long division to write the function
as a polynomial plus $r(x)/Q(x)$, where the degree of $r$ is
less than that of $Q$.)


Though not crucial for the calculus, we will spend the next subsection 
looking roughly at the theory behind the general
form of these partial
fraction decompositions (PFD's), in hopes it will help
reinforce the rules themselves.  
In Subsection~\ref{RulesForPDFSubsection} we will 
definitively write the
rules for PFD's without reference to integrals.
Finally, we will see how to solve for the coefficients 
of a particular PFD, in the context of computing antiderivatives
of these.

\subsection{Theory Behind the Forms of PFD's (Optional)}
The argument here is usually omitted from calculus texts,
and instead left to linear algebra courses.  However, the basic intuition 
is not difficult so we include it here, though the real work
is in later subsections.  In all of these we
are looking at functions
\begin{equation}
\frac{P(x)}{Q(x)},\qquad \text{$P$ and $Q$ polynomials, degree $P$ $<$
                                 degree $Q$}.
\label{DegP<DegQ}\end{equation}
Before stating the general rules for PFD's, we look at several examples
illustrating the underlying theory.

\bex For this example, we will argue in steps.
\begin{enumerate}
\item Consider all functions of the form 
\begin{equation}\frac{a x+b}{(x+1)(x-2)}.\label{Ex1ForProofOfPFD-Together}
\end{equation}
\item Now there are two linearly independent\footnotemark
functions, specifically $\frac{x}{(x+1)(x-2)}$
and $\frac1{(x+1)(x-2)}$ which---with linear combinations---can give us
any such function (\ref{Ex1ForProofOfPFD-Together}).  Indeed,
$$\frac{ax+b}{(x+1)(x-2)}=a\cdot\left[\frac1{(x+1)(x-2)}\right]
                         +b\cdot\left[\frac{x}{(x+1)(x-2)}\right].$$
In a linear-algebraic sense, we would say the functions of the
form (\ref{Ex1ForProofOfPFD-Together}) form a 2-dimensional space
(or 2-dimensional {\it vector space}), because to specify such
a function requires two constants, $a$ and $b$.

\item Now instead consider another 2-dimensional space of functions
given by linear combinations of the form
\begin{equation}
\frac{A}{x+1}+\frac{B}{x-2}
=A\cdot\left[\frac1{x+1}\right]+B\cdot\left[\frac1{x-2}\right].
\label{Ex1ForProofOfPFD-Apart}
\end{equation}
\item 
The functions $\frac1{x+1}$ and $\frac{1}{x-2}$ are indeed also linearly
independent, so the set of all functions of the form
(\ref{Ex1ForProofOfPFD-Apart}) also forms a 2-dimensional
vector space; to specify any such function requires
specifying two constants, $A$ and $B$.

\item Now notice that 
$$\frac{A}{x+1}+\frac{B}{x-2}
=\frac{A(x-2)+B(x+1)}{(x+1)(x-2)}
=\frac{(A+B)x+(-2A+B)}{(x+1)(x-2)},$$
which is of form (\ref{Ex1ForProofOfPFD-Together})
with $a=A+B$ and $b=-2A+B$.  In other words, any function of the
form (\ref{Ex1ForProofOfPFD-Apart}) can also be written in the 
from (\ref{Ex1ForProofOfPFD-Together}).
\item This tells us that the two-dimensional space
of functions $\frac{A}{x+1}+\frac{B}{x-2}$ is contained in the
two-dimensional space of functions $\frac{ax+b}{(x+1)(x-2)}$.
It is a fact of linear algebra that the only way for a
two-dimensional space to be contained in another two-dimensional
space is for them to be the same spaces.  (Think about 
a plane being contained in another plane, and realize that
they must then be the same plane.)

\item Finally, since (by 6 above) the space of all functions
of the form $\frac{ax+b}{(x+1)(x-2)}$ is the {\bf same} as the 
space of all functions of the form
$\frac{A}{x+1}+\frac{B}{x-2}$, it follows that
any function of the form $\frac{ax+b}{(x+1)(x-2)}$
can also be written in the form $\frac{A}{x+1}+\frac{B}{x-2}$.
\end{enumerate}
\eex
\footnotetext{We call functions $f_1,f_2,\cdots,f_n$ 
{\it linearly independent} if and only if 
it is impossible to write any of these as linear combinations
of the others. In other words, we {\it exclude} cases where there
exist {\it constants} $a_1,\cdots,a_{k-1},\ a_{k+1},\cdots,a_n\in\Re$
such that
\begin{align*}
f_k&=a_1f_1+a_2f_2+\cdots+a_{k-1}f_{k-1}+a_{k+1}f_{k+1}+\cdots+a_nf_n,\qquad
i.e.,\\
(\forall x)[\qquad f_k(x)&=a_1f_1(x)+a_2f_2(x)+\cdots+
          a_{k-1}f_{k-1}(x)+
          a_{k+1}f_{k+1}(x)+\cdots+a_{n}f_{n}(x)\qquad].
\end{align*}
}

Notice that functions of the form (\ref{Ex1ForProofOfPFD-Together})
are indeed also of the form $P(x)/Q(x)$ where $P$ is of degree
less than $Q$, since the degree of $P$ is at most 1 (zero if $a=0$) and the
degree of $Q$ is 2.

The argument above guarantees that a PFD such as 
(\ref{FirstPFDExamplePFD}) exists.  It is more desirable
for integration purposes to have form (\ref{Ex1ForProofOfPFD-Apart})
than (\ref{Ex1ForProofOfPFD-Together}).

\bex An argument similar to that of the previous example shows that
the following forms give exactly the same functions:
\begin{equation}\frac{ax^2+bx+c}{(x+1)(x+2)(x+3)}=\frac{A}{x+1}+
\frac{B}{x+2}+\frac{C}{x+3}.\label{Ex2ForProofOfPFD-Both}
\end{equation}
Of course $a,b,c$ are likely to differ from $A,B,C$.  Here the
underlying sets of linearly independent functions are, respectively,
\begin{align*}U&=\left\{\frac{x^2}{(x+1)(x+2)(x+3)}, 
           \frac{x}{(x+1)(x+2)(x+3)},
           \frac{1}{(x+1)(x+2)(x+3)}\right\},\\
              V&=\left\{\frac1{x+1},\frac1{x+2},\frac1{x+3}\right\}.
\end{align*}
Both sets of vectors span\footnote{%%%
%%% FOOTNOTE
The noun form of {\bf span} has a precise technical meaning.  The {\it span}
of ``vectors'' $v_1,v_2,\cdots,v_n$ is the set of all possible
linear combinations of those vectors.  Thus for example
$$\text{Span}\left\{\frac1{x+1},\frac1{x+2},\frac1{x+3}\right\}
=\left\{\left.a\cdot\left[\frac1{x+1}\right]+b\cdot\left[\frac1{x+2}\right]
        +c\cdot\left[\frac1{x+3}\right]\ \right| \ a,b,c\in\Re\right\}.$$

We would then say that the functions (vectors, in the linear algebra sense)
$\frac1{x+1},\frac1{x+2},\frac1{x+3}$, taken together, {\it span} 
the set described above.
%%% END FOOTNOTE
} 3-dimensional spaces.
It is not hard to see that functions on the right-hand side of
(\ref{Ex2ForProofOfPFD-Both}) can also be in the form on the left.
Indeed, if we combine the fractions on the right, we get
$$\frac{A}{x+1}+\frac{B}{x+2}+\frac{C}{x+3}
=\frac{\overbrace{A(x+2)(x+3)}^{\text{degree }\le2}+
        \overbrace{B(x+1)(x+3)}^{\text{degree }\le2}
        +\overbrace{C(x+1)(x+2)}^{\text{degree }\le2}}
      {(x+1)(x+2)(x+3)},$$
which gives us a polynomial in the numerator with degree
at most 2, as on the left-hand side of (\ref{Ex2ForProofOfPFD-Both}).
Here we have the span of $V$ contained in the span of $U$,
though they are both 3-dimensional spaces.  Thus they must be
the same spaces (we have one 3-dimensional space inside of another,
so they must be the same!), so in fact, anything written like
the left-hand side of (\ref{Ex2ForProofOfPFD-Both})
can be written like the right-hand side.  (In a later subsection
we will show how to find $A,B,C$ given $a,b,c$.)

It should be clear that integrating a function written like the
right-hand side of (\ref{Ex2ForProofOfPFD-Both}) is likely much simpler 
than integrating one in the form on the left.
\eex

\bex Next we argue that the following forms describe the same (space of)
functions:
\begin{equation}
\frac{ax^2+bx+c}{(x+7)^3}
  =\frac{A}{x+7}+\frac{B}{(x+7)^2}+\frac{C}{(x+7)^3}.
\label{Ex3ForProofOfPFD-Both}
\end{equation}
The underlying sets of linearly independent functions are, 
respectively,
$\left\{\frac{x^2}{(x+7)^3},\frac{x}{(x+7)^3},\frac1{(x+7)^3}\right\}$
and
$\left\{\frac1{x+7},\frac1{(x+7)^2},\frac1{(x+7)^3}\right\}$,
both spanning 3-dimensional spaces.  To show they are the same
spaces, we note that
$$\frac{A}{x+7}+\frac{B}{(x+7)^2}+\frac{C}{(x+7)^3}
  =\frac{A(x+7)^2+B(x+7)+C}{(x+7)^3},$$
which eventually simplifies to the form on the left-hand side of 
(\ref{Ex3ForProofOfPFD-Both}). Arguing as before, the underlying
sets of linearly independent functions must span the same
3-dimensional spaces, so anything of the form on the left-hand side
of (\ref{Ex3ForProofOfPFD-Both}) can be written also in the
form on the right-hand side.
\eex

\bex For our last example, we claim the following
forms give the same functions:
\begin{equation}
\frac{ax^4+bx^3+cx^2+dx+e}{x^3(x^2+1)}
=\frac{A}x+\frac{B}{x^2}+\frac{C}{x^3}+\frac{Dx+E}{x^2+1}.
\label{Ex4ForProofOfPFD-Both}
\end{equation}
Here the spanning sets of linearly independent functions are respectively
\begin{align*}
U&=\left\{\frac{x^4}{x^3(x^2+1)},\ \frac{x^3}{x^3(x^2+1)},\ 
\frac{x^2}{x^3(x^2+1)},\ \frac{x}{x^3(x^2+1)},\ \frac{1}{x^3(x^2+1)}\right\},\\
V&=\left\{\frac1x,\ \frac1{x^2},\ \frac1{x^3},\ \frac{x}{x^2+1},\ \frac1{x^2+1}
   \right\}.
\end{align*}
Again, the form on the right of (\ref{Ex4ForProofOfPFD-Both})
can be rewritten as below, simplifying into the form on the left-hand
side of (\ref{Ex4ForProofOfPFD-Both}) as
$$\frac{Ax^2(x^2+1)+Bx(x^2+1)+C(x^2+1)+(Dx+E)(x^2+1)}{x^3(x^2+1)}.$$
(Note that this  has numerator of degree at most 4.)
Because both sides of 
(\ref{Ex4ForProofOfPFD-Both}) must therefore
describe exactly the same functions
in a 5-dimensional vector space, it follows that anything
written in the form on the left of (\ref{Ex4ForProofOfPFD-Both})
can also be written in the form on the right.
\eex

In the next subsection we generalize the logic of
the examples above to write exact rules for the 
form of a PFD based upon the original fraction's denominator.  
Then in Subsection~\ref{SubsecForFindingIntsWPFD's}
we look at three methods of finding
the coefficients, $A$, $B$, $C$, etc., of the PFD expansion,
and immediately apply the methods to problems of computing integrals of
such functions.
\newpage
\subsection{Partial Fraction Decompositions: The Rules
\label{RulesForPDFSubsection}}

It is a fact of algebra (corollary to the Fundamental
Theorem of Algebra) that any polynomial with real coefficients can be 
factored uniquely---up to rearrangement of 
multiplicative constants---into powers of linear terms $(ax+b)^n$
and powers of irreducible quadratic\footnote{%%%
%%% FOOTNOTE
It is easy to see when a quadratic term is ``irreducible
over the real numbers,'' meaning we cannot write
it as $(ex+f)(gx+h)$, where $e,f,g,h\in\Re$, the latter being
equivalent to there being real numbers $\alpha,\beta$
such that the polynomial is zero there
(i.e., at $\alpha=-f/e$, $\beta=-h/g$). Using the quadratic formula,
it is plain that no such real solutions to the quadratic being
zero occur if and only if $b^2-4ac<0$ (i.e., when the term under the
radical in the quadratic formula is negative).}
%%% END FOOTNOTE
terms $(ax^2+bx+c)^m$ with real coefficients.  So for instance, 
$$x^3-x^2+x-1=(x-1)(x^2+1),$$
and there is no other way to factor it, except for 
instance $2(x-1)(\frac12x^2+\frac12)$, etc.
With that in mind, the rules for partial fraction decompositions
follow.

First, we are given a rational function $\frac{P(x)}{Q(x)}$,
where $P$ and $Q$ are polynomials.
\begin{enumerate}\setcounter{enumi}{-1}
\item In 1 and 2 below we assume deg $P <$ deg $Q$.
      If the deg $P \ge$ deg $Q$, then first we apply
      poynomial long division to achieve
      $$\frac{P(x)}{Q(x)}=p(x)+\frac{r(x)}{Q(x)},$$
      where $p,r$ are polynomials and deg $r <$ deg $Q$.
      Then the following rules apply to $\frac{r(x)}{Q(x)}$.
\item If $(ax+b)^n$, where $a\ne0$ occurs as a factor in $Q(x)$, then
      the partial fraction decomposition (PFD) of $\frac{P(x)}{Q(x)}$
      will contain terms
      $$\frac{A_1}{ax+b}+\frac{A_2}{(ax+b)^2}+\cdots+\frac{A_n}{(ax+b)^n}.$$
\item If $ax^2+bx+c$ is an irreducible quadratic, and
      $(ax^2+bx+c)^m$ occurs as a factor in $Q(x)$, then
      the PFD of  $\frac{P(x)}{Q(x)}$ will contain terms
      $$\frac{A_1x+B_1}{ax^2+bx+c}
          +\frac{A_2x+B_2}{(ax^2+bx+c)^2}
          +\cdots+\frac{A_mx+B_m}{(ax^2+bx+c)^m}.$$
\end{enumerate}

The application of these rules can be somewhat confusing at first,
so we will look at several examples before proceeding to 
solve for the coefficients $A_1, B_1$, etc.  
For the second case, we will mostly be interested in 
irreducible quadratics of the form $x^2+k^2$, where $k>0$.
Note that
we will usually use letters without subscripts, such as
$A,B,C$, and so on for our PFD coefficients (to be found later).

\bex Write the partial fraction decompositions for the given 
rational functions:
\begin{itemize}
\item $\ds{\frac{3x^5-11x^3+15x-2}{(x+1)^2(x-3)^4}
  =\frac{A}{x+1}+\frac{B}{(x+1)^2}+\frac{C}{x-3}
    +\frac{D}{(x-3)^2}+\frac{E}{(x-3)^3}+\frac{F}{(x-3)^4}}$.
\item $\ds{\frac1{(x+1)^2(x-3)^4}
=\frac{A}{x+1}+\frac{B}{(x+1)^2}+\frac{C}{x-3}
    +\frac{D}{(x-3)^2}+\frac{E}{(x-3)^3}+\frac{F}{(x-3)^4}}$.
\end{itemize}
In both cases, we had a polynomial of degree less than
6 divided by a polynomial of degree 6, so ``Rule 0''
is not invoked.  We also had two factors of $x+1$
in the denominator, so we needed a constant over the
first power, plus another constant over the second power,
of $x+1$.  With $(x-3)$ appearing to the third power in 
the denominator, we needed a constant over each of the
first, second, and third powers of $x+3$. (Of course the
choice of constants  $A,B,C,D,E$ will be different for these
two functions above, but the form of their PFD's is the same.) 
\eex

We do not want to be redundant in our PFD's, so if $Q(x)$
contains the factor $(x-3)^4$ but does not contain $(x-3)^5$,
for instance, we require constants divided by
$(x-3)$, $(x-3)^2$, $(x-3)^3$ and $(x-3)^4$ (but not $(x-3)^5$).
Now one could say that
such a $Q(x)$ also contains $(x-3)^2$, but we do not then
require in our PFD constants divided by $(x-3)$ and $(x-3)^2$ {\it again},
since these are already taken care of by those required by
the factor $(x-3)^4$ in $Q(x)$.

To rephrase the rules in light of the last paragraph,
if exactly $n$ factors of $(ax+b)$ appear in $Q(x)$, then
the PFD  contains terms 
$\frac{A_1}{ax+b}+\cdots+\frac{A_n}{(ax+b)^n}$.
If exactly  $m$ factors of $(ax^2+bx+c)$ appear, with $b^2-4ac<0$,
then the PFD contains terms
$\frac{A_1x+B_1}{ax^2+bx+c}+\cdots+\frac{A_mx+B_m}{(ax^2+bx+c)^m}$.
\bex Here are more PFD expansion forms.  (We do not solve for the 
     coefficients yet.)
\begin{itemize}
\item $\ds{\frac{x^4+x+1}{x^3(x^2+9)^2}
     =\frac{A}x+\frac{B}{x^2}+\frac{C}{x^3}
      +\frac{Dx+E}{x^2+9}+\frac{Fx+G}{(x^2+9)^2}}$.
\item $\ds{\frac1{(x^2+1)(x^2+4)}=\frac{Ax+B}{x^2+1}+\frac{Cx+D}{x^2+4}}$.
\item $\ds{\frac{x^5-8}{x(2x+1)^2(9-x)^3}
 =\frac{A}x+\frac{B}{2x+1}+\frac{C}{(2x+1)^2}
   +\frac{D}{9-x}+\frac{E}{(9-x)^2}+\frac{F}{(9-x)^3}}$.
\item $\ds{\frac2{x^2-5}=\frac2{\left(x-\sqrt5\right)\left(x+\sqrt5\right)}
     =\frac{A}{x-\sqrt5}+\frac{B}{x+\sqrt5}}$.
\item $\ds{\frac1{x^4-1}=\frac1{(x^2-1)(x^2+1)}
       =\frac1{(x+1)(x-1)(x^2+1)}
       =\frac{A}{x+1}+\frac{B}{x-1}+\frac{Cx+D}{x^2+1}}$.
\end{itemize}
\eex

The last two PFD's above required us to factor the denominators before
we started to implement the rules.  Note that we must be careful
to identify factors which are truly distinct.  Consider the following:
$$\frac1{x(x-3)(3x-9)}=\frac1{3x(x-3)^2}
=\frac{A}x+\frac{B}{x-3}+\frac{C}{(x-3)^2}.$$
The factors $(x-3)$ and $(3x-9)$ were not really distinct factors,
but were constant multiples of eachother.  If we do not notice this
we will find ourselves attempting a PFD 
with $\frac{A}{x}+\frac{B}{x-3}+\frac{C}{3x-9}$, but
the ``$B$'' and ``$C$'' terms are not independent, so we will miss
one dimension of possibilities for our PFD.  
Note also that the
factor $\frac13$ can be included in the first PFD term
(i.e., we could replace $\frac{A}x$ with $\frac{A}{3x}$, of course giving
a different ``$A$''), or its influence absorbed into the $A$, $B$ and $C$
terms.  We will usually opt for the latter approach (as we did above).

\subsection{Finding the Coefficients for PFD's
\label{SubsecForFindingIntsWPFD's}}
There are two main methods, and one auxiliary method, for finding
the coefficients $A$, $B$, etc., for a PFD.  The most efficient
method for a particular PFD is usually a mixture of the two 
main methods; perhaps the first method can be used to find $A$
and $C$, and the second to find $B$, for instance.
Efficiently 
computing the coefficients is thus somewhat of an art.\footnote{%%%
%%% FOOTNOTE
In fact either method is---strictly 
speaking---sufficient, and indeed there are textbooks which
teach only one or the other method.  However, trying to fit a particularly
complicated PFD into any single method will make for much more difficult
computations than are necessary. That said, for computer programming one
would likely choose one method and let the computer calculate the coefficients
by ``brute force.''} 
%%% END FOOTNOTE

The methods are based upon some properties of polynomials.  
Consider two polynomials 
\begin{align*}
 f(x)&=a_nx^n+a_{n-1}x^{n-1}+\cdots+a_2x^2+a_1x+a_0,\\
 g(x)&=b_mx^m+b_{m-1}x^{m-1}+\cdots+b_2x^2+b_1x+b_0.\end{align*}
The statement that  $f(x)=g(x)$ is an ``equality of polynomials,'' i.e., 
that $f(x)$ and $g(x)$ are the {\it same} polynomial is
equivalent to each of the following two conditions (separately):\footnotemark
%%% FOOTNOTE
\footnotetext{Some texts use the notation $f(x)\equiv g(x)$, read
``$f(x)$ is identically equal to $g(x)$.''  In other words, 
$f(x)$ and $g(x)$ are the same functions.
The word {\it identically}
is in the spirit of, for instance, trigonometric identities,
so one could write for example $\sin^2\theta+\cos^2\theta\equiv1$.
Of course we have a different use of the symbol ``$\equiv$,''
and will thus refrain from using it in this context, but the
reader should be aware of this common alternative use of the
symbol.} 
%%% END FOOTNOTE
\begin{enumerate}
\item $(\forall x\in\Re)[f(x)=g(x)]$.  In other words, $f$ and $g$
      are the same functions.
\item $(\forall i\in\{1,2,\cdots,\max\{m,n\}\})[a_i=b_i]$, that is, all the
      coefficients are the same.  (Note that it is possible, for 
      instance, that $m<n$, in which case we just take 
      $b_{m+1},\cdots,b_n=0$.)

\end{enumerate}
Furthermore, if $f(x)$ and $g(x)$ are the same polynomials, then
$f'(x)=g'(x)$, $f''(x)=g''(x)$, $f'''(x)=g'''(x)$ $\cdots$, in the 
sense of being the same polynomials, and so 1 and 2 from above
apply to these derivatives as well.

Our first method will exploit 1, the second 2, and the auxiliary
method will make use of
final observation about derivatives. The first method
essentially ``probes'' the two polynomials at different points,
usually chosen strategically, to get some quick information
out of a polynomial equality, and is often called an
``evaluation method.''  The  second method is often referred to
as ``comparing coefficients,'' and can also be useful for finding
quick information.  The auxiliary method exploits the fact that
the first methods can be applied to the derivatives (of any
order) of $f$ and $g$ to get further information quickly.

\bex Compute the integral $\ds{\int\frac1{x^2-5x+6}\,dx}$.

\underline{Solution}:  Here we have a degree-0 polynomial
divided by a degree-2 polynomial,
so the PFD rules apply.  Now one usually writes the PFD form of the 
integrand, complete with the unknown coefficients, before
proceeding to the methods of computing the coefficients.  In
other words, our first step would be to write:
$$\int\frac1{x^2-5x+6}\,dx
 =\int\frac1{(x-2)(x-3)}\,dx
 =\int\left[\frac{A}{x-2}+\frac{B}{x-3}\right]\,dx.$$
The next two lines can be skipped with practice, though
the first time one works this section they are worth writing
so the mechanics of the method can be understood and reinforced.
First we write the algebraic step (PFD) which was contained in the
rewriting of the integrands above:
$$\frac1{(x-2)(x-3)}=\frac{A}{x-2}+\frac{B}{x-3}.$$
This came from the fact that we have $x-2$ as a factor in the
denominator, but only once, and the same for $x-3$.
Next we multiply both sides by {\bf the denominator on the left}:
$$(x-2)(x-3)\left[\frac1{(x-2)(x-3)}\right]
  =(x-2)(x-3)\left[\frac{A}{x-2}+\frac{B}{x-3}\right].$$
On the left, the whole denominator cancels and we have the
numerator of the original fraction.  On the right we 
have to distribute the $(x-2)(x-3)$ across the sum in the
brackets.  For the ``$A$'' term the $(x-2)$ cancels, while
for the ``$B$'' term the $(x-3)$ cancels, giving us an
equality of polynomials:
%\newpage
\begin{equation}1=A(x-3)+B(x-2).\label{PolyEqualityForPFD1}
\end{equation}
Because this is an equality of polynomials ($f(x)=g(x)$ where 
$f(x)=1$ and $g(x)=A(x-3)+B(x-2)$), it must be true for any 
$x\in\Re$.  Now we choose two values of $x$ strategically.
\begin{alignat*}{2}
\underline{x=3}:\qquad&& 1&=A(3-3)+B(3-2)\implies\boxed{1=B}\\
\underline{x=2}:\qquad&& 1&=A(2-3)+B(2-2)\\
              &&\implies 1&=-A\implies\boxed{A=-1}.\end{alignat*}
Now we summarize what we have so far, and compute the desired integral:
\begin{align*}
\int\frac1{x^2-5x+6}&=\int\left[\frac{-1}{x-2}+\frac{1}{x-3}\right]\,dx\\
                    &=-\ln|x-2|+\ln|x-3|+C\\
                    &=\ln\left|\frac{x-3}{x-2}\right|+C.
\end{align*}
(The last step is not necessary, but for reasons of style
many textbooks combine logarithmic terms into a single logarithm.)
\eex

Because (\ref{PolyEqualityForPFD1}) was an equality of polynomials
(meaning the polynomial on the left is {\it the same polynomial}
as that on the right\footnotemark), we could substitute
any number for $x$ in (\ref{PolyEqualityForPFD1})
and still have a true statement.
Fortunately, there were choices which could eliminate 
an unknown, leaving an equation in the other unknown
which is easily solved.


%%% FOOTNOTE
\footnotetext{It should be pointed out that when
we write a PFD, for instance
$$\frac1{(x-2)(x-3)}=\frac{A}{x-2}+\frac{B}{x-3},$$
we mean that these are the same functions as well, so once
we find $A$ and $B$, the right-hand side would simplify to become the
left-hand side. To find $A$ and $B$ we actually solve the
{\it polynomial} equality (\ref{PolyEqualityForPFD1})
for $A$ and $B$.

Note also the distinction between ``equations'' such as $2x-1=5$,
which is true only for $x=3$, and ``equalities'' such as
$(x+1)^2=x^2+2x+1$, true for all $x$, meaning the function
$(x+1)^2$ is the same as the function $x^2+2x+1$.
%%%
%%% END FOOTNOTE
}

The second method for finding $A$ and $B$ (not preferred here
but not terribly difficult here either) is to look at the
coefficients of the polynomials on the left-hand side and
right-hand side of (\ref{PolyEqualityForPFD1}),
and realize that the coefficients of the
various powers of $x$ must agree for these to be
the {\it same} polynomials.  Though perhaps not necessary for
this simple case, one sometimes expands the right-hand side
and collects like terms 
$$1=(A+B)x+(-3A-2B).$$
From this or just reading from (\ref{PolyEqualityForPFD1}),
we can in turn set equal the coefficients of the $x^1$ terms
and the constant (some
say $x^0$) terms to get the following system of two
equations in two unknowns:
\begin{equation}\left\{\begin{array}{rcrcr}
 0&=&A&+&B\hphantom{.}\\
 1&=&-3A&-&2B.\end{array}\right.\label{PolyEqualityForCompCoeffForPFD1}
\end{equation}
The first equation came from the fact that there is no $x^1$-term
on the left-hand side of (\ref{PolyEqualityForCompCoeffForPFD1}),
or alternatively, the $x^1$-term is $0x^1$ on the left.
To solve such a system one might add three times the
first equation to the second, to get $B=1$, and
use that information in the first to get $A=-1$, as before.

Whenever the denominator of our function $P(x)/Q(x)$ has
a linear factor $(ax+b)$, evaluating the associated {\it polynomial}
equality---such as (\ref{PolyEqualityForPFD1})---at that
$x$-value which makes this linear factor zero (namely 
$x=-b/a$) will
quickly yield one of the coefficients, since all but one
term in the polynomial equation will have $(ax+b)$ as a factor,
and therefore vanish at $x=-b/a$.
Thus this first method should always be employed to find that
coefficient if the
denominator $Q$ has a linear factor.  If the denominator 
has all linear terms to the first power, then this ``evaluation''
method will quickly yield all coefficients.

\bex Compute $\ds{\int\frac{2x^2-3x+2}{x(x+5)(2x+1)}\,dx}$.

\underline{Solution} It is important to notice that the numerator
is degree 2, and the denominator degree 3, so the PFD rules do apply.
$$\int\frac{2x^2-3x+2}{x(x+5)(2x+1)}\,dx
 =\int\left[\frac{A}x+\frac{B}{x+5}+\frac{C}{2x+1}\right]\,dx.$$
Eventually we will cease writing the next two lines, but to be
sure we will include them here so that the logic is clear:
\begin{alignat*}{2}
&&\frac{2x^2-3x+2}{x(x+5)(2x+1)}&=\frac{A}x+\frac{B}{x+5}+\frac{C}{2x+1}
   \\
&\implies
  &x(x+5)(2x+1)\left[\frac{2x^2-3x+2}{x(x+5)(2x+1)}\right]
    &=x(x+5)(2x+1)\left[\frac{A}x+\frac{B}{x+5}+\frac{C}{2x+1}\right]\\ \\
&\implies&2x^2-3x+2&=A(x+5)(2x+1)+Bx(2x+1)+Cx(x+5).
\end{alignat*}
Into this last line we can now enter values for $x$ which will
quickly yield the coefficients.

\begin{alignat*}{2}
\underline{x=0}:\qquad&& 2&=A(5)(1)\implies\boxed{A=\frac25}\\
\underline{x=-5}:\qquad&& 20+15+2&=B(-5)(-9)\\
              &&\implies 37&=45B\implies\boxed{B=\frac{37}{45}}.\\
\underline{x=-\frac12}:\qquad&&
              2\cdot\frac14-3\cdot\left(-\frac12\right)+2
                &=C\left(-\frac12\right)\left(-\frac12+5\right)\\
              &&\implies \frac12+\frac32+2&=C\left(-\frac12\right)
                    \left(\frac92\right)\\
              &&\implies4&=-\frac94C\implies
                       \boxed{C=-\frac{16}9}.
              \end{alignat*}
Putting this together with our original integral, we get
\begin{align*}
\int\frac{2x^2-3x+2}{x(x+5)(2x+1)}\,dx
&=\int\left[\frac{2/5}x+\frac{37/45}{x+5}+\frac{-16/9}{2x+1}\right]\,dx\\
&=\frac25\ln|x|+\frac{37}{45}\ln|x+5|-\frac{16}9\cdot\frac12\ln|2x+1|+C\\
&=\frac25\ln|x|+\frac{37}{45}\ln|x+5|-\frac89\ln|2x+1|+C.
\end{align*}
\eex
Next we look at an example where all factors of $Q(x)$ are linear, 
but one of these linear factors appears to the second power.

\bex Compute $\ds{\int\frac{x+1}{x^2(x-5)(x+4)}\,dx}$.

\underline{Solution}: This time we will describe but
omit the explicit multiplication
step in the PFD.\footnote{%%%
%%% FOOTNOTE
The pattern of cancellation, when we multiply the PFD by the original
denominator $Q(x)$, should become second nature with a small amount
of practice. That said, it is important to remember what we are doing
(multiplying by $Q(x)$) to get from the PFD to the polynomial equality,
and how the various factors cancel (or do not cancel) in that multiplication.
%%% END FOOTNOTE
}
$$\int\frac{x+1}{x^2(x-5)(x+4)}\,dx
  =\int\left[\frac{A}x+\frac{B}{x^2}+\frac{C}{x-5}+\frac{D}{x+4}\right]\,dx,$$
where 
$$\frac{x+1}{x^2(x-5)(x+4)}
=\frac{A}x+\frac{B}{x^2}+\frac{C}{x-5}+\frac{D}{x+4}.$$
Multiplying by $x^2(x-5)(x+4)$ then gives us
\begin{equation}x+1=Ax(x-5)(x+4)+B(x-5)(x+4)+Cx^2(x+4)+Dx^2(x-5).
  \label{PolyEqualityForPFD3}\end{equation}
With this equation, choosing $x=0,5,-4$ will yield three of the
four coefficients.
\begin{alignat*}{2}
\underline{x=0}:&\qquad&1&=B(-5)(4)\implies\boxed{B=-\frac1{20}}\\
\underline{x=5}:&&6&=C(5^2)(9)\\
          &&\implies\frac{2\cdot3}{5^2\cdot3\cdot3}
                   &=C\implies\boxed{C=\frac2{75}}\\
\underline{x=-4}:&\qquad&-3&=D((-4)^2)(-9)\\
              &&\implies-3&=D[-16\cdot9]\\
              &&\implies 3&=D\cdot16\cdot3\cdot3\implies
                     \boxed{D=\frac1{48}} 
\end{alignat*}
This exhausts the evaluations which give equations in one coefficient.
Next we have several methods for finding $A$.
\begin{description}
\item[Method 1.] Compare coefficients.  In particular, we look at the
highest-order $x$-terms which appear---at least initially---in 
the polynomial equality,
which for (\ref{PolyEqualityForPFD3}) means the $x^3$-terms.
Here we have no $x^3$-terms on the left, and on the right,
even without a complete expansion, we can see that 
the $x^3$-terms will be $A+C+D$.  (The middle-order terms are more
difficult to read from (\ref{PolyEqualityForPFD3}).)  Fortunately
we already know the values of $C$ and $D$, so we have enough
information to find $A$:
\begin{alignat*}{2}
\underline{x^3\text{-term}}:&\qquad&0&=A+C+D\\
                      &&\implies0&=A+\frac2{75}+\frac1{48}
                                  =A+\frac2{3\cdot5^2}+\frac1{2^4\cdot3}\\
                      &&\iff0&=A+\frac{2\cdot2^4+1\cdot5^2}{3\cdot5^2\cdot2^4}
                              =A+\frac{32+25}{3\cdot5^2\cdot2^4}\\
                      &&\iff-\frac{57}{1200}&=A\implies
                                \boxed{A=-\frac{19}{400}}
\end{alignat*}

From this we can complete the integration:
\begin{align*}\int\frac{x+1}{x^2(x-5)(x+4)}\,dx
&=\int\left[\frac{-\frac{19}{400}}{x}+\frac{-\frac1{20}}{x^2}
     +\frac{\frac2{75}}{x-5}+\frac{\frac1{48}}{x+4}\right]\,dx\\
&=-\frac{19}{400}\ln|x|-\frac1{20}\cdot\frac{-1}{x}+\frac2{75}\ln|x-5|
       +\frac1{48}\ln|x+4|+C.
\end{align*}
\item[Method 2.] One can instead evaluate the polynomial 
equality (\ref{PolyEqualityForPFD3}) at still another $x$-value,
though no such value will produce $A$ alone:
$$\underline{x=1}:\qquad 2=A(1)(-4)(5)
                     +B(-4)(5)+C(1)^2(5)+D(1)^2(-4).$$
Since we already know $B$, $C$ and $D$, we can insert that information
and solve for $A$. 
\item[Method 3.] This will be more useful later, but this method
(referred to earlier as the {\rm auxiliary} method) certainly applies.
The idea is that we apply $\frac{d}{dx}$ to both sides
of (\ref{PolyEqualityForPFD3}), which is valid because the left-hand
side and right-hand side of (\ref{PolyEqualityForPFD3})
are the same {\rm functions}.

In order to use this method, it is useful to recall the generalized
product rule.  For three functions $u(x)$, $v(x)$ and $w(x)$, for instance,
we have
$$(uvw)'=u'vw+uv'w+uvw'.$$
For reference we recall (\ref{PolyEqualityForPFD3}), from which we then
compute the derivatives. Equation (\ref{PolyEqualityForPFD3}) reads:
$$x+1=Ax(x-5)(x+4)+B(x-5)(x+4)+Cx^2(x+4)+Dx^2(x-5).$$
\begin{align*}
\underline{\frac{d}{dx}}:\qquad\qquad1&=A[(1)(x-5)(x+4)+x(1)(x+4)+x(x-5)(1)]\\
  &\qquad+B[(1)(x+4)+(x-5)(1)]\\
  &\qquad\qquad+C[(2x)(x+4)+x^2(1)]\\
  &\qquad\qquad\qquad+D[(2x)(x-5)+x^2(1)].\end{align*}
Now when we evaluate this at $x=0$, we see that we get
$$1=A[(1)(-5)(4)]+B[(1)(4)+(-5)(1)]+0+0.$$
This gives us $1=-20A-B$, so then $A=(1+B)/(-20)=(19/20)/(-20)=-19/400$,
as before.
\end{description}
\eex

A simple principle buried in the third method is the following:
\begin{theorem}
If $(x-a)^m$, where $m>1$ is a factor of a polynomial $f(x)$,
             then $(x-a)^{m-1}$ is a factor of $f'(x)$.
\label{DerivLowersPowerOf(X-A)Theorem} 
\end{theorem}
For a proof, we note that $(x-a)^m$ being a factor of $f(x)$
is equivalent to $f(x)=(x-a)^mg(x)$, where $g(x)$ is another 
polynomial.  Thus
$$f'(x)=(x-a)^mg'(x)+m(x-a)^{m-1}(1)g(x)
=(x-a)^{m-1}\underbrace{\left[(x-a)g'(x)+mg(x)\right]}_{\text{polynomial}},$$
so indeed $(x-a)^{m-1}$ is a factor of $f'(x)$.


Now evaluating both sides of (\ref{PolyEqualityForPFD3}) at $x=0$
caused  those terms with $x$ and $x^2$ factors to vanish,
leaving an equation with $B$ only.  When we differentiate 
(\ref{PolyEqualityForPFD3}), those terms with $x^2$ factors
{\it still vanish}---because one power of $x$ remains---leaving 
only the $B$-term (as before) 
and the $A$-term (which had a factor $x$ but not $x^2$).
Already knowing $B$, we could solve for $A$.


A few guidelines for efficiently finding the PFD coefficients
should be made at this point.

\begin{enumerate}
\item When linear factors are present in $Q(x)$, it is best
      to exhaust this method for finding some of the coefficients
      easily. This means evaluating the relevant polynomial equality
      at each value for which $Q(x)=0$.
\item When those values are exhausted, we should next 
      compare coefficients of the
      powers of $x$, particuarly the highest power which occurs
      on the right-hand side.
\item If $(ax+b)^m$ is a factor of $Q(x)$, where $m>1$, 
      and the first two methods fail to get all coefficients,
      then differentiation of the polynomial equality may
      yield more coefficients.
\item If there are still coefficients to be found, then further
      evaluations, differentiations, or coefficient comparisons
      should be implemented.
\end{enumerate}
\bex Compute $\ds{\int\frac{5x^3-17x^2+19x-13}{(x+1)(x-2)^3}\,dx}$.

\underline{Solution}: As usual we start with the PFD.
$$\frac{5x^3-17x^2+19x-13}{(x+1)(x-2)^3}
=\frac{A}{x+1}+\frac{B}{x-2}+\frac{C}{(x-2)^2}+\frac{D}{(x-2)^3}$$
\begin{equation}
5x^3-17x^2+19x-13
 =A(x-2)^3+B(x+1)(x-2)^2+C(x+1)(x-2)+D(x+1)
\label{PolyEqualityForPFD4}\end{equation}
\begin{alignat*}{2}
\underline{x=-1}:&\qquad&-5-17-19-13&=A(-27)\\
                 &&\implies-54&=-27A\implies\boxed{A=2}\\
\underline{x=2}:&\qquad&5(8)-17(4)+19(2)-13&=D(3)\\
                &&\implies-3&=3D\implies\boxed{D=-1}\\
\underline{x^3\text{-term}:}&&5&=A+B\\
                &&\implies 5&=2+B\implies\boxed{B=3}.
\end{alignat*}
While we could perform another evaluation ($x=0$ comes to mind),
or look at another coefficient (prone to error), instead we will
differentiate (\ref{PolyEqualityForPFD4}):
\begin{align*}
15x^2-34x+19&=A[3(x-2)^2(1)]\\
             &\qquad+B[(1)(x-2)^2+(x+1)\cdot2(x-2)(1)]\\
             &\qquad\qquad+C[(1)(x-2)+(x+1)(1)]\\
             &\qquad\qquad\qquad+D(1).\end{align*}
Now we evaluated at $x=2$:
\begin{align*}
15(4)-34(2)+19&=C(3)+D\\\implies 11&=3C-1\implies \boxed{C=4}.
\end{align*}
Thus
\begin{align*}
\int\frac{5x^3-17x^2+19x-13}{(x+1)(x-2)^3}\,dx
&=
\int\left[\frac2{x+1}+\frac3{x-2}+\frac4{(x-2)^2}-\frac1{(x-2)^3}\right]\,dx\\
&=2\ln|x+1|+3\ln|x-2|-\frac4{x-2}-\frac1{-2}\cdot\frac1{(x-2)^2}+C\\
&=\ln\left|(x+1)^2(x-2)^3\right|-\frac4{x-2}+\frac1{2(x-2)^2}+C.
\end{align*}


\eex

A quick corollary to our Theorem~\ref{DerivLowersPowerOf(X-A)Theorem} 
is that if
$(x-a)^m$ is a factor of a polynomial $f(x)$, then for $k<m$ we have
$(x-a)^{m-k}$ is a factor of $f^{(k)}(x)=\frac{d^k}{dx^k}f(x)$.
This follows from repeated applications of the theorem, which 
can be paraphrased as saying that we lose at most one factor of $(x-a)$
for each derivative we take, until we run out of factors of
$(x-a)$.  If our latest example had $(x-2)^4$ in the denominator,
we could have taken a second derivative of the corresponding
polynomial equation, and then those terms with 
$(x-2)^3$ or $(x-2)^4$ will stil be zero at $x=2$, but the 
other terms would likely be nonzero.\footnote{%%%
%%% FOOTNOTE
Note that it is quite possible that $x-a$ is not a factor of 
a polynomial $f(x)$, but {\it is} a factor of $f'(x)$.  
That is the case when $x=a$ is a {\it critical point} of
$f(x)$. For example, $x-1$ is not a factor of $f(x)=x^2-2x+21$,
but is a factor of $f'(x)=2x-2=2(x-1)$.

Note also that the theorem applies to any linear factor $ax+b$,
where $a\ne0$, since $ax+b=a\left(x+\frac{b}a\right)$.
Thus if $x^m$ is a factor of a polynomial $f(x)$, 
the $x^{m-1}$ is a factor of $f'(x)$, etc., as is the 
case if we replace $x^m$ with $(ax+b)^m=a^m\left(x+\frac{b}a\right)$.
%%% END FOOTNOTE
}


Now we turn our attention to PFD's where the denominators contain
irreducible quadratic factors.\footnote{%%
%%% FOOTNOTE
 Until the next section, we will
not be able to integrate the general case where we have an 
irreducible quadratic factor to a power greater than 1,
with some exceptional cases.  
%%% END FOONTOTE
}
One problem with such factors is that they are nonzero for 
any $x\in\Re$,\footnote{%%%
%%% FOOTNOTE
Recall that if $f(x)$ is a polynomial of degree $\ge1$,
then $f(a)=0\iff (x-a)$ is a factor of $f(x)$.
%%% END FOOTNOTE
}
so the evaluation method's usefulness is limited in these cases.
For such PFD's, we will need to rely more upon the coefficient
comparison method to find our coefficients.

\bex Compute $\ds{\int\frac{4x^3-7x^2+31x-38}{x^4+13x^2+36}\,dx}$.

\underline{Solution}:
PFD rules apply since the degree of the numerator is less than that
of the denominator.
We need to begin by factoring the denominator of the integrand,
after which we can write the general form of the PFD.
$$
\int\frac{4x^3-7x^2+31x-38}{x^4+13x^2+36}\,dx
=\int\frac{4x^3-7x^2+31x-38}{(x^2+4)(x^2+9)}\,dx
=\int\left[\frac{Ax+B}{x^2+4}+\frac{Cx+D}{x^2+9}\right]\,dx.$$
Now taking the second equation, we underlying PFD becomes
the polynomial equality
\begin{equation}4x^3-7x^2+31x-38=(Ax+B)(x^2+9)+(Cx+D)(x^2+4).
\label{PolyEqualityForPFD5}
\end{equation}
Now we look at the coefficients.\footnote{%%%
%%% FOOTNOTE
Note that the constant (``$x^0$'') term equation is what we 
would get if we evaluated (\ref{PolyEqualityForPFD5}) at $x=0$.
It is easy to see that this is always the case.%%%
%%% END FOOTNOTE
}
\begin{alignat*}{2}
\underline{x^3\text{-term}}:&&\qquad 4&=A+C\\
\underline{x^2\text{-term}}:&&-7&=B+D\\
\underline{x^1\text{-term}}:&&31&=9A+4C\\
\underline{x^0\text{-term}}:&&-38&=9B+4D.
\end{alignat*}
Though this looks like (and is) four equations in four
unknowns, in fact it ``decouples'' into two systems,
each with two unknowns, since the first and third
equations have only $A$ and $C$, and the second and fourth have
$B$ and $D$ only.  We solve these in turn.
$$\begin{array}{rcrcr}
4&=&A&+&C\\
31&=&9A&+&4C\end{array}
\qquad\qquad
\begin{array}{rcrcr}
-7&=&B&+&D\\
-38&=&9B&+&4D\end{array}$$
For the first system, we multiply the first equation by $-9$ and add
to the second, to get $-5=0A-5C\implies C=1$. From that we have
the original first equation giving $A=4-C=4-1=3$.

For the second system, we do the same, that is, mutliply the first
equation by $-9$ and add to the second, giving $63-38=-5D
\implies25=-5D\implies-5=D$.  From the original first equation
in that system, we then get $B=-7-D=-7+5=-2$.

Now we compute the integral, noting that it is easier if we
break the PFD into four distinct terms:
\begin{align*}
\int\frac{4x^3-7x^2+31x-38}{(x^2+4)(x^2+9)}\,dx
&=\int\left[\frac{3x}{x^2+4}-\frac{2}{x^2+4}
              +\frac{x}{x^2+9}-\frac5{x^2+9}\right]\,dx\\
&=\frac32\ln(x^2+4)-\frac22\tan^{-1}\frac{x}2+\frac12\ln(x^2+9)
        -\frac53\tan^{-1}\frac{x}3+C\\
&=\ln\sqrt{(x^2+4)^3(x^2+9)}-\tan^{-1}\frac{x}2
         -\frac53\tan^{-1}\frac{x}3+C.
\end{align*}
\eex

In the example above, we used the following common integration formula,
which is particularly useful in problems encountered in this section.
It is derivable with the usual substitution methods, and not too difficult
to verify by differentiation.  The formula is the following:
\begin{equation}
\int\frac1{x^2+a^2}\,dx=\frac1a\tan^{-1}\frac{x}a+C.
\label{Int1/(x^2+a^2)}\end{equation}
We also used 
$$\int\frac{x}{x^2+k^2}\,dx=\frac12\ln(x^2+k^2)+C,$$
assuming $k\ne0$.  Note that we do not need absolute values
inside the logarithm since $x^2+k^2\ge k^2>0$.

When we have irreducible quadratic factors in the denominator
$Q(x)$, it is likely that we will need to compare coefficients.\footnote{%%%
%%% FOOTNOTE
Or something equivalent to comparing coefficients.  For instance,
$x=0$ gives $-38=9B+4D$, and one derivative of (\ref{PolyEqualityForPFD5})
gives 
$$12x^2-14x+31=(A)(x^2+9)+(Ax+B)(2x)+(C)(x^2+4)+(Cx+D)(2x),$$
which, when we consider the datum $x=0$ gives
$31=9A+4C$.  Both of these we had before.  More derivatives,
evaluated at $x=0$,  give
multiples of the other two equations in our system (four equations in four
unknowns).%%
%%% END FOOTNOTE
}
Afterall, there are no real numbers which will make all but
one of those coefficients vanish.  (We can make two vanish with $x=0$,
but that still leaves two.)  If linear terms are also present, however,
the evaluation method will yield one or more of the coefficients quickly.

\bex Compute $\ds{\int\frac{12x^4+190x^2+13x-6}{(2x-1)(x^2+16)}\,dx}$.

\underline{Solution}: First we note that the numerator has degree
which is not less than the denominator, so we must use long division.
To do so we need to expand the denominator:
$(2x-1)(x^2+16)=2x^3-x^2+32x-16$.

Now through polynomial long division we get
\begin{equation}
\frac{12x^4+190x^2+13x-6}{(2x-1)(x^2+16)}
 =\frac{12x^4+190x^2+13x-6}{2x^3-x^2+32x-16}
 =6x+3+\frac{x^2+13x+42}{2x^3-x^2+32x-16}.\end{equation}
Refactoring our denominator, our integral now becomes
$$\int\left[6x+3+\frac{x^2+13x+42}{(2x-1)(x^2+16)}\right]\,dx
=\int\left[6x+3+\frac{A}{2x-1}+\frac{Bx+C}{x^2+16}\right]\,dx.$$
The first two terms are easy enough.  For our PFD, we need only
concern ourselves with the remaining fraction:
$$\frac{x^2+13x+42}{(2x-1)(x^2+16)}=
\frac{A}{2x-1}+\frac{Bx+C}{x^2+16}.$$
The corresponding polynomial equation is then
\begin{equation}
x^2+13x+42=A(x^2+16)+(Bx+C)(2x-1).
\label{PolyEqualityForPFD6}\end{equation}
We begin with an evalutation, followed by a coefficient comparison.
\begin{alignat*}{2}
\underline{x=\frac12}:&\qquad&\frac14+\frac{13}2+42&=
              A\left(\frac14+16\right)\\
       &&\implies \frac{1+26+168}{4}&=\frac{65}4A\\
       &&\implies {195}&=65A\implies\boxed{A=3}\\
\underline{x^2\text{-term}}:
           &&1&=A+2B\\
           &&\implies1&=3+2B\implies\boxed{B=-1}.
\end{alignat*}
Perhaps the simplest next step is to find $C$ by evaluation of
(\ref{PolyEqualityForPFD6}) at,
say, $x=0$:
\begin{align*}
\underline{x=0}:\qquad 42&=16A-C\\
                  \implies42&=16(3)-C\\
                  \implies C&=16(3)-33=48-16\implies \boxed{C=6}.
\end{align*}
Thus our original integral, including the polynomial terms, becomes
\begin{align*}
\int\frac{12x^4+190x^2+13x-6}{(2x-1)(x^2+16)}\,dx
&=\int\left[6x+3+\frac{3}{2x-1}-\frac{x}{x^2+16}+\frac{6}{x^2+16}
  \right]\,dx\\
&=3x^2+3x+\frac32\ln|2x-1|-\frac12\ln(x^2+16)+\frac64\tan^{-1}\frac{x}4
+C\\&=3x(x+1)+3\ln\sqrt{|2x-1|}-
\ln\sqrt{x^2+16}+\frac32\tan^{-1}\frac{x}4
+C.\end{align*}
The second from the last line was complete; the last line just gives
some alternative styles for the particular terms.
\eex

Of course with any new technique, we have to be sure that
we do not neglect the earlier methods.


\section{Miscellaneous Methods}

In this section we will use completing the square, and other
methods to rewrite several types of integrals into forms
where we can more easily use either partial fractions
or trigonometric substitution.  We will also look at examples
where a substitution will bring us to such forms.  Finally, we
will consider the use of integration tables, which 
can be found in numerous publications, but which require
some sophistication to be used properly.







\newpage


