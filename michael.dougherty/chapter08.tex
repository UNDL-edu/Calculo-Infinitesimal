%\setcounter{page}{900}

\chapter{Applications of Definite Integrals I: General Arguments
\label{AppsDefIntegrals}}

Here we look at many further quantities which give rise to 
antidifferentiation.  




Most modern calculus textbooks contain numerous excellent examples
of applications of definite integrals.  Indeed, most examples
which are likely to be seen in further studies of the physical
sciences can trace back to calculus textbook-type problems.

Here we will make an attempt to accomplish the presentations
of the usual topics, and others.  We will also 
do the following:

\begin{enumerate}
\item Re-introduce the notion of infinitesimals into the
physical analysis of these problems.  This was a traditional
approach which has fallen out of favor.  While we will refer
back to the Riemann sums for each case, that will be more
for a ``spot-check'' of the reasoning behind the infinitesimals,
and perhaps some proofs.  However, the first introduction to 
most topics will be through infinitesimals.  Besides, they
make for prettier pictures!

Students are unlikely to be inspired by the Riemann sum proofs,
which are often used to show that the quantity represented
by the integral has the integrand as derivative.  The proofs
are technical, and leave a student to believe he would never
come up with it himself.  The differentials ``cut to the chase,''
and can be proved later after guessed, rather than derived from 
Riemann sums.

\item Finite Riemann sums used for numerical approximations of
the quantities involved.  This contains all the intuition
(short of the proofs) contained in the most textbook developments
of these integrals.

\item Explanations of why some guesses for the infinitesimals
will not work.

\item Explanation of how to tell---by sight---if a particular
differential is correct, and whether the Riemann sum form
will actually converge to the desired quantity as the partition
is refined.
\end{enumerate}





%Recall that we already gave one argument for two situations where
%a given quantity can be represented by a definite integral:
%\begin{align}
%   s\left(t_f\right)-s\left(t_0\right)
%        &=\int_{t_0}^{t_f}a(t)\,dt,\label{S(T_F)-S(T_O)ForDefIntI}\\
%   v\left(t_0\right)-v\left(t_0\right)
%        &=\int_{t_0}^{t_f}a(t)\,dt.\label{V(T_F)-V(T_O)ForDefIntI}
%\end{align}
%This was just the working definition
%of the definite integral, 
%\begin{equation}\int_a^bf(x)\,dx=F(b)-F(a),\label{FundTheoremAsWorkingDef}
%\end{equation} 
%assuming $f(x)$ is continuous on $[a,b]$ and $F'(x)=f(x)$ on $[a,b]$.
%As a matter of fact, (\ref{FundTheoremAsWorkingDef}) is not the actual
%definition of the definite integral $\int_a^bf(x)\,dx$.  Indeed,
%the definition is more complicated.  However, it is quite useful.


\section{Riemann Sums and Approximations of Cumulative Quantities}

Suppose we had some data on the velocity of an object, and
we wish to approximate its net displacement over a time interval.
Suppose the data we have is the following:

%\begin{tabular}{m{.4in}|m{.3in}|m{.3in}|m{.3in}|m{.3in}|m{.3in}|%
%m{.3in}|m{.3in}|m{.3in}|m{.3in}|m{.3in}|}
%\cg $t=$&\cg0&\cg1&\cg2&\cg2.5&\cg3&\cg4&\cg6&\cg7&\cg8.5&\cg10%\\
%\cg $v=$&\cg10.9&\cg9.6&\cg8.9&\cg8.775&\cg8.8&\cg9.3&\cg12.1&\cg14.4
%               &\cg18.975&\cg24.9
%\end{tabular}


\section{Other Complications}
Consider cases where the function is only piecewise continuous,
and try to recover a continuous antiderivative.
Consider, perhaps, some cases from electricity and magenetism
regarding fields or potentials across interfaces.

Consider two ``antiderivatives'' of 
$$f(x)=\left\{\begin{aligned}1,\qquad&x>0,\\
                           -1,\qquad x<0.\end{aligned}\right.
$$
One is continuous, the other not.

Can always recover a continuous antiderivative from a piecewise continuous
function, so long as we don't have vertical asymptotes, for instance.
