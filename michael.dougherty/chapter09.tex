%\setcounter{page}{1000}
\chapter{Improper Integrals and Advanced Limit Techniques
\label{ImpropIntsChapter}}
In this chapter we will develop some more advanced techniques for
computing limits.  In the first section, we will look at many
nonindeterminate forms.  In the next section we compute many limits 
of indeterminate forms $0/0$ or $\infty/\infty$ using a very 
useful---but not universal---technique known as L\^opital's Rule.
Next we look at other indeterminate forms, revisiting
$\infty-\infty$ and $0\cdot\infty$, and
introducing for the first time  $(0^+)^0$,
$\infty^0$, and $(1^+)^\infty$ using a simple twist on the  
L\^opital's Rule.\footnote{%%%
%%% FOOTNOTE
Other texts are usually not so refined, instead calling these forms
$0^0$, $\infty^0$ (as we do), and $1^{\infty}$.  With the technique
we will develop, it will in fact not be necessary to notice the
form sinc the first steps of the method do not rely upon the
form at all.
%%% END FOOTNOTE
} Finally we apply these and previous methods
to so-called {\it improper integrals}, meaning those for 
which we relax the rules that the integrand must be a continuous
function $f(x)$ on a closed and bounded interval $[a,b]$. 
Examples of useful integrals which break these rules include
$$\int_{-2}^2\frac1{x^{2/3}}\,dx,\qquad
\int_0^1\ln x\,dx,\qquad
\int_1^\infty\frac1{x^2}\,dx,\qquad
\int_{-\infty}^\infty\frac1{x^2+1}\,dx,
$$
and others.  In all of the above, either the function is not
continuous in the entire range $[a,b]$ of values
for $x$, or those ``limits of integration'' $a$ and $b$ are
not finite.  We will eventually make sense of such integrals by 
intuitive approaches which will require all of our previous limit
methods 

\section{Some Asymptotics of Functions}

It will be important to be able to spot the ``forms'' of the limits 
we will encounter in this chapter and beyond.  This was also the case in
Chapter~\ref{LimitsAndContinuityChapter}, but we have encountered
many more functions since then.  

In this section we first take note of the behaviors of functions
near vertical asymptotes, and as $|x|\to\infty$.  This behavior
we will collectively call the {\it asymptotics} of a given
function.  In our first 
subsection we will look at particular functions.  Then we will
look at compositions and combinations of these functions, in
cases where we can read the behavior of these more complicated
functions as following from the behaviors of the underlying functions.

\begin{figure}
\begin{center}

\begin{pspicture}(-3.5,-2)(3.5,4.5)
\psaxes{<->}(0,0)(-3.5,-1)(3.5,4.5)
\psplot[plotpoints=1000]{-3.5}{1.5}{2.718281828 x exp}
\rput(0,-1.3){Figure \ref{GrOfFun4AsyBeh}.a: Graph of $y=e^x$}
\end{pspicture}
\hfill
\begin{pspicture}(-3.5,-4.5)(3.5,2)
\psaxes{<->}(0,0)(-3.5,-3.5)(3.5,2)
\psplot[plotpoints=1000]{.030197383}{3.5}{x log 0.434294482 div}
\rput(0,-3.8){Figure \ref{GrOfFun4AsyBeh}.b: Graph of $y=\ln x$}
\end{pspicture}
\vfill


\begin{pspicture}(-3.5,-3.5)(3.5,3)
\psset{xunit=1.5cm,yunit=1.5cm}
\psaxes[Dy=10]{<->}(0,0)(-2,-2)(2,2)
\parametricplot[plotpoints=1000]{-1.570796327}{1.570796327}%
{t 3.1415926536 div 180 mul sin t}
\psline(-.1,1.570796)(.1,1.570796)
  \rput(-.4,1.570796){$\pi/2$}
\psline(-.1,-1.570796)(.1,-1.570796)
  \rput(-.5,-1.570796){$-\pi/2$}
\pscircle*(-1,-1.570796){.05}
\pscircle*(1,1.570796){.05}

\rput(0,-2.2){Figure \ref{GrOfFun4AsyBeh}.c: Graph of $y=\sin^{-1}x$}
\end{pspicture}
\hfill
\begin{pspicture}(-3.5,-1)(3.5,5.5)
\psset{xunit=1.5cm,yunit=1.5cm}
\psaxes[Dy=10]{<->}(0,0)(-2,-.3334)(2,3.66666666666667)
\parametricplot[plotpoints=1000]{0}{3.141592654}%
{t 3.1415926536 div 180 mul cos t}
\psline(-.1,1.570796)(.1,1.570796)
  \rput(-.4,1.570796){$\pi/2$}
\psline(-.1,3.1415926535)(.1,3.1415926535)
  \rput(-.3,3.1415926535){$\pi$}
\pscircle*(1,0){.05}
\pscircle*(-1,3.1415926535){.05}

\rput(0,-.5344)%
{Figure \ref{GrOfFun4AsyBeh}.d: Graph of $y=\cos^{-1}x$}
\end{pspicture}


\vfill

\begin{pspicture}(-3.5,-3)(3.5,3)
\psset{xunit=.5cm,yunit=.75}
\psaxes[Dx=10,Dy=10]{<->}(0,0)(-7,-3)(7,3)
\parametricplot[plotpoints=1000]{-1.428899272}{1.428899272}%
{t 3.1415926536 div 180 mul sin %
t 3.1415926536 div 180 mul cos div t}
\psline(-.2,1.570796)(.2,1.570796)
  \rput(3.5,1.8){$y=\pi/2$}
\psline(-.2,-1.570796)(.2,-1.570796)
  \rput(-3.5,-2){$y=-\pi/2$}

\rput(0,-3.5){Figure \ref{GrOfFun4AsyBeh}.e: Graph of $y=\tan^{-1}x$}

\psline[linestyle=dashed](-7,-1.570796)(7,-1.570796)
\psline[linestyle=dashed](-7,1.570796)(7,1.570796)


\end{pspicture}
\hfill
\begin{pspicture}(-3.5,-1.5)(3.5,4.5)
\psset{xunit=.5cm,yunit=.75}
\psaxes[Dx=10,Dy=10]{<->}(0,0)(-7,-1)(7,5)
\parametricplot[plotpoints=1000]{0}{1.428899272}%
{1 t 3.1415926536 div 180 mul cos div  t}
\parametricplot[plotpoints=1000]{1.714143896}{3.1415926535}%
{1 t 3.1415926536 div 180 mul cos div  t}


\psline(-.2,1.570796)(.2,1.570796)
  \rput(3.5,1.8){$y=\pi/2$}


\rput(0,-1.5){Figure \ref{GrOfFun4AsyBeh}.f: Graph of $y=\sec^{-1}x$}
\psline[linestyle=dashed](-7,1.570796)(7,1.570796)

\psline(-1,-.2)(-1,.2)
  \rput(-1,-.8){$-1$}
  \pscircle*(-1,3.1415926535){.05}
\psline(1,-.2)(1,.2)
  \rput(1,-.8){$1$}
  \pscircle*(1,0){.05}
\psline(-.3,3.1415926353)(.3,3.1415926353)
   \rput(.5,3.1415926535){$\pi$}

\end{pspicture}
\end{center}
\vfill

\caption{Some graphs of important functions, illustrating 
         asymptotic behaviors.}
\label{GrOfFun4AsyBeh}\end{figure}

The functions in Figure~\ref{GrOfFun4AsyBeh} have all been 
encountered earlier in the text.  Taking these functions in turn, we 
now list  of their asymptotic and other limiting
behaviors, which we can read off of the
graphs.  
\newpage
\begin{enumerate} 
\item $e^x$:
   \begin{enumerate}
   \item $x\to\infty\implies e^x\to\infty$
   \item $x\to-\infty\implies e^x\to0^+$
   \end{enumerate}
\item $\ln x$ (note the relationship between this and $e^x$, 
   which is the inverse of $\ln x$):
   \begin{enumerate}
   \item $x\to\infty\implies\ln x\to\infty$
   \item $x\to0^+\implies\ln x\to-\infty$
   \end{enumerate}
\item $\sin^{-1}x$:
   \begin{enumerate}
   \item $x\to-1^+\implies\sin^{-1}x\to\left(-\frac{\pi}{2}\right)^+$
   \item $x\to1^-\implies\sin^{-1}x\to\left(\frac{\pi}{2}\right)^-$
   \end{enumerate}
\item $\cos^{-1}x$:
   \begin{enumerate}
   \item $x\to-1^+\implies\cos^{-1}x\to\pi^-$
   \item $x\to1^-\implies\cos^{-1}x\to0^+$
   \end{enumerate}
\item $\tan^{-1}x$:
   \begin{enumerate}
   \item $x\to-\infty\implies\tan^{-1}x\to\left(-\frac{\pi}2\right)^+$
   \item $x\to\infty\implies\tan^{-1}x\to\left(\frac{\pi}2\right)^-$
   \end{enumerate}
\item $\sec^{-1}x$:
   \begin{enumerate}
   \item $x\to-\infty\implies\sec^{-1}x\to\left(\frac{\pi}2\right)^+$
   \item $x\to\infty\implies\sec^{-1}x\to\left(\frac{\pi}2\right)^-$
   \end{enumerate}
\end{enumerate}
%\newpage

%\begin{align}
%x\to\infty&\implies e^x\to\infty\\
%x\to-\infty&\implies e^x\to0^+\\
%\notag\\
%x\to\infty&\implies\ln x\to\infty\\
%x\to0^+&\implies\ln x\to-\infty\\
%\notag\\
%x\to1^-&\implies\sin^{-1}x\to\left(\frac{\pi}2\right)^-\\
%x\to-1^+&\implies\sin^{-1}x\to\left(-\frac{\pi}2\right)^+\\
%\notag\\
%x\to1^-&\implies\cos^{-1}x\to\pi^-\\
%x\to-1^+&\implies\cos^{-1}x\to0^+\\
%\notag\\
%x\to\infty&\implies\tan^{-1}x\to\left(\frac{\pi}{2}\right)^-\\
%x\to-\infty&\implies\tan^{-1}x\to\left(-\frac{\pi}2\right)^+\\
%\notag\\
%x\to\infty&\implies\sec^{-1}x\to\left(\frac{\pi}2\right)^-\\
%x\to-\infty&\implies\sec^{-1}x\to\left(\frac{\pi}2\right)^+
%\end{align}

The above limiting behaviors are all clear from the graphs.
We can now apply these to more complicated limits where relevant.
We now consider some examples.
\begin{align*}
\lim_{x\to\infty}\tan^{-1}(\ln x)&\overset{\tan^{-1}\infty}{\lllongeq}
                                  \frac{\pi}2,\\
\lim_{x\to0^+}\tan^{-1}(\ln x)&\overset{\tan^{-1}(-\infty)}{\lllongeq}
                                   -\frac{\pi}2.
\end{align*}
When looking at these combinations of functions, it is important
to look ``inside-out'' to see how the ``inner'' and ``component''
functions behave in the limit.  To emphasize this, we will sometimes
illustrate computations such as the above in ways such as the
following:
\begin{align*}
\lim_{x\to\infty}\tan^{-1}(\underbrace{\ln x}_{\begin{array}{cc}
     \downarrow\\ \infty\end{array}})=\frac{\pi}2,\\
\lim_{x\to0^+}\tan^{-1}(\underbrace{\ln x}_{\begin{array}{cc}
     \downarrow\\ -\infty\end{array}})=-\frac{\pi}2.
\end{align*}
Yet another way to compute these limits is through substitution.
For instance, in this last limit we can write set 
$u=\ln x$, so that $x\to0^+\implies u\to-\infty$ properly
(see Definition~\ref{DefOfX->A=>U->BProperly}, 
page~\pageref{DefOfX->A=>U->BProperly} and its subsequent discussion) and 
so 
$$
\lim_{x\to0^+}\tan^{-1}(\ln x)=\lim_{u\to-\infty}\tan^{-1}u=\frac{-\pi}{2}.$$

Note that neither $\lim_{x\to-\infty}\tan^{-1}(\ln x)$ nor
$\lim_{x\to0^-}\tan^{-1}(\ln x)$ exist, since
the natural logarithm is not defined as $x$ ``travels the
path'' prescribed by the limit.

Many of the limits which occur naturally in subsequent sections
are compositions of functions as above.  We will consider a few
more of these before moving on to limiting behaviors of
other combinations (products, quotients, etc.) of functions.

\section{L'H\^opital's Rule\label{LHRSection}}

In this section we introduce a very powerful technique for 
computing limits of the forms $0/0$ and $\infty/\infty$,
and their variations.\footnote{%%%
%%% FOOTNOTE
By variations we mean that, for instance, $0^+/0^-$, $\infty/(-\infty)$,
etc.
%%% END FOOTNOTE
}

 


\section{Other Indeterminant Forms}

