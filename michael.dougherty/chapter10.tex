%\setcounter{page}{1100}
\chapter{Series of Constants\label{FirstSeriesChapter}}
In this chapter we consider ``infinite sums,'' such as
\begin{equation}
\sum_{n=1}^\infty a_n=\underbrace{a_1}_{n=1}+\underbrace{a_2}_{n=1}
  +\underbrace{a_3}_{n=3}+\cdots.
\label{SeriesFirstIntroducedEquation}\end{equation}
The ``sum'' above begins with $a_1$, but we will often begin
with a term $a_0$, or $a_2$, etc.  It is not the beginning
terms which determine if we can in fact compute such a
sum, but rather it is the infinite ``tail'' of the series.
This is reasonable because we can always, in principle, 
add as many terms together as we like, so long as there are
finitely many of them.  As with other calculus concepts,
the tool which breaks the finite/infinite barrier is limit.
Indeed, to make sense of a sum such as (\ref{SeriesFirstIntroducedEquation}),
we consider the $N$th {\it partial sum},
\begin{equation}
S_N=\sum_{n=1}^na_n=a_1+a_2+\cdots+a_{N-1}+a_N,
\label{PartialSumFirstIntroducedEquation}\end{equation}
and then look at the {\it sequence} $S_1,S_2,S_3,\cdots$,
i.e.,
\begin{align*}
S_1&=a_1,\\
S_2&=a_1+a_2,\\
S_3&=a_1+a_2+a_3,\\
S_4&=a_1+a_2+a_3+a_4,\end{align*}
and so on.  To determine if (\ref{SeriesFirstIntroducedEquation})
makes sense is then considered (by definition) equivalent
to determining the behavior of the sequence $\left\{S_N\right\}_{N=1}^\infty$.
We say the series (\ref{SeriesFirstIntroducedEquation})
{\it converges} to $L\in\Re$ if and only if $S_N\longrightarrow L$
as $N\to\infty$.

In a few cases we will actually be able to compute a simple
formula for $S_N$, and thus be able to compute the 
series by taking $N\to\infty$.  However, in many cases
we cannot find a compact formula for $S_N$.  In those cases
we have to develop other methods for determining if
the series converges at all, and if so, how to approximate
the value of the series with as much precision as we require,
by determining how large we require $N$ to be so that
we can approximate the full series by $S_N$.

In this text we will take more steps than most other texts in developing
the theory of series, since this topic is the source of much
confusion for students.  Indeed we devote this entire chapter
to the topic of series of constant terms, leaving nonconstant
terms for their own chapter.  The concepts are 
intuitive---perhaps deceptively so---but require practice so that,
for example, one can recognize when and where to apply a particular
test of convergence or divergence.

In the next chapter
we will look at functions defined by series
$$f(x)=\sum_{n=0}^\infty a_n(x-a)^n
  =a_0+a_1(x-a)+a_2(x-a)^2+\cdots.$$
In fact most functions we have dealt  with in this text can be written
in the form above, at least on some open intervals, so such functions are
very important theoretically. In fact
these functions also have use in applications, when we do not know
any other method of representing a particular function arising
from a particular application except by such a series.


\newpage
\section{Series and Partial Sums}
As mentioned in the introduction to this chapter, the convergence
of a series is defined  as equivalent to the convergence of its 
partial sums.  
For convenience, we will define the $N$th partial sum
to be the sum of all terms of the underlying sequence up to
the term whose subscript is $N$.  Thus if the sequence is
series is $\sum_{n=k}^\infty a_n$, with
underlying sequence $\left\{a_n\right\}_{n=k}^\infty$, then
\begin{equation}
S_N=\sum_{n=k}^N a_n=a_k+a_{k+1}+\cdots+a_N.\end{equation}
So for a series $a_0+a_1+a_2+\cdots$, the partial
sum $S_N=a_0+a_1+\cdots+a_N$ would actually have $N+1$ terms,
though we will still call it the $N$th partial sum. (Of course
if $N<k$ we do not define an $N$th partial sum.)

\bex Consider the series
$$\sum_{n=1}^\infty \frac{(-1)^n}{n^2+1}.$$
Find the first five partial sums.

\underline{Solution}: We do this directly:
\begin{alignat*}{3}
S_1&=\sum_{n=1}^1\frac{(-1)^n}{n^2+1}
  &&=\frac{(-1)^1}{1^2+1}&&=\frac{-1}2\\
S_2&=\sum_{n=1}^2\frac{(-1)^n}{n^2+1}
  &&=\frac{(-1)^1}{1^2+1}+\frac{(-1)^2}{2^2+1}
  &&=\frac{-1}2+\frac1{5}=-\frac{3}{10}=0.3\\
S_3&=\sum_{n=1}^3\frac{(-1)^n}{n^2+1}
  &&=\frac{(-1)^1}{1^2+1}+\frac{(-1)^2}{2^2+1}+\frac{(-1)^3}{3^2+1}
    &&=-\frac{3}{10}+\frac{-1}{10}=\frac2{10}=0.2\\
S_4&=\sum_{n=1}^4\frac{(-1)^n}{n^2+1}
  &&=\frac{(-1)^1}{1^2+1}+\frac{(-1)^2}{2^2+1}+\frac{(-1)^3}{3^2+1}
         +\frac{(-1)^4}{4^2+1}
  &&=S_3+\frac{1}{17}=\frac{44}{170}\approx0.2588235294\\
S_5&=\sum_{n=1}^5\frac{(-1)^n}{n^2+1}&&
    =S_4+\frac{(-1)^5}{5^2+1}
   &&=\frac{44}{170}+\frac{-1}{26}\approx0.220361991.
\end{alignat*}
\eex
Note we used the simple recursion relationship for partial sums
of a series: given a series
$\ds{\sum_{n=k}^\infty a_n}$, and $N\ge k$ we have
\begin{equation}
S_{N+1}=S_N+a_{N+1},\label{RecursionSn+1=Sn+An+1}
\end{equation}
that is,
\begin{align*}
S_{N+1}&=\sum_{n=k}^{N+1}a_n=\underbrace{a_k+a_{k+1}+\cdots+a_N}_{S_N}
                +a_{N+1}=\sum_{n=k}^Na_n+a_{N+1}\\
       &=S_N+a_{N+1},\text{ q.e.d.}\end{align*}
In a later section 
we will see that the series in the above example does in fact
converge, though we can only approximate its exact value
by computing $S_N$ for large values of $N$.

\subsection{Telescoping Series}
Telescoping series do occur on occasion, but the main reason
they are included in most calculus textbooks is that their
partial sums simplify in nice ways, leaving us able to compute
their limits and thus the whole series.  Indeed, the behavior
of telescoping series is unusually ``nice''---rivaled only by
that of the much more important
geometric series we will see later in this section---and therefore
well-suited for early examples of the general notion of series.


The simplest type of telescoping series is one in which
the terms added are themselves sums of two terms, constructed
in such a way that there is cancellation such as the following:
\begin{align}
\sum_{n=1}^\infty a_n&=\sum_{n=1}^\infty
\left[b_n-b_{n-1}\right]\label{SimpleAbstractTelescopingSeries}\\
&=\left(b_1-b_0\right)+\left(b_2-b_1\right)+\left(b_3-b_2\right)
+\left(b_4-b_3\right)+\cdots.\notag\end{align}
After a careful examination of the terms which appear in 
(\ref{SimpleAbstractTelescopingSeries}), it seems that
all cancel except for $-b_0$. However we must be even more
careful since there are infinitely many terms we are claiming
we can cancel.  The correct approach is to carefully examine the partial 
sums:
\begin{align*}
S_1&=b_1-b_0,\\
S_2&=\not{b_1}-b_0+b_2-\not{b_1}=b_2-b_0,\\
S_3&=\not{b_1}-b_0+\not{b_2}-\not{b_1}+b_3-\not{b_2}=b_3-b_0,\\
S_4&=\not{b_1}-b_0+\not{b_2}-\not{b_1}+\not{b_3}-\not{b_2}
   +b_4-\not{b_3}=b_4-b_0,\end{align*}
and so on, whereby we can conclude that, for this simplest type
of example (\ref{SimpleAbstractTelescopingSeries}), we have
\begin{equation}
S_n=b_n-b_0.\label{SnForSimpleAbstractTelescopingSeries}
\end{equation}
Now such a series will therefore converge if and 
only if $\left\{b_n\right\}_{n=1}^\infty$ converges.
If $b_n\longrightarrow B\in\Re$ as $n\to\infty$, then
by (\ref{SnForSimpleAbstractTelescopingSeries}) we have
$S_n\longrightarrow B-b_0$, whence
$\sum_{n=1}^\infty\left[b_n-b_{n-1}\right]=B-b_0$.

More complicated telescoping series also occur, though the
basic idea is that the partial sums can be written
in such a way that all but a few terms found in the partial sums
eventually cancel,
and where we can compute the limits of those terms which do not.\footnote{%%%
%%% FOOTNOTE
It is interesting to visualize why the term {\it telescoping}
is used to describe such a series.  
One of the {\it Webster's} dictionaries defines 
the intransitive verb form of {\it telescope} as follows:
\begin{quote}
{\it to slide together, or into something else, in the manner of the
tubes of a jointed telescope.}\end{quote}
For another example, a ``telescoping antenna'' comes to mind.
The reader should keep such images in mind as we consider 
so-called telescoping series.
%%% END FOOTNOTE
} Rather than memorizing the sample telescoping
forms (\ref{SimpleAbstractTelescopingSeries})
and (\ref{SnForSimpleAbstractTelescopingSeries}), it is better to 
consider each example separately, writing out the terms of $S_N$
for enough values of $N$ that the pattern emerges.
\bex Consider the series $\ds{\sum_{n=1}^\infty\left[\frac1{n+1}-\frac1n\right]
}$.  Compute the form of each partial sum $S_N$ (as a function of $N$), and
the value of the series if it converges.

\underline{Solution}: We will write out a few partial sums longhand, 
from which the pattern will emerge.
Indeed, all but two terms will cancel in each of the following.
\begin{alignat*}{2}
S_1&=\left[\frac12-1\right]&&=\frac12-1,\\
S_2&=\left[\frac12-1\right]+\left[\frac13-\frac12\right]
     &&=\frac13-1,\\
S_3&=\left[\frac12-1\right]+\left[\frac13-\frac12\right]
   +\left[\frac14-\frac13\right]&&=\frac14-1,\\
S_4&=\left[\frac12-1\right]+\left[\frac13-\frac12\right]
   +\left[\frac14-\frac13\right]+\left[\frac15-\frac14\right]
   &&=\frac15-1.
\end{alignat*}
From this we do indeed see a pattern in which 
$$S_N=\frac1{N+1}-1.$$
Taking $N\to\infty$, we see $S_N=\frac1{N+1}-1\longrightarrow0-1=-1$,
and so we conclude that the series converges to $-1$, i.e.,
$$\sum_{n=1}^\infty\left[\frac1{n+1}-\frac1n\right]=-1.$$
\eex

Sometimes we need to do a little more work to detect a
telescoping series, and its formula for $S_N$.
Note that the general term of the added sequence terms,
namely $\left[\frac1{n+1}-\frac1n\right]$,
in our series above looks like a partial fraction decomposition
if the variable is $n$.  For that reason, when
the general term can be written in a PFD, the series may 
in fact be telescoping.  This is the case with the following 
example. 

\bex Consider the series $\ds{\sum_{n=2}^\infty\frac1{n^2-1}}$.
Compute a general formula for the $N$th partial sum $S_N$,
and compute its limit, if $S_N$ converges, thereby computing the series.

\underline{Solution}: Note first that there is no $S_1$ here.
That said, the technique which we will use for this
is to first look at the partial fraction decomposition (PFD)
for $\frac1{n^2-1}$.  Of course we need the denominator factored,
giving us the form
$$\frac1{n^2-1}=\frac1{(n+1)(n-1)}=\frac{A}{n+1}+\frac{B}{n-1}.$$
Multiplying by $(n+1)(n-1)$ in the second equation then gives us
$$1=A(n-1)+B(n+1).$$
Now we use the usual methods for computing the coefficients $A$ and $B$:
\begin{alignat*}{2}
\underline{n=1}:&\qquad &1&=B(2)\implies\boxed{B=\frac12}\\
\underline{n=-1}:&\qquad&1&=A(-2)\implies\boxed{A=-\frac12}.
\end{alignat*}
From this we can rewrite our series
$$\sum_{n=2}^\infty
 \left[\frac{-1/2}{n+1}+\frac{1/2}{n-1}\right]
 =\sum_{n=2}^\infty\left[\frac12\left(\frac{-1}{n+1}+\frac1{n-1}\right)\right].
$$
There is no $S_1$, so we begin with $S_2$.
\begin{alignat*}{2}
S_2&=\frac12\left(\frac{-1}{3}+1\right)
       &&=\frac12\left(\frac{-1}{3}+1\right),\\
S_3&=\frac12\left(\frac{-1}{3}+1\right)
       +\frac12\left(\frac{-1}4+\frac12\right)
       &&=\frac12\left(\frac{-1}{3}+1+\frac{-1}4+\frac12\right),\\
S_4&=\frac12\left(\frac{-1}{3}+1\right)
       +\frac12\left(\frac{-1}4+\frac12\right)
       +\frac12\left(\frac{-1}5+\frac13\right)
       &&=\frac12\left(1+\frac12-\frac14-\frac15\right),\\
S_5&=S_4+\frac12\left(\frac{-1}6+\frac14\right)
       &&=\frac12\left(1+\frac12-\frac15-\frac16\right),\\
S_6&=S_5+\frac12\left(\frac{-1}7+\frac15\right)
       &&=\frac12\left(1+\frac12-\frac16-\frac17\right),\\
S_7&=S_6+\frac12\left(\frac{-1}8+\frac16\right)
       &&=\frac12\left(1+\frac12-\frac17-\frac18\right)
\end{alignat*}
By this point a pattern has clearly  emerged:
$$S_N=\frac12\left(1+\frac12-\frac1N-\frac1{N+1}\right),$$
and so 
$S_N\longrightarrow \frac12\left[1+\frac12-0-0\right]=\frac12\cdot\frac32
         =\frac34$ as $N\to\infty$.
We can thus conclude that
$$\sum_{n=2}^\infty\frac1{n^2-1}
  =\sum_{n=2}^\infty
         \left[\frac12\left(\frac{-1}{n+1}+\frac1{n-1}\right)\right]
  =\frac34.$$
\eex
\bex Find $S_N$ and discuss the convergence (or divergence)
     of the series
$$\sum_{n=0}^\infty\left[\sqrt{n+1}-\sqrt{n}\right].$$

\underline{Solution}:
\begin{alignat*}{2}
S_0&=\left[\sqrt1-\sqrt0\right]&&=\sqrt1-\sqrt0,\\
S_1&=\left[\sqrt1-\sqrt0\right]+\left[\sqrt2-\sqrt1\right]&&=\sqrt2-\sqrt0,\\
S_2&=\left[\sqrt1-\sqrt0\right]+\left[\sqrt2-\sqrt1\right]
   +\left[\sqrt3-\sqrt2\right]&&=\sqrt3-\sqrt0,\\
S_3&=\left[\sqrt1-\sqrt0\right]+\left[\sqrt2-\sqrt1\right]
   +\left[\sqrt3-\sqrt2\right]+\left[\sqrt4-\sqrt3\right]
   &&=\sqrt4-\sqrt0,
\end{alignat*}
and so on, so that 
$$S_N=\sqrt{N+1}-\sqrt0=\sqrt{N+1}\longrightarrow\infty
\text{ as }N\to\infty.$$
Thus the series diverges (to infinity, to be more descriptive).
\eex
Note that we could simplify our earlier expressions for $S_N$,
since for instance $\sqrt0=0$, $\sqrt1=1$ and $\sqrt4=2$, but
to do so would more likely obscure the pattern of cancellation.

\subsection{Geometric Series}
The class of series considered  here is arguably the most important
we will encounter.  Many important series analyses
depend upon how a particular series compares to, or mimics the
behavior of, an appropriately chosen geometric series.
As with the telescoping series, the geometric series
is one for which we can actually compute a general formula
for $S_N$, from which we can tell if the series converges,
and if so compute its sum.

What makes a series $\sum a_n$ geometric is that there exists a constant 
$r\in\Re-\{0\}$ such that
\begin{equation}
(\forall n)\left[\frac{a_{n+1}}{a_n}=r\right].\label{RForGeometricSeries}
\end{equation}
In other words, such a series can be defined recursively
by $a_{n+1}=r\cdot a_n$.  (Note that this is equvialent to 
$a_n=r\cdot a_{n-1}$, so long as $a_{n-1}$ is defined.)
Put more colloquially, a geometric series is one in which
the next term added is a constant multiple of the
term immediately prior to it.  Examples of geometric series follow:
\begin{itemize}
\item $\ds{\sum_{n=0}^\infty\left(\frac12\right)^n
   =1+\frac12+\frac14+\frac18+\cdots}$ \qquad($r=1/2$),
\item $\ds{\sum_{n=2}^\infty \frac6{5^n}=\frac6{25}
          +\frac6{125}+\frac6{625}+\frac6{3125}+
       \cdots}$ \qquad($r=1/5$),
\item $\ds{\sum_{n=1}^\infty\frac{2(-1)^n}{3^n}
   =-\frac23+\frac29-\frac2{27}+\frac2{81}-\cdots}$\qquad
     ($r=-1/3$),
\item $\ds{\sum_{n=1}^\infty\frac1{3^{2n}}
   =\frac19+\frac1{81}+\frac1{729}+\frac1{6561}+\cdots}$
   \qquad($r=1/9$).
\end{itemize}
Note that this last series can be 
rewritten $\sum_{n=1}^\infty\frac1{9^n}$,
or even $\sum_{n=0}^\infty\left[\frac19\cdot\left(\frac19\right)^n\right]$.
In fact, unlike the
telescoping series, every geometric series can be written in the 
same form, namely\footnote{%%
%%% FOOTNOTE
With geometric series, it is understood that ``$r^0$'' represents
$1$, even though technically this is only correct if $r>0$.
In each general setting in which we follow the convention
that $r^0$ is defined to be $1$ (regardless of the sign of $r$), 
we will remark on this point.
%%% END FOOTNOTE
}
\begin{equation}
\sum_{n=0}^\infty \alpha r^n
  =\alpha+\alpha r+\alpha r^2+\alpha r^3+\cdots,
\label{GeneralGeometricSeries}
\end{equation}
where
\begin{align}
&\alpha\text{ is the {\it first term} of the series, and}\\
&r\text{ is the {\it constant ratio}, } a_{n+1}/a_n.
\end{align}
In the examples above, the first terms
are $\alpha=1,6/25,-2/3,1/9$ respectively.
Each of the series above can be rewritten
in $\Sigma$-notation in the form (\ref{GeneralGeometricSeries}),
starting with $n=0$.
For instance, the third series above can be rewritten,
using $\alpha=-2/3$ and $r=-1/3$, as
$$
\sum_{n=1}^\infty\frac{2(-1)^n}{3^n}
   =-\frac23+\frac29-\frac2{27}+\frac2{81}-\cdots
   =\sum_{n=0}^\infty \frac{-2}3\left(\frac{-1}3\right)^n.
$$
In fact, once we know a series is geometric (that is,
that $a_{n+1}=r\cdot a_n$ for each $n$), all we need
to do is to identify $\alpha$ and $r$, and we can write
the series in the exact $\Sigma$-notation form 
(\ref{GeneralGeometricSeries}).

\bex Write the series $4+\frac23+\frac19+\frac1{54}+\frac1{324}+\cdots$
in the form (\ref{GeneralGeometricSeries}).

\underline{Solution}: Though perhaps not immediately obvious,
in fact each successive term is $\frac16$ times its immediate
predecessor.  The first term is $4$.  We translate these two facts as
$\alpha=4$ and $r=\frac16$, and so
this series is the same as the series 
$$\sum_{n=0}^\infty 4\cdot\left(\frac16\right)^n.$$
\eex

As with telescoping series, a geometric series
allows for a simple formula for $S_N$.
To use the formula, however, we need to make two assumptions:
\begin{enumerate}
\item that the series is already written in the form
      $\ds{\sum_{n=0}^\infty\alpha r^n=\alpha+\alpha r+\alpha r^2+
                 \alpha r^3+\cdots}$, and
\item that $r\ne1$.
\end{enumerate}
As we have seen, the first requirement is easy enough to accomplish: we 
need only identify $\alpha$ (the first term in the geometric 
series) and $r$.  The second requirement is for technical reasons
we will encounter momentarily.  We do not loose much
in assuming $r\ne1$, since in the case $r=1$ 
the series is simply $\alpha+\alpha+\alpha+\cdots$,
which is clearly a divergent series if $\alpha\ne0$, and trivial if
$\alpha=0$.\footnote{%%%
%%% FOOTNOTE
We will not generally consider the case $\alpha=0$ because
it is trivial, and because we cannot identify a unique $r$.
Indeed, if $\alpha=0$, then any geometric recursion is valid,
but our original method of defining $r$, namely (\ref{RForGeometricSeries})
on page~\pageref{RForGeometricSeries}, is undefined if 
$\alpha=0$.
%%% END FOOTNOTE
}
Now we state our theorem.
\begin{theorem} For a geometric series $\ds{\sum_{n=0}^\infty\alpha r^n
=\alpha+\alpha r+\alpha r^2+\alpha r^3+\cdots}$, 
assuming $r\ne1$, we have
\begin{equation}
S_N=\frac{\alpha\left(1-r^{N+1}\right)}{1-r}.\label{SnForGeometric}
\end{equation}
\end{theorem}
\begin{proof}
The usual method of proof of (\ref{SnForGeometric}) is to exploit
the geometric nature of the series in the following way:
$$\begin{array}{lccccccccccc}
&S_N&=&\alpha&+&\alpha r&+&\alpha r^2&+&\cdots&+&\alpha r^N\\
\implies
&r\cdot S_N&=&\alpha r&+&\alpha r^2&+&\alpha r^3&+&\cdots&+&\alpha r^{N+1}\\
\hline
\implies&(1-r)S_N&=&\alpha&+&0&+&0&+&0&-&\alpha r^{N+1}\end{array}$$
In the first line we wrote the definition of $S_N$.  In the next
line we multiplied that equation by $r$.  In the third line,
the second line is subtracted from the first. In doing so,
the terms $\alpha r,\alpha r^2,\cdots,\alpha r^{N}$ cancel,
leaving only $\alpha-\alpha r^{N+1}$ on the right-hand side. 
This gives us
$$(1-r)S_N=\alpha\left(1-r^{N+1}\right).$$
Since we are assuming $r\ne1$, we can divide by $1-r$ and get
(\ref{SnForGeometric}), as desired.
\end{proof}

To utilize (\ref{SnForGeometric}), one needs to know $\alpha$, $r$ and $N$.
Note that $N$ is not the number of terms, but the highest power
of $r$ which occurs.  In fact there are $N+1$ terms added
to arrive at $S_N$, since the first is $\alpha r^0$.

\bex Consider the series $1+\frac12+\frac14+\frac18+\cdots$.
     Find the sum of the first 9 terms.

\underline{Solution}: What we are seeking here is 
$S_8=\frac{\alpha\left(1-r^{8+1}\right)}{1-r}$,
where $\alpha=1$ and $r=\frac12$.  Thus
$$S_8=\frac{1\left[1-\left(\frac12\right)^9\right]}{1-\frac12}
     =\frac{1-\frac1{512}}{\frac12}
     =\frac{1-\frac1{512}}{\frac12}\cdot\frac{512}{512}
     =\frac{512-1}{256}=\frac{511}{256}=1.99609375$$
\eex

The formula (\ref{SnForGeometric}) also works when $r<0$.
\bex Consider the series $1-\frac12+\frac14-\frac18+\cdots$.
Find the sum of the first 9 terms.

\underline{Solution}: Again we want $S_8$, but while $\alpha=1$ as before,
here we have $r=-1/2$.

$$S_8=\frac{1\left[1-\left(\frac{-1}2\right)^9\right]}{1-\left(-\frac12\right)}
     =\frac{1-\frac{-1}{512}}{\frac32}
     =\frac{1+\frac1{512}}{\frac32}\cdot\frac{512}{512}
     =\frac{512+1}{256\cdot3}
     =\frac{513}{256\cdot3}
     =\frac{171}{256}=0.66796875.$$





\eex

\bex Suppose one deposits into an account (without interest)
     one penny (\$0.01) on the first
     day of a month, then deposits two pennies (\$0.02)
     the next day, four pennies the next, and so on,
     each day depositing twice what was deposited the
     day before.  How much money is in the account
     after the first week (7 payments), 
     second week, third week, and thirty-first
     day?

\underline{Solution}: This is the same as asking for
     partial sums of the series $0.01+0.02+0.04+0.08+\cdots$.
     This is a geoemtric series (\ref{GeneralGeometricSeries}) with
     $\alpha=0.01$ and $r=2$.  Here we have to be careful
     about $N$, since after the first day $N=0$, after the 
     second $N=1$, etc.  Now we compute the total deposit after
     \begin{itemize}
     \item 1 week, i.e., 7 days, we have $N=6$ and
        $$S_6=\frac{0.01\left[1-2^7\right]}{1-2}
             =\frac{0.01\left[1-2^7\right]}{-1}
             =0.01\left(2^7-1\right)=0.01(127)=1.27.$$
     \item 2 weeks, i.e., 14 days, we have $N=13$ and (continuing the
           pattern above)
         $$S_{13}=\frac{0.01\left[1-2^{14}\right]}{1-2}
                 =0.01\left(2^{14}-1\right)=0.01(16383)=163.83.$$
     \item 3 weeks, i.e., 21 days, we have $N=20$ and
         $$S_{20}=\cdots=0.01\left(2^{21}-1\right)=0.01(2,097,151)
                 =20,971.51$$
     \item 31 days, so we have $N=30$, and
         $$S_{30}=\cdots=0.01\left(2^{31}-1\right)
                 =0.01(2,147,483,647)=21,474,836.47.$$
      \end{itemize}
\eex
This latest example illustrates that, when $r>1$, the
function $N\mapsto S_N$ is essentially exponential.
Indeed, {\it as a function of} $N$, 
$$S_N=\frac{\alpha}{1-r}\left[1-r^{N+1}\right]
=\frac{\alpha}{1-r}+\frac{-\alpha\cdot r^{N+1}}{1-r}     
=\frac{\alpha}{1-r}+\left[\frac{\alpha}{r-1}\cdot r\right]
\cdot r^{N}=A+Br^N,$$
where $A=\frac{\alpha}{r-1}$ and $B=\frac{\alpha r}{r-1}$.
Thus as a function of $N$, $S_N$ is basically
a vertical translation of an 
exponential growth $Br^N$, assuming again that $r>1$.\footnote{%%%
%%% FOOTNOTE
If $r\in(0,1)$ we get a translation
of exponential decay; if $r\in(-1,0)$ we get a kind of
``damped  oscillation''; if $r=-1$ we get steady oscillation;
and if $r<-1$ we get a growing oscillation.  Details are left to
the reader.
%%% END FOOTNOTE
}
This partially explains why some use the term ``geometric growth''
when referring to exponential growth.

\subsection{Convergence/Divergence in Geometric Series}
Now we look at necessary and sufficient conditions for a 
geometric series to converge.  If a given geometric
series does converge, we compute its sum.  Our result
is the following:

\begin{theorem}
For a geometric series $\ds{\sum_{n=0}^\infty \alpha r^n
   =\alpha+\alpha r+\alpha r^2+\alpha r^3+\cdots}$, where $\alpha\ne0$,
\begin{enumerate}
\item the series {\bf converges} if and only if $|r|<1$, i.e., $r\in(-1,1)$;\\
      equivalently, the series {\bf diverges} if and only if $|r|\ge 1$, i.e., 
      $r\in(-\infty,-1]\cup[1,\infty)$.
\item if $|r|<1$, then the series converges to $\ds{\frac{\alpha}{1-r}}$.
\end{enumerate}
\end{theorem}
Restated, the geometric series converges to $\frac{\alpha}{1-r}$ if $|r|<1$,
and diverges otherwise.

\begin{proof} The proof requires some care, as the various cases 
contain their own technicalities.  
\begin{itemize}
\item Case $r=1$. In such a case, it is not diffiult to see (we just
      count the terms!) that
$$S_N=\sum_{n=0}^N\alpha=(N+1)\alpha\longrightarrow\infty
       \qquad\text{as }N\to\infty.$$
Thus $r=1$ gives a divergent series.
\item Case $r=-1$. In such a case, we have
$$\sum_{n=0}^\infty\alpha(-1)^n=\alpha-\alpha+\alpha-\alpha+\cdots,$$
and so 
$$S_N=\left\{\begin{array}{rl}
\alpha,&\qquad\text{if }n\text{ is even,}\\
0,&\qquad\text{if }n\text{ is odd.}\\
\end{array}\right.$$
In other words, $\left\{S_N\right\}_{N=0}^\infty=\alpha,0,\alpha,0,\alpha,0,
   \cdots$, which is clearly a divergent sequence, i.e., 
   the series itself is divergent (by definition).\footnote{%%%
%%% FOOTNOTE
Recall that the convergence of the series is defined by the convergence
of the (sequence of) partial sums.
%%% END FOOTNOTE
}
\item Case $|r|>1$. Here we can use the formula for the partial sums:
$$S_N=\frac{\alpha\left(1-r^{N+1}\right)}{1-r}.$$
Now there is only one term which is not a fixed constant, and so
the convergence of this expression depends upon only the
convergence that,  $r^{N+1}$-term.  Clearly if $r>1$, 
this is an exponential growth, and diverges.  For the general case
$|r|>1$, we get that\footnote{%%%
%%% FOOTNOTE
This follows from
continuity of the function $x\mapsto|x|$ giving us the ``$\implies$.''
See Theorem~\ref{ContFunctCarriesSequenceToF(Limit)},
page~\pageref{ContFunctCarriesSequenceToF(Limit)}.
%%% END FOOTNOTE
}
\begin{equation}
r^{N+1}\text{ converges }\implies \left|r^{N+1}\right|\text{ converges }
\iff|r|^{N+1}\text{ converges.}\label{AConvergenceArgument1}\end{equation}
But for $|r|>1$, we have $|r|^{N+1}$ diverges, so with
the contrapositive of (\ref{AConvergenceArgument1}) we have
\begin{align*}
|r|>1&\implies |r|^{N+1}\text{ diverges }\implies
r^{N+1}\text{ diverges } \\
&\implies S_N=\frac{\alpha\left(1-r^{N+1}\right)}{1-r} \text{ diverges }
\iff \sum_{n=0}^\infty\alpha r^n\text{ diverges.}\end{align*}
\item Case $|r|<1$.
Again we look at the variable part of the formula for $S_N$.
It is enough to show that $|r|<1\implies r^{N+1}$ converges.
One method is to use the sandwich theorem. In the
argument below, note that $|r|<1\implies|r|\in(0,1)
\implies |r|^{N+1}\longrightarrow0$.  The relevant
sandwich theorem application is then (as $N\to\infty$):\footnote{%
%%% FOOTNOTE
Recall that for any $x\in\Re$, we have $-|x|\le x\le |x|$.
%%% END FOONOTE
}
\begin{center}
\begin{pspicture}(0,2)(8.7,4.5)
\rput[l](0,4){$\underbrace{-|r|^{N+1}}=-\left|r^{N+1}\right|$}
  \rput[l](3.8,4){$\le$}
\rput[l](4.6,4){$r^{N+1}$}
  \rput[l](6.2,4){$\le$}
\rput[l](7,4){$\left|r^{N+1}\right|=\underbrace{|r|^{N+1}}$}
\psline{->}(.63,3.5)(.63,2.3)
\psline{->}(8.99,3.5)(8.99,2.3)
 \rput(.63,2){0}
 \rput(9,2){0}
\end{pspicture}
\end{center}
Thus $|r|<1\implies r^{N+1}\longrightarrow 0$ as $N\to\infty$.
We can conclude that
$$|r|<1\implies
S_N=\frac{\alpha\left(1-r^{N+1}\right)}{1-r}
   \longrightarrow \frac{\alpha(1-0)}{1-r}=\frac{\alpha}{1-r}
\text{ (as }N\to\infty).$$
\end{itemize}
This completes the proof.
\end{proof}
The implication above is worth repeating in a summarized form:
\begin{equation}
|r|<1\implies\sum_{n=0}^\infty\alpha r^n=\frac{\alpha}{1-r}.
\label{|r|<1ImpliesSumOfGeometricSeries}\end{equation}
\bex Here are some series computations using 
   the theorem and (\ref{|r|<1ImpliesSumOfGeometricSeries}).
\begin{itemize}
\item $\ds{\sum_{n=0}^\infty 2\left(\frac13\right)^n
   =\frac2{1-\frac13}=\frac2{\frac23}=2\cdot\frac32=3}$.  
   \qquad($\alpha=2$, $r=\frac13$.)
\item $\ds{\sum_{n=0}^\infty0.99^n=\frac1{1-0.99}=\frac1{.01}=100}$.
   \qquad($\alpha=1$, $r=0.99$.)
\item $\ds{\sum_{n=0}^\infty 1.01^n}$ diverges.\qquad
   ($\alpha=1$, $r=1.01$ so $|r|>1$, and the series diverges.)
\item $\ds{\sum_{n=2}^\infty\frac1{3^n}=\frac{\frac19}{1-\frac13}
    =\frac{\frac19}{\frac23}=\frac19\cdot\frac32=\frac16}$.
    \qquad(First term is $\alpha=\frac19$, $r=\frac13$.)
\item $\ds{1-\frac12+\frac14-\frac18+\frac{1}{16}-\cdots
       =\frac1{1-\left(\frac{-1}2\right)}=\frac1{\frac32}=\frac23}$.
     \qquad($\alpha=1$, $r=-\frac12$.)
\item $\ds{\sum_{n=1}^\infty e^{-n}=\frac{e}{1-\frac1e}
           =\frac{e}{1-\frac1e}\cdot\frac{e}e=\frac{e^2}{e-1}}$.
     \qquad($\alpha=e$, $r=\frac1e$.)
\item $\ds{\sum_{n=1}^\infty\frac5{3^{2n}}
       =\sum_{n=1}^\infty\frac5{9^n}=\frac{5/9}{1-\frac19}
       =\frac{5/9}{8/9}=\frac58.}$
       \qquad($\alpha=\frac59$, $r=\frac19.$)
\end{itemize}



\eex







\newpage
\begin{center}
\underline{\Large{\bf Exercises}}\end{center}

\begin{multicols}{2}

\begin{enumerate}
\item Show that the following series can be
      written as a telescoping series, and discuss
      its convergence: $$\sum_{n=1}^{\infty}\ln\left(\frac{n}{n+1}\right).$$
\item Give an alternative proof of the formula (\ref{SnForGeometric})
      for the partial sums of geoemtric series.
      For this new proof, begin with the formula for $S_N$ as
      in the original proof (page~\pageref{SnForGeometric}),
      and then multiply by $(1-r)$, noting how the right-hand side
      simplifies.  (See also page~\pageref{aN-bN})
      



\end{enumerate}
\vfill
\end{multicols}
\eject


\section{Elementary Test for Divergence}
Because it is the exceptional case (e.g., geometric, telescoping)
that we can actually find a compact
formula for $S_N$, we have to develop other tests for the convergence
or divergence of series.  There will be several such tests, and
which particular test or tests are expeditious and conclusive 
will vary from series to series.  We explore the first of those
tests in this section.  We start with a theorem.

\begin{theorem} Given a series $\ds{\sum_{n=k}^\infty a_n}$.
If the series converges, then $a_n\longrightarrow0$ as $n\to\infty$:
\begin{equation}
\left[\sum_{n=k}^\infty a_n\text{ converges }\right]\implies
\left[\lim_{n\to\infty}a_n=0\right].
\label{SeriesConvergesImpliesTermsShrinkToZero}
\end{equation}
\label{SeriesConvergesImpliesTermsShrinkToZeroTheorem}\end{theorem}
It should be intuitively clear that, if we are going to ``add up''
infinitely many terms, and have the sums approach a finite number, 
then the  
terms we are adding are going to have to shrink to zero.  
The proof uses the fact
that $S_n\to L\implies S_{n-1}\to L$, the latter limit occurring
``one step behind'' the former, but occurring nonetheless
since $n\to\infty\implies n-1\to\infty\implies S_{n-1}\to L$.
Now to the proof.

\begin{proof}
Suppose $\ds{\sum_{n=k}^\infty a_n}$ converges, i.e.,
$\ds{\sum_{n=k}^\infty a_n=L}$ for some $L\in\Re$
(and in particular $L$ is finite).
Then by definition
$$S_N\longrightarrow L\text{ as }N\to\infty.$$
Now recall $S_{n}=a_n+S_{n-1}$, so that $a_n=S_n-S_{n-1}$.
Taking $n\to\infty$ we get
$$a_n=S_n-S_{n-1}\longrightarrow L-L=0,\qquad\text{q.e.d.}$$
\end{proof}
Note that it was important that $L$ be finite in the limit
computation above.  For instance, if $S_n\to\infty$, we
would have $a_n=S_n-S_{n-1}$ giving $\infty-\infty$-form 
(which is indeterminant) as
$n\to\infty$.

Again, the intuition behind the theorem is that, in order to
be able to add infinitely many terms---one at a time in the
sense that we compute $S_k$, $S_{k+1}$, $S_{k+1}$, etc., 
and look for a trend towards $L$---the terms that we add,
i.e., $a_k$, $a_{k+1}$, $a_{k+2}$, etc., have to shrink
if the partial sums are to approach a finite number $L$.
In fact, the form of the theorem which we use is the
contrapositive.  Recall
the logical equivalence $P\longrightarrow Q
\iff(\sim Q)\longrightarrow(\sim P)$.\footnote{%%%
%%% FOOTNOTE
To be sure, here $P\longrightarrow Q$ is read, 
``$P$ implies $Q$.''  The symbol ``$\sim$'' is still
the ``not,'' or logical negation, operator.
The symbol ``$\iff$'' stands in for logical equivalence.
%%% END FOOTNOTE
}
In this case, $P$ is the statement that the series converges (to 
a finite number $L$), while $Q$ is the statement that $a_n\to0$.
The contrapositive for of 
Theorem~\ref{SeriesConvergesImpliesTermsShrinkToZeroTheorem}
is our main result in this section, and we dub that result
{\it Elementary Test for Divergence}, or ETFD:
\begin{theorem} {\bf (ETFD)}
If it is {\bf not} the case that $a_n\to0$, then $\ds{\sum_{n=k}^\infty a_n}$
diverges.
Put symbolically,
\begin{equation}
a_n\not{\!\!\to0}\implies\sum_{n=k}^\infty a_n\text{ diverges.}
\label{ETFDequation}
\end{equation}
\label{ETFD}\end{theorem}

\begin{proof} It is enough to say that this is the contrapositive of 
Theorem~\ref{SeriesConvergesImpliesTermsShrinkToZeroTheorem},
and therefore also true.
One way to write this symbolically is the following.
$$\underbrace{\sum_{n=k}^\infty a_n\text{ converges }
\longrightarrow \left(a_n\to 0\right)}_{\text{always true by
Theorem~\ref{SeriesConvergesImpliesTermsShrinkToZeroTheorem}}}
\iff
\underbrace{
\left[\sim\left(a_n\to 0\right)\right]
\longrightarrow
\sum_{n=k}^\infty a_n\text{ diverges}}_{\text{statement of
Theorem~\ref{ETFD}}}.$$
The statement on the right must be always true (a tautology),
since it is equivalent to the statement on the left,
which---being the statement of  
Theorem~\ref{SeriesConvergesImpliesTermsShrinkToZeroTheorem}---is
itself a tautology.  The statement on the right being a 
tautology means that it stands alone, with ``$[\ ]\implies\sum$''
replacing ``$[\ ]\longrightarrow\sum$,'' q.e.d.
\end{proof}
This theorem is undoubtedly one of 
the most misunderstood and misapplied results
in all of Calculus I and II.  It is as important to understand
what it does not say, as it is to understand what it says.
The theorem says that if the terms of a series do not 
shrink to zero, then the series must diverge.  This is 
``elementary,'' hence the name ``Elementary Test for Divergence.\footnotemark''
\footnotetext{%%
%%% FOOTNOTE
Other texts call this the {\it $n$th term test}, or 
the {\it $n$th term test for divergence}, for reasons which should be
clear.  The author is not aware if any other text calls
it the elementary test for divergence.  However, both names
are descriptive enough that one is not likely to be misunderstood
in conversation or written reports if using either name for
this result.}
%%% END FOOTNOTE

But it is not as comprehensive as one might think.  Afterall,
if the terms of a series {\it do} shrink to zero, the 
theorem is silent!  (Therein lies the unfortunately
very common mistake made by Calculus Students.)
To emphasize this we look at the following examples.

\bex Discuss what Theorem~\ref{ETFD} has to say about
the series \newline
(a) $\ds{\sum_{n=1}^\infty\frac{n}{n+1}}$;\hfill
(b) $\ds{\sum_{n=1}^\infty\frac1{n+1}}$\hfill
(c) $\ds{\sum_{n=1}^\infty\cos\frac1{n}}$;\hfill
(d) $\ds{\sum_{n=1}^\infty\sin\frac1{n}}$;\hfill
(e) $\ds{\sum_{n=1}^\infty\sin\frac1{n^2}}$.

\underline{Solution}: 
\begin{enumerate}[(a)]
\item $\ds{\frac{n}{n+1}\longrightarrow1\ne0
        \overset{\text{\rm ETFD}}{\implies}
           \sum_{n=1}^\infty \frac{n}{n+1}}$ diverges.
\item $\ds{\frac1{n+1}\longrightarrow}$.  The 
      ETFD is inconclusive.
\item $\ds{\cos\frac1n\longrightarrow\cos0=1\ne0
        \overset{\text{\rm ETFD}}{\implies}
        \sum_{n=1}^\infty\cos\frac1{n}}$ diverges.
\item $\ds{\sin\frac1n\longrightarrow\sin0=0}$.
       The ETFD is inconclusive.
\item $\ds{\sin\frac1{n^2}\longrightarrow\sin0=0}$.
       The ETFD is inconclusive. 
\end{enumerate}
\eex

Looking closely at the symbolic statement of ETFD given
in (\ref{ETFDequation}), we see that there is never an 
implication of convergence.  Indeed, the test either concludes 
divergence, or is inconclusive.  This is a very quick but incomplete test,
which can only detect divergence in certain (still common) circumstances,
namely that $a_n\not{\!\!\to}0$.

Indeed, in (a) and (c) above, ETFD gave us divergence.
However, it said nothing in (b), (d) and (e), as the
``if'' part of the theorem was not true.  In fact,
of these three in which ETFD is silent---(b), (d) and (e)---it
turns out that $(b)$ and $(d)$ are divergent, while
$(e)$ is convergent.  The methods to see this are introduced
in later sections.  An example from the exercises
in the last section gives a case where we can in fact
show that it is possible that $a_n\to0$, but the series diverges:
\bex Consider the series 
$\ds{\sum_{n=1}^\infty\ln\left(\frac{n}{n+1}\right)}$.
Determine if it converges or diverges.

\underline{Solution}: First we note ${\ln\left(\frac{n}{n+1}\right)
\longrightarrow\ln1=0}$ as $n\to\infty$.  Thus ETFD is inconclusive.\footnote{
%%% FOOTNOTE
Some textbooks would write that the test ``fails.''
That seems a bit strong.  It is merely inconclusive, so we need
to look deeper at the particular series and perhaps employ some other test
which will be conclusive.}
%%% END FOOTNOTE

Looking closer at this series, we should eventually notice that
it is telescoping.  This becomes clear
if we rewrite it:
$$\sum_{n=1}^\infty\left[\ln(n)-\ln(n+1)\right]
  =\left[\ln 1-\ln 2\right]+\left[\ln-\ln3\right] 
    +\left[\ln 3-\ln 4\right]+\cdots.$$
It is not difficult to see that $S_N=\ln1-\ln N=-\ln N$,
and thus
$$S_N=-\ln N\longrightarrow-\infty\text{ as }N\to\infty.$$
Since the partial sums diverge, by definition so does the series.
\eex 

On the other hand, there are series which converge.  However the ETFD
is not powerful enough to ever prove it.  Rather than risking confusion
by elaborating this point further, we will end the discussion here.

\newpage\section{The Integral Test for Convergence/Divergence}



