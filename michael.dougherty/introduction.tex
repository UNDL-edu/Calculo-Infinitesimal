\chapter{Introduction}
The discovery of calculus was one of the most important and exciting
achievements in the history of intellectual progress.
Virtually every field which deals with quantities
has benefited from calculus.
It allowed Sir Isaac Newton to derive the laws of planetary motion,
Albert Einstein to derive relativity, 
many an economist to model and analyze
market variables, and countless other achievements.
In particular, it is reasonable to estimate that physics would
be centuries behind its present maturity were it not for the
availability of calculus.

Despite the technicalities involved in the lofty fields already
mentioned, the fundamental principles of calculus are quite
accessible, especially now that the subject
has been distilled into a more  coherent form in the
passing of 
centuries since its initial discoveries.  Generations of 
researchers and authors have refined the presentations
to be understandable to motivated college students with
a variety of  interests. Some larger universities
have separate
calculus courses specifically for majors in business, agriculture,
forestry, and even English, as well as the mainstays
of mathematics, engineering and science.
%It is a priviledge of every calculus author
%and professor to continue this tradition of refinement, and
%to pass this knowledge to the next generation.
Calculus is the marquee mathematical subject
for many of these programs of study, particularly science 
and engineering.
Its importance can not be overstated.
Even the algebra-trigonometry courses at our
institutions have been fashioned largely to groom
students for eventual study of calculus. 

So what is calculus?  The short answer is that it is 
the field of mathematics which deals with change,
both instantaneous and cumulative.  Respectively,
this means calculus is mainly---but certainly not exclusively---interested 
in solving the following two problems:
\begin{enumerate}[(1)]
\item given algebraic relationships among variables, compute 
      their rates of change with respect to each other;
\item given the rates of change of variables with respect to
      each other, find the algebraic relationships
      among the variables.
\end{enumerate}
Indeed the second is simply the first in reverse.  For a simple,
though abstract example, consider the following questions:
\begin{enumerate}[(i)]
\item If we know the position $s$ of an object at every time $t$,
      can we know the velocity $v$ of the object at every time $t$?
\item If we know the velocity $v$ of an object at every time $t$,
      can we know the position $s$ of the object at every time $t$?
\end{enumerate}
Now velocity is ultimately
just the measure of how the position $s$ is changing with  time $t$,
so indeed (i) and (ii) are examples of (1) and (2) above.
The answer to (i) is ``yes,'' and the answer to (ii) is ``almost.''
In fact, for (ii) we need some more information, like 
where the object is (i.e., $s$) for a particular time $t$,
and then we can usually ``pin down'' the position $s$ for all time $t$.  

Problems of type (1) are part of the {\it differential calculus},
also known as calculus of {\it derivatives}.  Problems of type
(2) are part of {\it integral calculus} where, perhaps
predictably, we will compute many {\it antiderivatives}.
Problems of this second type tend to be more (sometimes much more)
difficult than problems of the first type.

Before we even begin to work in the differential or integral calculus, we
will need some preliminaries.  We will begin with a 
chapter on symbolic logic so that we can employ that language throughout
the text.  Next we exercise that logic on some algebraic preliminaries.
Our first preliminaries specific to calculus are the concepts
of continuity
and limits, which together form much of the
theoretical foundation of the calculus, and so we will spend
considerable effort on these.  The bulk of our
work is then contained in the
chapters on differential calculus and integral calculus.
A final major topic is series, which finishes our work here.
This last topic will require much of its own foundational 
development, but is a very important aspect of the classical
calculus.  Within that study are the answers to questions
such as how a calculator can find $\sin78^\circ$ to ten digits
of accuracy, but that is only a very small sample of the usefulness
of that theory.


Throughout the text we will see other applications of 
limits, derivatives, antiderivatives and series, and we will
explore as many of those as reasonable for a text of this scope.
The development of the analytical tools is our main goal.  The
student well-versed in the mechanics of those tools will surely
(and, it is hoped, easily) find
numerous other uses for the methods developed here.


\chapter{Reading this Book}

The main body of the book is organized into Chapters 1, 2, 3, and so on.
Chapters are then organized into sections, so for
instance Chapter~1 is
divided into Sections 1.1, 1.2, 1.3, etc.
Most sections correspond to the amount of material a college professor
should be able to introduce in a one or two hour lecture, not including
time spent answering homework questions.
Sections are themselves sometimes divided, so Section 5.2
(i.e., the second section of Chapter 5) may
be divided into Subsections 5.2.1, 5.2.2, and so on.
This is done to maintain a type of outline format,
with each chapter, section and subsection given a
title so that it is clear which topic (or subtopic, or sub-subtopic)
is being developed at any particular location in the text.  
Because there are numerous tangential points and clarifications
to be made, numbered footnotes---some
quite lengthy---are employed extensively so that the regular flow
of the text need not be interrupted.\footnote{%%
%%% FOOTNOTE
This is an example of a footnote.  
%%% END FOOTNOTE
}

While all this heierarchy and
numbering may at first seem excessive, it
has become standard practice, and does help to mark where
a particular topic is developed.
As mentioned in the preface, this book attempts to be a stand-in for
an actual professor lecturing at a chalkboard in a calculus class.
It is reasonable for students to expect the professor to write 
the main points on the board in an outline form.  Textbook styles,
however, differ from the usual lecture-notes outline form,
which includes major headings (here chapters), Roman numerals (sections),
upper-case Latin letters (subsections), Persian-Arabic numerals (definitions,
examples, steps in the general explanation), and so on.  But the 
basic idea of an outline, branching from general to specific, 
is the same.  Of course it is not uncommon for the professor to stand
away from the board from time to time and verbally elaborate on
various points, or to mention relevant external topics which do not
fit neatly into the flow of the outline and indeed may distract
from the strictly-defined purposes of the course if included in
the written outline.  Nonetheless such points can do much to 
clarify the material, especially by importing relevance to outside
topics.  The footnotes provide the author of the textbook
the same chance,
to figuratively step away from the main flow of the text, 
clarify the discussion and connect it to the rest of the world.

This text also uses numerical labels for equations, theorems, 
corollaries, definitions,
tables and figures.  When a particular equation warrants, 
it is given  a sequentially-assigned label based
upon the chapter number for easy reference.
For instance, Equation (7.23) would be the twenty-third such equation
in Chapter~7.  Theorems, definitions and figures are 
similarly numbered.  (The exception is that each chapter's
footnote labels reset to 1, so that Chapter~1 has footnotes
numbered 
1, 2, 3, etc., as do Chapters~2, 3, and so on.)
These are all standard,
formal styles of labeling found in much of the technical literature.
This calculus text presents perhaps an ideal opportunity for 
a first introduction to its extensive use.

The text, while not strictly linear, 
is mostly cumulative, with new topics constantly referring to
earlier topics.  Thus the material should be read in the
order it is presented, with few exceptions possible
(and then best chosen by an instructor).

It may not always be possible for the student to master each topic
as it is presented.  Nonethless, 
it is very important to work as hard as possible
to become as proficient as possible in 
the topics as they are first encountered.  This may sometimes
require a very slow and deliberate approach to the exercises
and explanations.  However, 
it may occasionally be necessary to ``table'' a 
confusing topic, in order to return to it after seeing
how it fits into the greater scheme.
This is natural, and in fact even a topic seemingly
``mastered'' will undoubtedly
benefit from a revisit. Still, strong efforts on all
topics will continually pay returns as one's 
intuition for calculus as a whole is nurtured.


There are numerous comments within the text explaining
the importance of the various topics, and relating how
difficult particular topics have historically proven
to students.  Again, all topics are important, but such
comments are provided to indicate where particularly strong effort
may be required.  Comments regarding
common mistakes are also common within the text.



\chapter{Table of Greek Letters}
We often use Greek letters in this text, as is standard for
technical writing.  In calculus, $\Delta$, $\delta$, $\epsilon$ and
$\Sigma$  have particularly special roles, 
as do $\theta$ and $\phi$, among others, in trigonometry.
We will also use other Greek letters when they are either 
appropriate stand-ins for their English counterparts, or when
we want them to be conspicuous  in mathematical
expressions.  Finally, a reasonably informed student
in any technical discipline is expected to eventually know 
all the members of the Greek alphabet.  For these reasons
we include a table of Greek letters below.


%\begin{center}
%{\Large Table of Greek Letters}
%\bigskip
%\end{center}
\vfill

\begin{alignat*}{3}
&A&\qquad&\alpha&\qquad\qquad&\text{alpha}\\
&B&&\beta&&\text{beta}\\ 
&\Gamma&&\gamma&&\text{gamma}\\
&\Delta&&\delta&&\text{delta}\\
&E&&\epsilon&&\text{epsilon}\\
&Z&&\zeta&&\text{zeta}\\
&H&&\eta&&\text{eta}\\
&\Theta&&\theta&&\text{theta}\\
&I&&\iota&&\text{iota}\\
&K&&\kappa&&\text{kappa}\\
&\Lambda&&\lambda&&\text{lambda}\\
&M&&\mu&&\text{mu}\\
&N&&\nu&&\text{nu}\\
&\Xi&&\xi&&\text{xi}\\
&O&&o&&\text{omicron}\\
&\Pi&&\pi&&\text{pi}\\
&P&&\rho&&\text{rho}\\
&\Sigma&&\sigma&&\text{sigma}\\
&T&&\tau&&\text{tau}\\
&Y&&\upsilon&&\text{upsilon}\\
&\Phi&&\phi&&\text{phi}\\
&X&&\chi&&\text{chi}\\
&\Psi&&\psi&&\text{psi}\\
&\Omega&&\omega&&\text{omega}
\end{alignat*}
\vfill\eject
