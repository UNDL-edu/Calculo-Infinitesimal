\chapter{Preface}





Calculus is  notable in that any competent mathematician
with at least a masters  degree, and many with just a strong bachelors,
should be fluent enough in the subject to passably teach the courses, since
calculus and calculus-descendant studies form such a large
part of their training.  Perhaps consequently, there are almost as
many opinions about how it should be taught as there are
people teaching it.  A fuzzy and somewhat artificial division
into ``traditional'' and ``reform'' camps has been 
the rage for some fifteen years now, though neither seems
able to define their own camp very well, let alone the other camp.
Textbooks are often sold labeled as traditional or
reform, with a growing number giving homage
to both. Actually this is not difficult since the camps
seem to really disagree mostly on emphasis.  When given
a choice, professors pick whatever textbook most closely
resembles their own philosophy, and make up for the differences
using the lectures.  When not given a choice,
many professors {\it still}  give the nearly the
same lectures since again, they
understand calculus very thoroughly and have their own ideas
about how to best make sense of it to college students.
If it were not for the scale of such a project, in both writing
and dealing with the actual publishing aspects,
there would surely be many more---and more diverse---calculus 
textbooks available to reflect these opinions.

Into this mix I submit this textbook, hoping it will
appeal to like minded instructors.  It grew out of
my own ideas about what was right, and what was lacking in
the textbooks I learned and taught from.  This text has been in
the works conceptually since my own graduate school days, when I
was privileged to twice teach summer Calculus I at Purdue
University using the very ambitious text by Richard Hunt.
When I later, as an assistant professor,
found that his second edition would not be published, 
I searched in vain for an alternative that was of a similar
spirit and could find none. 
Some seven years after teaching those courses at Purdue, and
never being totally comfortable with the texts available in the
market,
I finally began
putting pen to paper and fingers to keyboard on this
 work.\footnote{%
%%% FOOTNOTE
This is not to say that this textbook is a clone of Richard Hunt's.
I only claim that his was my first inspiration, and none
of the other available texts seemed to me as inspired with
a vision as did his.  I have heard Hunt's strategy described
as ``sneaking in some real analysis.''  This seems an apt description
of his.  I do the same, though to some extent I wonder who
sneaked the real analysis {\it out} of the calculus.  
But my mission is not just to return to that, as the rest of the preface
here will explain.
%%% END FOOTNOTE 
}  
In talking to colleagues over the years, I 
am led to believe several do share visions similar to my own.
To them I offer this as at least a step in their direction.
I hope that its format will prove consistent with
the goals of those colleagues, and that the text will
fill their needs. 
How well we teach, and how well the students learn calculus are
functions of our own enthusiasm.  I
hope that the approach is fresh and energizing to
some of my fellow 
calculus instructors, high school as well as college,
who have been looking for a textbook
with some of the elements offered here.



Incidentally, the title of this textbook is not meant to exclude
students who are in fields of study other than mathematics.  Indeed, it 
is hoped that anyone who pursues calculus for whatever reason will
do so as a ``student of mathematics.''  It is not uncommon for
a gathering of individuals to contain some who may be considered
``students of Shakespeare'' but never formally completed
a Literature or related 
major. The phrase simply means that such individuals care enough
about the subject to  take personal time to 
examine it thoughtfully and continuingly, and
to become respectably articulate  in the subject, at least when
among one's peers.
On the other hand, one can study the mechanics of, say,  calculus without
considering its more technical details or its conceptual
content.  To do so is akin to learning to quote Shakespeare's
plays without actually understanding the themes, or knowing the contexts.
.  With calculus
as with Shakespeare such lack of understanding can lead to trouble,
in the form of embarrassment in the case of Shakespeare, and 
perhaps more catastrophic consequences in the case of calculus
applied to real-world problems.
The more of a ``student'' one is in a particular subject, the
more trouble can be avoided and the more the subject can be 
enjoyed and enriching.  Of course this textbook is intended to
be thorough for those whose major field of study is Mathematics.
However, it is hoped that 
{\it Calculus for Students of Mathematics} will inspire each
reader---whose study may be any field---to become, for a
while if not for a lifetime, a true ``student of mathematics.''

\bigskip
\noindent{\LARGE\bf What is different about this text?}
\bigskip

At the risk of appearing trite, I claim that this text is actually {\it
meant to be read. }
In my experience many texts are too sparse, and others too concise, in their
explanations and it is up to the lecturer to fill in the
details or give alternate explanations more fit for student consumption.  
This text is an attempt to reverse the contemporary roles of
mathematics textbook and professor,
so that the professor does not have to sprint through the 
details but can, in good conscience, give the highlights or
supplement with his own particular insights,
knowing that the students have a complete treatment in the textbook.
(Perhaps only in mathematics does the professor 
traditionally have the role as a more complete fountain of
information than the text.  Imagine a biology or history
professor giving quantitatively more details
in lecture than contained in the readings!)

Much effort has been made for the text to be self-contained. 
A reasonably
prepared and {\it dedicated} student should be able to learn
enough calculus independently with this text to be able to 
solve all but the most challenging problems contained here.
The text is naturally
more verbose than most, and is peppered with cross references
and footnotes.  This will be a different style for many
students, but one which is worth learning how to read.
\bigskip


\noindent{\bf Pedagogically Linear Order}
\bigskip

This is as opposed to theoretically linear order.
I have spent a great deal of thought on the order of topics,
and have experimented with various orders extensively with my
own classes.  I have found that a few simple changes can
make profound differences in the rate in which material is
absorbed.

While this text is more theoretical than most, it was written 
with an awareness that there is a momentum to learning.
Too many starts and stops in the development can dissipate energy
from a calculus class.  For that reason, it is sometimes 
better to show the final, ``working'' theory than risk
bogging down in the preliminary theorems,
with or without proofs.  For instance, many texts will develop the
natural logarithm as a definite integral, show that it works like
a logarithm should and therefore must be a logarithm of some
kind, and then call some theorem on inverse functions---a
topic often painfully developed in its own, barely motivated
section---to finally derive the function $e^x$ and its
algebraic and calculus properties.
I am in good company in deferring
the theoretical development until the reader
is well-practiced with both these functions, and then giving the axiomatic
theoretical development as an ``icing on the cake.'' The speed
in which the computational skills are developed is greater, and
the theoretical development is better-appreciated.  

I also develop all of the derivative rules in the same chapter
(Chapter~\ref{DerivativeChapter}).  A reasonable argument can be made
that the exponential, logarithmic and arc-trigonometric
functions should be introduced later,
in between other calculus topics, so students can first further digest the
earlier differentiation rules through applications.  Though that
is a standard pedagogical technique, instead I attempt
to exploit the momentum of learning the differentiation rules
so that they can be completely dispatched, and then reinforced through
use in the chapters on applications.




Similarly, after all the differentiation rules are developed, 
and a chapter has been devoted to applications of derivatives,
I devote two chapters  on indefinite integrals before using them
in Riemann Sum-motivated applications.  The first of these
integration chapters exhausts all
the functions introduced earlier, in substitution-type settings.
The second is my advanced integration technique chapter which
builds upon the momentum of the first integration chapter.
After these two chapters comes the chapter on Riemann Sums and
applications of definite integrals.
This approach allows the
text to maintain the momentum from the derivative chapters,
uninterrupted by Riemann Sums until they can be immediately
motivated by the applications, and the student should be able to handle any
integral which might arise, since by then
the student has accomplished
a considerable amount of integration.  While this approach is
actually less ``gentle'' for the development of antidifferentiation
techniques, it has less stops and starts, and should help the student
retain those skills throughout the applications.


\bigskip

\noindent{\bf Continuity before limits.}
\bigskip

One reason I feel comfortable developing a topic completely---without 
interrupting to allow the reader to ``sleep on it''---is that I 
front-load the text with
rigor.  Especially in the topics of limits and continuity, my
path is perhaps not the quickest through these topics, but
rather the path that will give the best hope for a comprehensive
understanding.  It is coincidentally also the
most linear for the theoretical development.

In particular I put continuity before
limits, defining both in their own rights,
using $\epsilon$-$\delta$ definition.  I strongly believe that Calculus
loses much rigor when we omit $\epsilon$-$\delta$ (even if 
students do not always understand these proofs as much
as we would like), and that this omission causes much
{\it ad hoc} explanation in the rest of our limit discussions
(which can then barely be called ``developments'').  
However, I realize that this is not a real analysis text,
and so I only require
the student to give $\epsilon$-$\delta$ proofs for the 
first continuity section where I think it is best motivated
(for instance by reference to tolerances), 
after which theorems ensure
we never need to use them again in the exercises. My section on
continuity on intervals has a couple of intuitive 
topological theorems\footnote{Topological  proofs are omitted to avoid
the need to define connectedness and compactness.}
on the images of intervals under continuous functions,
from which I can easily state the intermediate value
theorem and the extreme value theorem, using the former
to give a method for solving
polynomial and rational inequalities.  I then
have several limit sections to take care of all the
first semester techniques, including separate sections
for vertical and 
horizontal asymptote phenomena, rather than lumping them into
one or two sections as many texts do.

Compared to other texts, the extensiveness of this particular
chapter is perhaps the most innovative feature of the textbook.
It is my sincere hope that it will help solve many of the difficulties
associated with teaching these two topics.

\bigskip
\noindent{\bf Symbolic logic included.}
\bigskip

To help with the rigor and communication of ideas, I
include an early introduction to symbolic logic, which I then
mix into the prose throughout the rest of the text.
This is done for many reasons.  First, it adds clarity through
precision of the arguments.  Second, the symbols naturally illustrate
the logical ``flow" of the arguments.  Finally,
it is my hope
that this will be a hook for many students who
have had difficulty relating abstract mathematics to 
everyday life, since the symbolic logic arguments have
common sense appeal.  Learning about logical 
equivalence is particularly useful in 
calculus since many theorems are stated in one form,
used in another equivalent form, and possibly proved in
still another form.  Without some logical sophistication, 
such a discussion can be very confusing for calculus students.
In particular, the contrapositive and the difference
between implication and equivalence are stressed, 
as these can be problematic throughout one's college
studies and beyond.
Of course it is hoped that the discussion of logic will help
the dedicated student sharpen his or her own analytical
skills in all disciplines, mathematical or
otherwise, where logical argument is required.


Studying symbolic logic has several other advantages.
For instance, college calculus courses are often populated
by a mix of students who had some exposure to calculus in
high school, while the rest had none.  This often leads
to overconfidence in the former group and anxiety in the latter.
Beginning with symbolic logic evens the playing field at the 
start, and sends a clear message to those who had calculus before
that college calculus will be different, while giving both
the novice and the former high school calculus student
an opportunity to build the momentum to study calculus at
a college level.

The logic also sets a tone for a generally more abstract text
than most.  I feel justified in this since, after all, 
the underlying principles
are abstract and understanding these is crucial for
proper application.  In this spirit I include, for instance, the
axiomatic definition of the real numbers (though  again,
I am aware this is not 
supposed to be a real analysis text), in order that correct algebraic
operations can be discussed in more exalted language.
The discussion includes the least upper bound property
so that, much later, convergence of sequences and series will not
need to be explained in an {\it ad hoc} manner.
Using notation from logic, I  give a
somewhat different review of algebra and trigonometry
than what students may be used to, again to get them
thinking about these things from a more sophisticated and
hopefully fresher perspective.  

\bigskip

\bigskip\noindent{\bf Applications.}
\bigskip

For applications I 
stick more to physics examples,  and only occasionally inject biological or
social scientific examples.  I believe physics has the clearest connection 
to calculus, and offers the best motivations for its study.
In fact, I do not use the tangent line slope as
my main motivation for the derivative, but instead use velocity
(vis-\`a-vis position).  
I believe velocity is initially more intuitive
to more students.  The fact that the derivative is graphically
the slope of a tangent line is a very convenient device 
of course, and I exploit it extensively, but too many students
walk away uninspired from calculus thinking it is all
about tangent lines, and not instead about change
(instantaneous {\it and} cumulative).

\bigskip\noindent{\bf Other differences.}
\bigskip


Also different is the fact that this text is in black and white,
further reinforcing a more abstract spirit.
This may be more a matter of taste, but I believe
there is a place for such a text and that fancy,
four-color illustrations can be distracting from the
main themes (not to
mention far more expensive to produce!).  

The entire textbook is typeset in \LaTeX, using
the \LaTeX \ book style, with
graphics
handled by the \LaTeX \  {\tt pstricks} package.
Several other \LaTeX \ packages were also used, mainly
for modifying the format.  No graphics were imported
but are all generated using \LaTeX \ code from these packages.

\bigskip
\noindent{\LARGE\bf Acknowledgments}
\bigskip

First I would like to thank anyone who reads any part of this book.
I wrote it for you!  Even if you do not read it cover-to-cover, I
very much appreciate your interest.  And I would like very much to 
hear back, regardless of your opinion.

For helping to make this work possible,
I am most grateful to my wife of ten years, Hung-Chieh Chang
(a.k.a. Joy Dougherty), who never showed me any doubt in her mind
that this work would
eventually be finished, and who put up with the seemingly countless hours
I was bonding with several computers to finish this book. 
Her opinions on things mathematical, pedagogical and artistic were
invaluable.

I am also very grateful to Southwestern Oklahoma State University,
particularly all in the Department of Mathematics, for their encouragement
and support for this project.  Few institutions would allow
junior faculty so much freedom to undertake a calculus textbook.
In particular I was given some release time 
and was allowed to use photocopied excerpts of the work-in-progress
in my calculus courses.  This was both risky and expensive for
the department, and I sincerely hope my colleagues find that
it was worth it.

This text is strongly influenced by James Phelan, 
my own high school calculus
instructor. Though he also taught at a local college,
he did not teach specifically to prepare us for the
Advanced Placement test---as so many high school instructors
are directed to do---but
instead taught what {\it he} thought was a solid course.  
(Still, with much of the thanks to him
I passed the AP test with 5/5 anyhow, much to his  
satisfaction!)  His mission was to make us literate in mathematics
in as many ways as possible while teaching as much calculus as
we could absorb as high school seniors.
My desire to be a ``student of mathematics,'' that is, to 
acquire some mathematical sophistication, came first from
him.\footnote{When I was interviewing for my first
position as an assistant professor, and was asked about my teaching
philosophy, I explained
that the students had been reading too much Stephen King mathematics.
I wanted them to think they were reading Tolstoy!  
(A friend later told me, ``Yes, and in Russian!'')
I got the job.}  
%As mentioned before, this work is also influenced by 
%my years of experience teaching calculus as a teaching assistant
%and professor.  Admittedly, it also reflect a little bit of 
%faith that our students will rise to the level of foundational 
%mathematical sophistication presented here.

The professor who, by example, convinced me to 
become a mathematics major was Shih-Chuan Cheng at Creighton
University.  The coherence and sophistication of his lecture notes
first convinced me of the beauty of mathematics and probably
constitute the single greatest influence on the style of this text.
Other coursework under John Mordeson and James Carlson at
Creighton convinced me that there is a style of learning 
mathematics which stresses depth and coherence first
and breadth second, which is far from the ``sink or swim''
approach, and from which students can hold their own
among their peers from top-tiered schools.  In other words,
if you have the depth, the breadth can come later.\footnote{%
The opposite approach is that students will somehow
acquire depth from knowing a breadth of topics.
This is arguable, but for most students I personally believe
the breadth first approach will breed more confusion and
anxiety, and thus be less likely to produce understanding.
However, a reasonable argument can be made that student
understanding of mathematics generally goes from particular
to general.  My counterargument is that students taking
15--18 hours of college courses may not have the time
and mental energy to make all the connections (or ``synthesize''
in education-speak) we hope
for.  Still, I encourage the reader to have an open mind
on the subject, and indeed I sometimes include ``drill''
exercises in the problems.  But in the explanations and examples I 
attempt to consistently work from a ``depth first'' philosophy.}

I must also thank all who made desktop publishing of
mathematics possible.  In particular, those in the \LaTeX \ 
developers community who brought us not only \LaTeX \ but some amazing
supplemental packages, particularly {\tt pstricks, multicols, enumerate}
and {\tt caption}, and for Adobe for inventing Postscript and
PDF standards and, as importantly, keeping them ``open'' so
the \TeX \ community could exploit them for producing 
publication-quality mathematics with \TeX \ and \LaTeX.  
With these things I was able to produce
textbook quality copy to give to my students in class,
as well as online, when the text
was still in preliminary form, and to present  camera-ready
copy to a publisher.  This ability has been 
enormously helpful to the development of this text.



